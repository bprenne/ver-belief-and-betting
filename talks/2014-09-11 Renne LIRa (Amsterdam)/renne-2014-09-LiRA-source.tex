
%\includeonlyframes{current}

\usepackage{color}

\usepackage[normalem]{ulem}

\usepackage{amsmath,stmaryrd,amssymb,eurosym,mathrsfs}
\usepackage[T1]{fontenc}

%%
%% title/author info
%%
\title{Belief as Willingness to Bet}

\renewcommand\footnoterule{}
\renewcommand\footnotemark{}

\author{Bryan Renne \let\thefootnote\relax\footnotetext{\tiny \\
    Bryan's funding: Netherlands Organisation for Scientific
    Research (NWO) ``Veni'' Grant
  }}

\institute{(with Jan van Eijck)}

\date{\scriptsize
  LIRa, ILLC Amsterdam\\[.3em]
  September 11, 2014}

%%
%% theme
%%
\usetheme{Boadilla}
\usefonttheme{serif}
\useinnertheme{rounded}

\usenavigationsymbolstemplate{}
\setbeamertemplate{headline}{}
\setbeamertemplate{footline}{}
\setbeamertemplate{sidebar left}{}
\setbeamertemplate{sidebar right}{}

\setbeamercovered{
  still covered={\opaqueness<1->{0}},
  again covered={\opaqueness<1->{25}}
}

\setbeamercolor{myyellow}{fg=black,bg=yellow!40}
\setbeamercolor{mygreen}{fg=black,bg=green!30}
\setbeamercolor{myblue}{fg=black,bg=blue!20}
\setbeamercolor{mypink}{fg=black,bg=purple!30}

\newenvironment<>{myboxy}[1][\textwidth]{%
  \begin{actionenv}#2%
    \begin{center}%
      \begin{minipage}{#1}%
        \begin{beamercolorbox}[colsep=.3em,shadow=true,rounded=true]{myyellow}}
        {\end{beamercolorbox}\end{minipage}\end{center}\end{actionenv}}

\newenvironment<>{myboxg}[1][\textwidth]{%
  \begin{actionenv}#2%
    \begin{center}%
      \begin{minipage}{#1}%
        \begin{beamercolorbox}[colsep=.3em,shadow=true,rounded=true]{mygreen}}
        {\end{beamercolorbox}\end{minipage}\end{center}\end{actionenv}}

\newenvironment<>{myboxb}[1][\textwidth]{%
  \begin{actionenv}#2%
    \begin{center}%
      \begin{minipage}{#1}%
        \begin{beamercolorbox}[colsep=.3em,shadow=true,rounded=true]{myblue}}
        {\end{beamercolorbox}\end{minipage}\end{center}\end{actionenv}}
  
\newenvironment<>{myboxp}[1][\textwidth]{%
  \begin{actionenv}#2%
    \begin{center}%
      \begin{minipage}{#1}%
        \begin{beamercolorbox}[colsep=.3em,shadow=true,rounded=true]{mypink}}
        {\end{beamercolorbox}\end{minipage}\end{center}\end{actionenv}}

\usepackage{tikz}
\usetikzlibrary{trees,arrows,positioning,patterns,automata,shapes,decorations,decorations.pathmorphing}

\newcommand{\mystretch}{\renewcommand{\arraystretch}{1.5}}
\newcommand{\normalstretch}{\renewcommand{\arraystretch}{1}}

\newlength{\mysep}
\setlength{\mysep}{2.7cm}
\newlength{\mysepCloser}
\setlength{\mysepCloser}{2cm}
\newlength{\mysepFarther}
\setlength{\mysepFarther}{3cm}
\newlength{\mysepFarthest}
\setlength{\mysepFarthest}{4cm}

\newlength{\myMinsize}
\setlength{\myMinsize}{3em}
\newlength{\myBigger}
\setlength{\myBigger}{4em}
\newlength{\myNodeDistance}
\setlength{\myNodeDistance}{3cm}

\newenvironment{mytikz}[1][0em]{
\begin{tikzpicture}[>=latex,auto,node distance=\mysep,
  baseline={([yshift=#1]current bounding box.east)}]

  \normalstretch{}

  \tikzstyle{w}=[draw,circle,thick,minimum size=\myMinsize]

  \tikzstyle{e}=[draw,minimum size=\myMinsize,node distance=\myNodeDistance]
  
  \tikzstyle{every edge}=[draw,thick,font=\footnotesize]
  
  \tikzstyle{every label}=[font=\footnotesize]
  
  \tikzstyle{ev}=[anchor=west,node distance=\myNodeDistance]

  \tikzstyle{bigger}=[minimum size=\myBigger]

  \tikzstyle{closer}=[node distance=\mysepCloser]  

  \tikzstyle{farther}=[node distance=\mysepFarther]

  \tikzstyle{farthest}=[node distance=\mysepFarthest]

  \tikzstyle{l}=[node distance=\myNodeDistance]
}{\mystretch{}\end{tikzpicture}}
  
%%
%% progress shown at the beginning of each section
%%
\AtBeginSection[]
{
  \begin{frame}
    \frametitle{Overview}
    \tableofcontents[currentsection]
   \end{frame}
}

\usepackage{graphicx,tikz}
\usetikzlibrary{trees,arrows,automata,shapes,snakes,decorations,decorations.pathmorphing}

%%
%% commands, settings
%%

\newcommand{\Iff}{\Leftrightarrow}             % provable equivalence
\newcommand{\imp}{\Rightarrow}                 % implies
\renewcommand{\iff}{\Leftrightarrow}           % if and only if 

\newcommand{\bang}{{!}}                        % !
\newcommand{\qmark}{{?}}                       % ?
\newcommand{\up}{{\uparrow}}                   % upgrade
\newcommand{\Up}{{\Uparrow}}                   % lexicographic upgrade

%% mutli-agent colons:
%
% \newcommand{\col}[1]{\,{:}_{#1}\,}            % :_{#1}
% \newcommand{\cole}[1]{\,{:}_{#1}^e\,}         % :_{#1}^e
% \newcommand{\colc}[2]{\,{:}_{#1}^{#2}\,}      % :_{#1}^{#2}
% \newcommand{\colce}[2]{\,{:}_{#1}^{e,{#2}}\,} % :_{#1}^{e,#2}

\newcommand{\col}{\,{:}\,}                     % :_{#1}
\newcommand{\cole}{\,{:}^e\,}                  % :_{#1}^e
\newcommand{\colc}[1]{\,{:}^{#1}\,}            % :_{#1}^{#2}
\newcommand{\colce}[1]{\,{:}^{e,{#1}}\,}       % :_{#1}^{e,#2}

\newcommand{\ccol}{\,{::}\,}                     % :_{#1}
\newcommand{\ccole}{\,{::}^e\,}                  % :_{#1}^e

\newcommand{\colu}[1]{\,{:}^{#1}\,}
\newcommand{\coleu}[1]{\,{:}^{#1}_e\,}


\newcommand{\con}{\mathsf{con}}                % conj. of admissible flmas

\newcommand{\sem}[1]{\llbracket{#1}\rrbracket} % [[ #1 ]]

% \usepackage{rotating}
% \newcommand{\reverse}{\begin{sideways}%
%         \begin{sideways}\vspace{.1em}$\rightsquigarrow$\end{sideways}\end{sideways}}

\newcommand{\reverse}{%
           \mathrel{\raisebox{.1em}{%
           \rotatebox[origin=c]{180}{$\rightsquigarrow$}}}}

\newcommand{\from}{%
           \mathrel{\raisebox{.1em}{%
           \rotatebox[origin=c]{180}{$\to$}}}}

\newcommand{\Nat}{\mathbb{N}}  % natural numbers
\newcommand{\Rat}{\mathbb{Q}}  % rational numbers
\newcommand{\Ree}{\mathbb{R}}  % real numbers
\newcommand{\Int}{\mathbb{Z}}  % integers

\newcommand{\pow}{{\cal P}}    % powerset

\newcommand{\M}{{\cal M}}      % caligraphic model M
\newcommand{\N}{{\cal N}}      % caligraphic model N

\newcommand{\Prop}{{\bf P}}    % propositional letters

\newcommand{\Lang}{{\cal L}}   % language

\newcommand{\conv}{\check{\ }} % dual modality

\newcommand{\KB}{{\mathsf{KB}}}                        % base theory
\newcommand{\KBlt}{{\mathsf{KB}^{\mathsf{<.5}}}}       % theory for c<.5
\newcommand{\KBeq}{{\mathsf{KB}^{\mathsf{0.5}}}}       % theory for c=.5
\newcommand{\KBeqm}{{\mathsf{KB}^{\mathsf{0.5}}_{-}}}  % theory for c=.5 without (kbm)
\newcommand{\KBgt}{{\mathsf{KB}^{\mathsf{>.5}}}}       % theory for c>.5
\newcommand{\KBgeq}{{\mathsf{KB}^{\mathsf{\geq.5}}}}   % theory for c>=.5
\newcommand{\KBc}{{\mathsf{KB}^{\mathsf{>}c}}}         % theory for c

\newcommand{\KBmyc}[1]{{\mathsf{KB}^{\mathsf{>{#1}}}}} % theory for specific c = #1


\newcommand{\Ceq}{{\mathcal{C}^{\mathsf{0.5}}}}    % for c=0.5

\newcommand{\Rel}[1]{\stackrel{#1}{\Longrightarrow}}         % #1 over \Longrightarrow

\newcommand{\modelsn}{\models_{\mathsf{n}}}                  % \models_n
\newcommand{\semn}[1]{\llbracket{#1}\rrbracket_{\mathsf{n}}} % [[ #1 ]]_n

\newcommand{\modelsp}{\models_{\mathsf{p}}}                  % \models_p
\newcommand{\semp}[1]{\llbracket{#1}\rrbracket_{\mathsf{p}}} % [[ #1 ]]_p

%%
%% beginning of document
%%
\begin{document}

%%
%% title and introduction
%%
\frame{\titlepage}

\begin{frame}
  \frametitle{Probabilistic Reasoning}

  \begin{itemize}
  \item<+-> We study modal logic for \emph{qualitative} probabilistic
    reasoning:
    \begin{itemize}
    \item $K\varphi$ means $P(\varphi)=1$,

    \item $B\varphi$ means $P(\varphi)>c$ for a fixed
      $c\in(0,1)\cap\mathbb{Q}$.
    \end{itemize}

  \item<+-> Some previous work:
    \begin{itemize}
    \item<+-> Lenzen (1980): probabilistically complete logic for
      $c=\frac 12$.

    \item<+-> Halpern et al. (pre-2003, 2003 book):
      \emph{quantitative} logics.

    \item<+-> Herzig (2003): qualitative logics of action.
      \begin{center}
        $BA$ means $P(A)>P(\lnot A)$.
      \end{center}
      \begin{itemize}
      \item Equivalent to Lenzen's logic.

      \item Soundness but no completeness.
      \end{itemize}
      
    \item<+-> Kyberg and Teng (2012): ``acceptance'' of $A$ iff
      $P(\lnot A)\leq\epsilon$.
    \end{itemize}

  \item<+-> Our contributions:
    \begin{itemize}
    \item Modern reformulation of Lenzen's syntax, semantics, and
      results.

    \item New epistemic neighborhood semantics:
      \begin{itemize}
      \item Lenzen's logic is sound and complete for a sub-class
        of our models.

      \item Truth-preserving maping: probabilistic $\to$ neighborhood
        semantics.
      \end{itemize}
    \end{itemize}
  \end{itemize}
\end{frame}

\begin{frame}
  \frametitle{Our Epistemic Probability Models}
  \vspace{-1em}
  \[
  \M = (W,R,V,P)
  \]
  \vspace{-1em}
  \begin{itemize}
  \item<+-> $(W,R,V)$ is a \underline{finite} $\mathsf{S5}$ Kripke
    model, where
    \[
    [w] := \{ v\in W \mid wRv \}\enspace.
    \]

  \item<+-> $P$ is a probability measure on $\wp(W)$
    satisfying \emph{full support}:
    \[
    P(w)>0 \text{ for each } w\in W\enspace.
    \]

  \item<+-> For event $X\subseteq W$, agent assigns to $X$ at $w\in W$
    the probability
    \[
    P_w(X) :=
    \frac{P(X\cap[w])}{P([w])}
    = P(X|[w])\enspace.
    \]
    \begin{center}
      \footnotesize
      (Denominator $\neq 0$ by reflexivity \& full support.)
    \end{center}
  \end{itemize}

  \begin{uncoverenv}<+->
    \footnotesize
    \textbf{Notes.}
    \begin{itemize}
    \item Multi-agent version defined straightforwardly.

    \item Taking $\mathsf{S5}$ and full support is atypical but
      unproblematic.
    \end{itemize}
  \end{uncoverenv}
\end{frame}

\begin{frame}
  \frametitle{Our Epistemic Probability Models}
  \framesubtitle{Example 1}

  \begin{center}
    \begin{mytikz}
      %%
      %% nodes
      %%
      \node[w,label={below:$w_1$}] (w1) {$h_1$};

      \node[w,right of=w1,label={below:$w_2$}] (w2) {$h_2$};

      \node[w,right of=w2,label={below:$w_3$}] (w3) {$h_3$};

      %%
      %% edges
      %%
      \path (w1) edge[<->] node{$$} (w2);

      \path (w2) edge[<->] node{$$} (w3);
    \end{mytikz}
    \begin{eqnarray*}
      P &=& \textstyle\{w_1:\frac 36, w_2:\frac 26, w_3:\frac 16\}
    \end{eqnarray*}
  \end{center}
  \begin{itemize}
  \item Agent considers each of worlds $w_1$, $w_2$, $w_3$ possible.

  \item Agent assigns odds 3:2:1 to these worlds.

  \item Letter $h_i$ true at $w_i$ (``Horse $h_i$ wins the race in
    world $w_i$).
  \end{itemize}
\end{frame}


\begin{frame}
  \frametitle{Our Epistemic Probability Models}
  \framesubtitle{Example 2}
  \begin{center}
    \begin{mytikz}
      %%
      %% nodes
      %%
      \node[w,label={below:$w_1$}] (w1) {$h_1$};

      \node[w,right of=w1,label={below:$w_2$}] (w2) {$h_2$};

      \node[w,right of=w2,label={below:$w_3$}] (w3) {$h_3$};

      %%
      %% edges
      %%
      \path (w1) edge[<->] node{$$} (w2);

      %\path (w2) edge[<->] node{$$} (w3);
    \end{mytikz}
    \begin{eqnarray*}
      P &=& \textstyle\{w_1:\frac 36, w_2:\frac 26, w_3:\frac 16\}
    \end{eqnarray*}
  \end{center}
  \begin{itemize}
  \item<+-> Agent does not consider $w_3$ possible (relative to $w_1$
    or $w_2$):
    \[
    P_{w_1}(w_3) 
    =
    P(\{w_3\}|[w_1])
    =
    \frac{P(\{w_3\}\cap[w_1])}{P([w_1])}
    =
    \frac{P(\emptyset)}{P(\{w_1,w_2\})}
    = 0\enspace.
    \]

  \item<+-> Probability will always be evaluated w/r/t
    a world $w$ via $P_w(X)$.

  \item<+-> So even with full support, worlds can have
    (relative) probability $0$.
  \end{itemize}
\end{frame}

\begin{frame}
  \frametitle{Probabilistic Language}
  \begin{eqnarray*}
    \phi & ::= & 
    \top \mid p \mid \neg\phi \mid \phi\land\phi \mid
    t\geq 0
    \\
    t   & ::= & q \mid q \cdot P(\phi) \mid t+t
    \\
    &&
    \text{\footnotesize 
      $p\in\Prop$,
      $q\in\Rat$
    }
  \end{eqnarray*}
  \begin{itemize}
  \item Abbreviations for: Booleans, $\leq$, $>$, $<$, $=$, and linear
    (in)equalities.

  \item Standard semantics for Booleans, to which we add:
 \[
  \renewcommand{\arraystretch}{1.3}
  \begin{array}{lcl}
    \M,w\modelsp t\geq 0 & \text{iff} &
    \sem{t}_w\geq 0
  \end{array}
  \]
  \vspace{-2.3em}
  \begin{eqnarray*}
    \semp{\phi} & := &
    \{ u \in W\mid \M,u\modelsp\phi \}
    \\
    \sem{q}_w & := & q
    \\
    \sem{q\cdot P(\phi)}_w & := & 
    q\cdot P_{w}(\semp{\phi}) =
    q\cdot P(\semp{\phi}|[w])
    \\
    \sem{t+t'}_w & := &
    \sem{t}_w + \sem{t'}_w
  \end{eqnarray*}
  \end{itemize}
\end{frame}

\begin{frame}
  \frametitle{Knowledge and Belief in the Probabilistic Language}
  \vspace{-2em}
  \begin{eqnarray*}
    K\varphi &:=& P(\varphi)=1
    \\
    B^c\varphi &:=& P(\varphi)>c 
    \quad \text{for } c\in(0,1)\cap\Rat
  \end{eqnarray*}
  \vspace{-2em}
  \begin{itemize}
  \item<+-> Duals denoted by $\check K$ and $\check B^c$ (i.e.,
    $\check O$ is $\lnot O\lnot$); $B\phi$ means $B^{0.5}\phi$.

  \item<+-> Easy results:
    \begin{itemize}
    \item<+-> $K$ is $\mathsf{S5}$.

    \item<+-> Meaning of belief dual:
      \[
      \modelsp\check B^c\phi\iff(P(\phi)\geq 1-c)\enspace,
      \]
      so $\check B^{0.5}\phi$ says that $P(\phi)\geq 0.5$.

    \item<+-> $B^c$ is not normal (i.e., does not satisfy $K$):
        \begin{center}
    \begin{mytikz}
      %%
      %% nodes
      %%
      \node[w,label={below:$w_1$}] (w1) {$h_1$};

      \node[w,right of=w1,label={below:$w_2$}] (w2) {$h_2$};

      \node[w,right of=w2,label={below:$w_3$}] (w3) {$h_3$};

      %%
      %% edges
      %%
      \path (w1) edge[<->] node{} (w2);

      \path (w2) edge[<->] node{} (w3);
    \end{mytikz}

    $P=\{w_1:\frac 13, w_2:\frac 13, w_3:\frac 13\}$

    \medskip $\M\modelsp B\lnot h_1\land B\lnot h_2$ and yet
    $\M\not\modelsp B(\lnot h_1\land\lnot h_2)$.
  \end{center}
    \end{itemize}
  \end{itemize}
\end{frame}

\begin{frame}
  \frametitle{Knowledge and Belief in the Probabilistic Language}
  \vspace{-2em}
  \begin{eqnarray*}
    K\varphi &:=& P(\varphi)=1
    \\
    B^c\varphi &:=& P(\varphi)>c 
    \quad \text{for } c\in(0,1)\cap\Rat
  \end{eqnarray*}
  \vspace{-2em}
  \begin{itemize}
  \item<+-> Intuition for this notion of belief:
    \begin{itemize}
    \item Agent believes $\phi$ iff she is ``pretty sure'' of truth
      (i.e., $P(\phi)>c$).

    \item So ``less sure'' about conjunction $A\land B$ if
      $A\nsubseteq B$ and $B\nsubseteq A$.
    \end{itemize}
    
  \item<+-> This permits the ``lottery paradox'': it is consistent to
    believe
    \begin{itemize}
    \item there is some winning lottery ticket among the $n$ tickets,
      and

    \item for each ticket $i=1,\ldots,n$, ticket $i$ is not winning.
    \end{itemize}

  \item<+-> This notion of belief comes from subjective probability:
    \begin{itemize}
    \item Suppose agent believes $\varphi$ with threshold $p/q$; i.e.,
      $B^{p/q}\varphi$.

    \item Wagers $p$ dollars for chance to win $q-p$ dollars on bet
      $\varphi$ is true.

    \item Expected win:
      \vspace{-1em}
      \[
      (q-p)\cdot P(\varphi) - p\cdot(1-P(\varphi)) = q\cdot P(\varphi) -
      p\enspace.
      \]
      \vspace{-1em}
      
    \item Positive iff $P(\varphi)>p/q$, which is guaranteed by
      $B^{p/q}\varphi$.

    \item So bet is good iff has belief.
      ``Belief as willingness to bet.''
    \end{itemize}

  \item<+-> Will set aside philsophical considerations, accept
    belief here as-is.
  \end{itemize}
\end{frame}

\begin{frame}
  \frametitle{Segerberg Notation}
  \[
  (\phi_1,\dots,\phi_m\mathbb{I}\psi_1,\dots,\psi_m)\enspace,
  \]
  also written $(\phi_i\mathbb{I}\psi_i)_{i=1}^m$, abbreviates the
  formula
  \[
  K(C_0\lor C_1\lor C_2\lor \cdots\lor C_m)\enspace,
  \]
  where $C_i$ is the disjunction of all conjunctions
  \[
  d_1\phi_1\land\cdots\land d_m\phi_m\land
  e_1\psi_1\land\cdots\land e_m\psi_m
  \]
  satisfying:
  \begin{itemize}
  \item exactly $i$ of $d_k$'s are the empty string, 

  \item at least $i$ of the $e_k$'s are the empty string, 

  \item and the remaining $d_k$'s and $e_k$'s are the negation sign
    $\lnot$.
  \end{itemize}

  \begin{myboxb}<2-> \textbf{Intuitive meaning:} the agent knows that
    the number of true $\psi_k$'s is is no less than the number of
    true $\phi_k$'s.
  \end{myboxb}
\end{frame}

\begin{frame}
  \frametitle{Lenzen Scheme}
  \[
  \begin{array}{cl}
    \textstyle [
    %(\phi_1,\dots,\phi_m\mathbb{I}\psi_1,\dots,\psi_m)
    \alert<2>{(\phi_i\mathbb{I}\psi_i)_{i=1}^m}
    \land 
    \alert<3>{B^{c} \phi_1} 
    \land 
    \alert<4>{\bigwedge_{i=2}^m \check B^{c} \phi_i}
    ] \to
    \alert<5>{\bigvee_{i=1}^m B^{c}\psi_i}
    &
    \qquad\text{(Len)}
  \end{array}
  \]
  \begin{uncoverenv}<+->
    If we have that:
    \begin{itemize}
    \item<+-> the agent knows the true $\psi_k$'s are at least as many
      as the true $\phi_k$'s,
      
    \item<+-> the agent believes $\phi_1$ (with threshold $c$), and
      
    \item<+-> the remaining $\phi_k$'s are consistent with the agent's
      (thresh.-$c$) beliefs
    \end{itemize}
    \uncover<+->{then the agent believes one of the $\psi_k$'s (with
    threshold $c$).}
  \end{uncoverenv}

  \begin{myboxy}<+-> \textbf{Theorem (Lenzen for $c=\frac 12$).} (Len)
    is valid for $c\in\left(0,\frac 12\right]\cap\Rat$:
    \[
    \textstyle \modelsp [(\phi_i\mathbb{I}\psi_i)_{i=1}^m \land
    B^c\phi_1 \land \bigwedge_{i=2}^m \check B^c\phi_i] \to
    \bigvee_{i=1}^m B^c\psi_i\enspace.
    \]
    \vspace{-1em}
  \end{myboxy}

  \begin{uncoverenv}<+-> Lenzen showed:
    \begin{center}
      (Len) is key to modal-language completeness for fixed $c=\frac
      12$.
    \end{center}
  \end{uncoverenv}
\end{frame}

\begin{frame}
  \frametitle{Other Principles}
  \begin{itemize}
  \item<+-> $\not\modelsp B^c(\phi\to\psi)\to(B^c\phi\to
    B^c\psi)$.

    Belief is not closed under logical consequence (i.e., not normal).

  \item<+-> $\not\modelsp B^c\phi\to\phi$.

    Belief is not veridical.

  \item<+-> $\modelsp K\phi\to B^c\phi$.

    What is known is believed.

  \item<+-> $\modelsp\lnot B^c\bot$.

    The propositional constant $\bot$ for falsehood is not believed.

  \item<+-> $\modelsp B^c\top$.\hfill(Minimal Logic's N)

    The propositional constant $\top$ for truth is believed.

  \item<+-> $\modelsp B^c\phi\to KB^c\phi$.

    What is believed is known to be believed.

  \item<+-> $\modelsp \lnot B^c\phi\to K\lnot
    B^c\phi$.

    What is not believed is known to be not believed.
  \end{itemize}
\end{frame}

\begin{frame}
  \frametitle{Other Principles}
  \begin{itemize}
  \item<+-> $\modelsp B^c\phi\to B^cB^c\phi$.
    \hfill(4)

    Belief is positive introspective.

  \item<+-> $\modelsp \lnot B^c\phi\to B^c\lnot B^c\phi$.  \hfill(5)

    Belief is negative introspective.

  \item<+-> $\modelsp K(\phi\to\psi)\to(B^c\phi\to B^c\psi)$.

    Belief is closed under known logical consequence.

  \item<+-> $\modelsp \phi\to\psi$ implies
    $\modelsp B^c\phi\to B^c\psi$.
    \hfill (Minimal Logic's RM)

    Belief distributes over provable implication.

  \item<+-> $\modelsp B^c(\phi\land\psi)\to
    (B^c\phi\land B^c\psi)$.
    \hfill(Minimal Logic's M)

    Belief of a conjunction implies conjunction of beliefs.

  \item<+-> $\modelsp B^c\phi\to \check B^c\phi$.

    Belief is consistent: belief in $\phi$ implies
    disbelief in $\lnot\phi$.

  \item<+-> $\modelsp \check{B}^c \phi \land \check{K}(\neg \phi \land
    \psi) \rightarrow B^c (\phi \lor \psi)$ for $c\in\left(0,\frac
      12\right]\cap\Rat$.
    
    If $\phi$ is consistent with agent $a$'s beliefs and
    $\lnot\phi\land\psi$ is consistent with agent $a$'s knowledge,
    then agent $a$ believes $\phi\lor\psi$.
  \end{itemize}
\end{frame}

\begin{frame}
  \frametitle{Modal Language}
  \begin{eqnarray*}
    \phi & ::= & 
    \top \mid p \mid \neg\phi \mid \phi\land\phi \mid
    K\phi \mid B\phi
    \\
    &&
    \text{\footnotesize 
      $p\in\Prop$
    }
  \end{eqnarray*}
  \begin{itemize}
  \item<+-> Threshold $c\in(0,1)\cap\Rat$ omitted; will come
    from the semantics.

  \item<+-> Define the Segerberg formula
    $(\phi_i\mathbb{I}\psi_i)_{i=1}^m$ as before.

  \item<+-> Each $c\in(0,1)\cap\Rat$ maps modal formulas
    to probability formulas:
    \[
    \begin{array}{ccl@{\qquad}l}
      p^c & := & p \text{ for } p\in\Prop\cup\{\top\}
      \\
      (\neg \phi)^c & := & \neg \phi^c 
      \\
      (\phi \land \psi)^c & := & \phi^c \land \psi^c 
      \\
      (K\phi)^c & := & P(\phi^c) = 1
      & (= K\phi^c \text{ in probability lang.})
      \\
      (B\phi)^c & := & \textstyle P(\phi^c) > c
      & (= B^c\phi^c \text{ in probability lang.})
    \end{array}
    \]

  \item<+-> We define a neighborhood semantics; maps are
    truth-preserving.
  \end{itemize}
\end{frame}

\begin{frame}
  \frametitle{Our Epistemic Neighborhood Models}
  \vspace{-1em}
  \[
  \M = (W,R,V,N)
  \]
  \vspace{-1em}
  \begin{itemize}
  \item<+-> $(W,R,V)$ is a \underline{finite} $\mathsf{S5}$ Kripke
    model; notation $[w]$ as before.

  \item<+-> $N:W\to\wp(\wp(W))$ is a neighborhood function
    satisfying:
    \begin{itemize}
    \item<+-> $X\in N(w)$ implies $X\subseteq[w]$,

      Agent does not believe something known to be false;

    \item<+-> $\emptyset\notin N(w)$,

      Agent does not believe logical falsehoods;

    \item<+-> $[w]\in N(w)$,

      Agent believes logical truths;

    \item<+-> $v\in[w]$ implies $N(v)=N(w)$,

      Agent believes $X$ iff she knows she believes $X$; and

    \item<+-> $X\subseteq Y\subseteq[w]$ and $X\in N(w)$ together
      imply $Y\in N(w)$,

      Agent believes all logcal consequences of a given belief.
    \end{itemize}

  \item<+-> Semantics ($\modelsn$, $\semn{\cdot}$): standard Boolean
    semantics plus
      \begin{eqnarray*} 
        \M, w \modelsn K \phi  & \text{ iff } & 
        [w]\subseteq\semn{\phi}^\M
        \\
        \M, w \modelsn B \phi  & \text{ iff } &
        [w]\cap \semn{\phi}^\M \in N(w)
  \end{eqnarray*}
  \end{itemize}
\end{frame}

\begin{frame}
  \frametitle{Our Epistemic Neighborhood Models}
  \framesubtitle{Additional Properties on $N$ for ``Mid-Threshold''
    Models} \vspace{-1em}
  \[
  \M = (W,R,V,N)
  \]
  \vspace{-1em}
  \begin{itemize}
  \item<+-> $X\in N(w)$ implies $[w]-X\notin N(w)$,

    Agent does not believe $X$ and $[w]-X$ (i.e., belief consistency);

  \item<+-> $[w]-X\notin N(w)$ and
    $X\subsetneq Y\subseteq[w]$ together imply
    $Y\in N(w)$,

    Agent believes strict implications of a non-belief's negation; and

  \item<+-> if $(X_i\mathbb{I}Y_i)_{i=1}^m$, $X_1\in N(w)$, and
    $[w]-X_2,\dots,[w]-X_m\notin N(w)$, then
    $Y_j\in N(w)$ for some $j$,

    a Lenzen-like property for neighborhoods.

    \footnotesize (Note: $(X_i\mathbb{I}Y_i)_{i=1}^m$ is defined
    semantically.)
  \end{itemize}

  \bigskip
  \begin{uncoverenv}<+-> \small (Will see: Lenzen's logic for $c=\frac
    12$ is complete for mid-threshold models.)
  \end{uncoverenv}
\end{frame}

\begin{frame}
  \frametitle{Connecting Probability and Neighborhood Models}
  \uncover<+->{}Each $c\in(0,1)\cap\Rat$ induces a map
  \[
  \M=(W,R,V,P) 
  \qquad{\overset {c\;}\mapsto}\qquad 
  \M^c=(W,R,V,N^c)
  \]
  \vspace{-1em}
  given by setting
  \[
  N^c(w) := \{ X\subseteq[w] \mid P_w(X) > c \}\enspace.
  \]
  \begin{myboxy}<+->
    \textbf{Theorem.} We have:
    \begin{itemize}
    \item $N^c$ satisfies the epistemic neighborhood model properties.

    \item<+-> $N^{\frac 12}$ satisfies the additional
      ``mid-threshold'' properties.

    \item<+-> For each modal formula $\phi$:
      \begin{center}
        $\M^c,w\modelsn\phi \quad\text{iff}\quad
        \M,w\modelsp\phi^c\enspace.$
      \end{center}
      \footnotesize $\Rightarrow$ The qualitative modal language
      ``talks correctly'' about probability.

      $\Rightarrow$ Epistemic neighborhood models are connected with
      probability models.
    \end{itemize}
  \end{myboxy}
\end{frame}

\begin{frame}
  \frametitle{The Basic Qualitative Theory $\KB$}
  \vspace{-1em}\uncover<+->{}\begin{center}
    \small
    \textsc{Axiom Schemes}\\[.4em]
    \renewcommand{\arraystretch}{1.3}
    \begin{tabular}[t]{cl}
      (CL) &
      Schemes of Classical Propositional Logic
      \\
      (KS5) &
      $\mathsf{S5}$ axiom schemes for each $K$
      \\
      (KBC) &
      $K\phi\to B\phi$
      \\
      (BF) &
      $\lnot B\bot$
      \\
      (N) &
      $B\top$
      \\
      (Ap) &
      $B\phi\to KB\phi$
      \\
      (An) &
      $\lnot B\phi\to K\lnot B\phi$
      \\
      (KBM) &
      $K(\phi\to\psi)\to(B\phi\to B\psi)$
    \end{tabular}
    \renewcommand{\arraystretch}{1.0}
    \\[1em]
    \textsc{Rules}\vspace{-.5em}
    \[
    \begin{array}{c}
      \phi\to\psi \quad \phi
      \\\hline
      \psi
    \end{array}\;\text{\footnotesize(MP)}
    \qquad
    \begin{array}{c}
      \phi
      \\\hline
      K\phi
    \end{array}\;\text{\footnotesize(MN)}
    \]
  \end{center}

  \begin{myboxy}<+->
    \textbf{Theorem.} $\KB$ is sound and complete for
    epist.~neighborhood models.
  \end{myboxy}

  \uncover<+->{Proof is tricky (extends Segerberg's ``logical
    finiteness'' notion).}
\end{frame}

\begin{frame}
  \frametitle{$\KB$ Probability Incompleteness} \uncover<+->{}There
  exists a modal formula $\phi$ for which $\modelsp\phi^c$ and
  $\KB\nvdash\phi$.

  \medskip
  \begin{uncoverenv}<+->
    \footnotesize
    Sketch of argument (adapted from Walley and Fine, 1979):
    \begin{itemize}
    \item Propositional letters are $a,b,c,\dots,g$.

    \item $X := \{efg, abg, adf, bde, ace, cdg, bcf\}$ with, e.g.,
      $efg=\{e,f,g\}$.

    \item $Y := \{fga, bca, beg, cef, bdf, dea, cdg\}$ with
      similar abbreviations.
      
    \item $\M:=(W,R,V,N)$ defied by: $W$ is letter set, $R$ is total,
      $V$ satisfies $p$ at $p$, and $N(w)$ is the superset closure of
      $X$. This is an epist.~neighborhood model.

    \item $\sigma$ is the modal formula describing $(\M,a)$:
      informally (easily formalized),
      \begin{center}
        $\sigma \quad:=\quad a\bar{b}\cdots\bar{g} \land KW
        \land (\bigwedge_{Z\in N(a)}BZ) \land
        (\bigwedge_{Y\in\wp(W)-N(a)}\lnot BY)$\enspace.
      \end{center}

    \item $\M,w\modelsn\sigma$, so $\not\modelsn\lnot\sigma$ and
      hence $\KB\nvdash\lnot\sigma$ by our completness result.

    \item Argue that $\modelsp\lnot\sigma^c$ by \emph{reductio}:
      assume $\N,w\modelsp\sigma^c$.
      \begin{itemize}
        \footnotesize
      \item $\N,w\modelsp P(W)=1 \land (\bigwedge_{Z\in
          N(a)}P(Z)>c)\land (\bigwedge_{Y\in\wp(W)-N(a)}P(Z)\leq c)$.

      \item Each letter $p$ occurs in exactly three $X$-sets:
        \begin{center}
          $7c<\sum_{x\in X} P_w(x) = \sum_{p\in W}3\cdot
          P\left(p\land\bigwedge_{q\in W-\{p\}}\lnot
            q\right)$\enspace.
        \end{center}

      \item Each letter $p$ occurs in exactly three $Y$-sets, none of
        which is in $N(a)$:
        \begin{center}
          $7c\geq\sum_{y\in Y} P_w(y) = \sum_{p\in
            W}3\cdot P\left(p\land\bigwedge_{q\in W-\{p\}}\lnot
            q\right)$\enspace.
        \end{center}
      \end{itemize}
    \end{itemize}
  \end{uncoverenv}
\end{frame}

\begin{frame}
  \frametitle{$\KB^{0.5}$: (our name for) Lenzen's Theory for $c=\frac
    12$} \vspace{-1em}\uncover<+->{}
  \begin{center}
    \small
    \renewcommand{\arraystretch}{1.3}
    \begin{tabular}[t]{cl}
      (KB) & Schemes and rules of $\KB$
      \\
      (D) &
      $B\phi\to \check B\phi$
      \\
      (SC) &
      $\check B\phi \land 
      \check K(\lnot\phi\land\psi) \to 
      B(\phi\lor\psi)$
      \\
      (L) &
      $\textstyle [(\phi_i\mathbb{I}\psi_i)_{i=1}^m
      \land B\phi_1 \land \bigwedge_{i=2}^m \check B\phi_i] \to
      \bigvee_{i=1}^m B\psi_i$
    \end{tabular}
    \renewcommand{\arraystretch}{1.0}
  \end{center}

  \begin{myboxp}<+->
    In minimal modal logic terms: $\KB^{0.5}$ is
    \begin{itemize}
    \item $\mathsf{EMND45}+\lnot B\bot+\text{(L)}$ for belief

    \item plus $\mathsf{S5}$ knowledge and knowledge-belief
      connection principles.
    \end{itemize}
  \end{myboxp}

  \begin{myboxy}<+->
    \textbf{Theorem.} $\KB^{0.5}$ is sound and complete for
    mid-threshold models.
  \end{myboxy}

  \begin{myboxb}<+-> \textbf{Theorem (Lenzen).} $\KB^{0.5}$ is
    probabilistically complete for $c=\frac 12$:
    \begin{center}
      $\KB^{0.5}\vdash\phi$
      \qquad{}iff\qquad
      $\modelsp\phi^{\frac 12}$\enspace.
    \end{center}
  \end{myboxb}
\end{frame}

\begin{frame}
  \frametitle{Conclusions}
  \begin{itemize}
  \item<+-> Contributions:
    \begin{itemize}
    \item Formulate Lenzen's result in modern modal language.

    \item Modal completeness for epistemic neighborhood semantics.

    \item Truth-preserving connection: neighborhoods and
      probabilities.
    \end{itemize}

  \item<+-> Open questions:
    \begin{itemize}
    \item Probabilistic comleteness for high-thresholds
      $c\in\left(\frac 12,1\right)\cap\Rat$:
      \begin{center}
        $\KB^c\vdash\phi$ iff $\modelsp\phi^c$
        for which $\KB^c$?
      \end{center}
      Principles: for $s':=c/(1-c)$ and $s:=\textsf{ceiling}(s')$,
      \begin{center}
        \footnotesize
        \renewcommand{\arraystretch}{1.3}
        \begin{tabular}[t]{cl}
          (SC$_0^s$) &
          $\textstyle(\check K\phi_0\land
          \bigwedge_{i=1}^s\check B\phi_i\land
          \bigwedge_{i\neq j=0}^s K(\phi_i\to\lnot \phi_j))\to 
          B(\bigvee_{i=0}^s \phi_i)$
          \\
          (SC$_1^s$) &
          $\textstyle(\bigwedge_{i=1}^s\check B\phi_i\land
          \bigwedge_{i\neq j=1}^s K(\phi_i\to\lnot \phi_j))
          \to B(\bigvee_{i=1}^s \phi_i)$
          \\
          (WL) &
          $\textstyle [(\phi_i\mathbb{I}\psi_i)_{i=1}^m
          \land \bigwedge_{i=1}^m B\phi_i] \to
          \bigvee_{i=1}^m B\psi_i$
        \end{tabular}
      \end{center}
      (WL) sound.  SC$_1^s$ sound if $s\neq s'$, SC$_0^s$ sound if
      $s=s'$.

    \item Segerberg's/Gardenfors' probability comparison
      $\phi\geq\psi$ connection?

    \item De Jongh--Ghosh belief strength comparison $\phi\geq\psi$
      connection?

    \item Neighborhoods for public announcements (i.e., Bayesian
      updates)?
    \end{itemize}
  \end{itemize}
\end{frame}

%%
%% at end
%%
\AtBeginSubsection[]{}

\begin{frame}[c]
  \begin{center}
    {\Large \textcolor{blue!70!black}{The End}}\\[3em]
    {\large Bryan Renne}\\
    {\scriptsize\href{http://bryan.renne.org}{\texttt{http://bryan.renne.org/}}}
  \end{center}
\end{frame}

%%% Local Variables: 
%%% mode: latex
%%% TeX-master: "renne-2014-09-LiRA"
%%% End: 
