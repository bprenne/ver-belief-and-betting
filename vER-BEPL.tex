\documentclass[12pt]{article}

%%
%% page setup and comments:
%% - comment out the \def\comments{1} if you do not want comments
%% - leave the \def\comments{1} uncommented if you do want comments
%%

%\def\comments{1}

%% leave the next few lines below alone

\ifx\comments\undefined
  % then do normal margins and ignore comments
  \usepackage[letterpaper,margin=1in,includefoot]{geometry}
  \newcommand{\XXXcomment}[1]{}
\else
  % then do big marginpar margins and insert comments
  \usepackage[a4paper,margin=.5in,right=2.5in,marginpar=2in,includefoot]{geometry}
  \newcommand{\XXXcomment}[1]{\marginpar{\color{blue}{\footnotesize #1}}}
\fi

%%
%% packages and package-specific settings
%%

\usepackage{latexsym,amssymb,amsmath,amscd,amsthm,alltt,stmaryrd,graphicx,mathrsfs}

\usepackage{times}

\usepackage{float}
\floatstyle{boxed}
\restylefloat{figure}
\restylefloat{table}

%%
%% tikz stuff, including custom environment
%%

\usepackage{tikz}
\usetikzlibrary{trees,arrows,positioning,patterns,automata,shapes,decorations,decorations.pathmorphing}

\newcommand{\mystretch}{\renewcommand{\arraystretch}{1.5}}
\newcommand{\normalstretch}{\renewcommand{\arraystretch}{1}}

\newlength{\mysep}
\setlength{\mysep}{2.7cm}
\newlength{\mysepCloser}
\setlength{\mysepCloser}{2cm}
\newlength{\mysepFarther}
\setlength{\mysepFarther}{3cm}
\newlength{\mysepFarthest}
\setlength{\mysepFarthest}{4cm}

\newlength{\myMinsize}
\setlength{\myMinsize}{3em}
\newlength{\myBigger}
\setlength{\myBigger}{4em}
\newlength{\myNodeDistance}
\setlength{\myNodeDistance}{3cm}

\newenvironment{mytikz}[1][0em]{
\begin{tikzpicture}[>=latex,auto,node distance=\mysep,
  baseline={([yshift=#1]current bounding box.east)}]

  \normalstretch{}

  \tikzstyle{w}=[draw,circle,thick,minimum size=\myMinsize]

  \tikzstyle{e}=[draw,minimum size=\myMinsize,node distance=\myNodeDistance]
  
  \tikzstyle{every edge}=[draw,thick,font=\footnotesize]
  
  \tikzstyle{every label}=[font=\footnotesize]
  
  \tikzstyle{ev}=[anchor=west,node distance=\myNodeDistance]

  \tikzstyle{bigger}=[minimum size=\myBigger]

  \tikzstyle{closer}=[node distance=\mysepCloser]  

  \tikzstyle{farther}=[node distance=\mysepFarther]

  \tikzstyle{farthest}=[node distance=\mysepFarthest]

  \tikzstyle{l}=[node distance=\myNodeDistance]
}{\mystretch{}\end{tikzpicture}}

%%
%% theorem environments
%%

\theoremstyle{definition}
\newtheorem{theorem}{Theorem}[section]
\newtheorem{corollary}[theorem]{Corollary}
\newtheorem{proposition}[theorem]{Proposition}
\newtheorem{definition}[theorem]{Definition}
\newtheorem{lemma}[theorem]{Lemma}
\newtheorem{conjecture}[theorem]{Conjecture}
\newtheorem{example}[theorem]{Example}
\newtheorem{question}[theorem]{Question}
\newtheorem{exercise}[theorem]{Exercise}
\newtheorem{remark}[theorem]{Remark}

%%
%% custom commands
%%

\newcommand{\Nat}{\mathbb{N}}  % natural numbers
\newcommand{\Rat}{\mathbb{Q}}  % rational numbers
\newcommand{\Ree}{\mathbb{R}}  % real numbers
\newcommand{\Int}{\mathbb{Z}}  % integers

\newcommand{\M}{{\cal M}}      % caligraphic model M
\newcommand{\N}{{\cal N}}      % caligraphic model N

\newcommand{\Prop}{{\bf P}}    % propositional letters

\newcommand{\Lang}{{\cal L}}   % language

\newcommand{\conv}{\check{\ }} % dual modality

\newcommand{\KB}{{\mathsf{KB}}}                        % base theory
\newcommand{\KBlt}{{\mathsf{KB}^{\mathsf{<.5}}}}       % theory for c<.5
\newcommand{\KBeq}{{\mathsf{KB}^{\mathsf{0.5}}}}       % theory for c=.5
\newcommand{\KBeqm}{{\mathsf{KB}^{\mathsf{0.5}}_{-}}}  % theory for c=.5 without (kbm)
\newcommand{\KBgt}{{\mathsf{KB}^{\mathsf{>.5}}}}       % theory for c>.5
\newcommand{\KBgeq}{{\mathsf{KB}^{\mathsf{\geq.5}}}}   % theory for c>=.5
\newcommand{\KBc}{{\mathsf{KB}^{\mathsf{>}c}}}         % theory for c

\newcommand{\KBmyc}[1]{{\mathsf{KB}^{\mathsf{>{#1}}}}} % theory for specific c = #1


\newcommand{\Ceq}{{\mathcal{C}^{\mathsf{0.5}}}}    % for c=0.5

\newcommand{\Rel}[1]{\stackrel{#1}{\Longrightarrow}}         % #1 over \Longrightarrow

\newcommand{\sem}[1]{\llbracket{#1}\rrbracket}               % [[ #1 ]]

\newcommand{\modelsn}{\models_{\mathsf{n}}}                  % \models_n
\newcommand{\semn}[1]{\llbracket{#1}\rrbracket_{\mathsf{n}}} % [[ #1 ]]_n

\newcommand{\modelsp}{\models_{\mathsf{p}}}                  % \models_p
\newcommand{\semp}[1]{\llbracket{#1}\rrbracket_{\mathsf{p}}} % [[ #1 ]]_p

%%
%% author, title, &c
%%

\newcommand{\ourtitle}{Belief as Willingness to Bet}

\newcommand{\jan}{Jan van Eijck}

\newcommand{\janAffiliation}{CWI \&\ ILLC, Amsterdam}

\newcommand{\bryan}{Bryan Renne}

\newcommand{\bryanAffiliation}{ILLC, University of Amsterdam}

\newcommand{\bryanFunding}{Funded by an Innovational Research
  Incentives Scheme Veni grant from the Netherlands Organisation for
  Scientific Research (NWO).}

\title{\ourtitle{}} 

\author{\jan{}\\{}{\small\janAffiliation{}} \and
  \bryan{}\footnote{\bryanFunding{}}\\{}{\small\bryanAffiliation{}}}

\usepackage[pdftex,
            bookmarks=true,
            bookmarksnumbered=true,
            pdfborder={0 0 0},
            plainpages=false,
            pdfpagelabels]{hyperref}

\hypersetup{ 
  pdfauthor={\jan{}, \bryan{}}, 
  pdftitle={\ourtitle{}}
}

%%
%% beginning of paper
%%

\begin{document}

\maketitle

\begin{abstract}
  We investigate modal logics of knowledge and belief in which
  knowledge is probabilistic certainty and belief is probability
  exceeding a fixed rational threshold $c\geq\frac 12$.  Taking $c$ as
  an agent's betting threshold leads to the motto ``belief is
  willingness to bet.''  The logic $\KBeq$ for $c=\frac 12$ has
  $\mathsf{S5}$ knowledge modalities along with sub-normal belief
  modalities that extend the minimal modal logic
  $\mathsf{EMND45}+\lnot B_a\bot$ by way of certain schemes relating
  knowledge and belief. Lenzen was the first to show that a version of
  this logic is sound and complete for the probability interpretation.
  We reformulate his results and present them here in a modern and
  accessible form.  In addition, we introduce a new epistemic
  neighborhood semantics that will be more familiar to modern modal
  logicians.  Using a well-known result due to Scott, we provide the
  properties that must be imposed on finite epistemic neighborhood
  models so as to guarantee the existence of a probability measure
  respecting the neighborhood function in the appropriate way.  This
  yields a certain link between probabilistic and modal neighborhood
  semantics that may be of use in future cross-disciplinary work.
\end{abstract}

\section{Introduction}

A number of authors have studied connections between modal logic and
probability; see, e.g., \cite{Halpern2003:rau,Herzig2003:mpbaa} and
citations therein.  Our interest here is in identifying the modal
logic of probabilistic certainty and belief exceeding a fixed
$c\in[\frac 12,1)\cap\Rat$ and then relating this logic to a more
familiar modal semantics.

Lenzen \cite{Lenzen2003:kbasp,Lenzen1980:gwuw} is to our knowledge the
first to consider this study for $c=\frac 12$. Actually, his
perspective on the relation between knowledge and belief was slightly
different from ours.  He identified ``the agent is convinced of $A$''
with $P(A)=1$ and ``$B$ is believed'' by $P(B)>\frac 12$. Conviction
(German: \emph{\"{U}berzeuging}) does not imply truth.  In order to
guarantee completeness for the probability interpretation, Lenzen uses
an infinite modal scheme that expresses a key condition in a theorem
due to Scott \cite{Sco64:JMP}.  This theorem provides the necessary
and sufficient conditions a binary order $\preceq$ over a finite
Boolean algebra must satisfy so as to guarantee the existence of a
probability measure $P$ on the algebra satisfying $x\preceq y$ iff
$P(x)\leq P(y)$.  Scott's theorem was exploited by Segerberg
\cite{Segerberg1971:qpiams} and by G{\"a}rdenfors \cite{Gardenfors75}
to develop a modal logic of qualitative probabilistic comparisons
(with a binary modal operator for ``$A$ is at least as probable as
$B$'').  Lenzen's logic is different because the connection with
probability is different; however, in both cases, completeness for the
probabilistic interpretation boils down to Scott's theorem.

In more recent related work, Herzig \cite{Herzig2003:mpbaa} considers
a logic of belief and action in which belief in $A$ is identified with
$P(A)>P(\lnot A)$. This is equivalent to Lenzen's notion.  Another
more recent work by Kyberg and Teng \cite{KyburgTeng2012:tlorkr}
investigates a notion of ``acceptance'' in which $A$ is accepted
whenever the probability of $\lnot A$ is at most some small
$\epsilon$.  This gives rise to the minimal modal logic
$\mathsf{EMN}$, which is different than Lenzen's logic.

We herein consider belief \`{a} la Lenzen not only for the case
$c=\frac 12$ but also for the case $c>\frac 12$.  As it turns out, the
logics for these cases are different, though our focus will be on the
logic for $c=\frac 12$.  Probability completeness for $c>\frac 12$ is
still open.  Thresholds $c<\frac 12$ permit simultaneous belief of $A$
and $\lnot A$ along with other unusual properties; these
``low-threshold'' beliefs are also left for future work (though we say
a bit more on this later).

In Section~\ref{Section:EPL} we identify a Kripke-style semantics for
probability logic similar to
\cite{EijckSchwarzentruber2014:epls,Halpern2003:rau} (and no doubt to
many others).  We require that all worlds are probabilistically
possible but not necessarily epistemically so, and we provide some
examples of how this semantics works.  In particular, we demonstrate
that our requirement is not problematic.

In Section~\ref{Section:CB} we define our modal notions of certain
knowledge and belief exceeding threshold $c$, explain the motto
``belief is willingness to bet,'' and prove a number of properties of
certain knowledge and this ``betting'' belief.  For instance, we show
that knowledge is $\mathsf{S5}$ and belief is not normal.  We show a
number of other threshold-specific properties of betting belief as
well.  In particular, we see that the belief modality extends the
minimal modal logic $\mathsf{EMND45}+\lnot B_a\bot$ by way of certain
schemes relating knowledge and belief.\footnote{The definition of this
  logic is provided later. See \cite[Ch.~8]{Chellas:ml} for further
  details on naming of some minimal modal logics.}

We then introduce a formal modal language in Section~\ref{Section:ENM}
and relate this language to the probabilistic notions of belief and
knowledge.  We also introduce a neighborhood semantics for this
language with a new epistemic twist.  The relationship between the
neighborhood and probabilistic semantics is provided in
Section~\ref{Section:BeliefBet}.  There we prove that for each
threshold $c$, there is a translation from epistemic probability
models to epistemic neighborhood models such that certain knowledge,
betting belief, and Boolean truth are all preserved.  We also show
that for $c=\frac 12$, certain desirable properties are preserved.

In Section~\ref{Section:Calculi}, we introduce the modal theory
$\KBeq$, which is our name for our modern reformulation of Lenzen's
modal theory of knowledge and belief.  We prove this theory is sound
and complete with respect to a certain class of our epistemic
neighborhood models.  Using our notation and concepts, we also present
Lentzen's proof \cite{Lenzen1980:gwuw} that $\KBeq$ determines belief
with threshold $c=\frac 12$.  Given the link between our new epistemic
neighborhood semantics and probability, our results may be viewed as a
contribution to the study connecting two schools of rational decision
making: the probabilist (e.g., \cite{koerner2008naive}) and the
AI-based (e.g., \cite{KyburgTeng2012:tlorkr}).


\section{Epistemic Probability Models}
\label{Section:EPL} 

\begin{definition}
  \label{definition:epistemic-probability-model}
  We fix a finite nonempty set $A$ of ``agents'' and a set $\Prop$ of
  propositional letters.  An \emph{epistemic probability model} is a
  structure $\M=(W,R,V,P)$ satisfying the following.
  \begin{itemize} 
  \item $(W,R,V)$ is a finite multi-agent $\mathsf{S5}$ Kripke model:
    \begin{itemize}
    \item $W$ is a finite nonempty set of ``worlds.''  An \emph{event}
      is a set $X\subseteq W$ of worlds.  When convenient, we identify a world
      $w$ with the singleton event $\{w\}$.
      
    \item $R:A\to\wp(W\times W)$ assigns an equivalence relation $R_a$
      on $W$ to each agent $a\in A$.  We let
      \[
      [w]_a:=\{v\in W\mid wR_av\}
      \]
      denote the $a$-equivalence class of world $w$.  This is the
      set of worlds that agent $a$ cannot distinguish from $w$.

    \item $V:W\to\wp(\Prop)$ assigns a set $V(w)$ of propositional
      letters to each world $w\in W$.
    \end{itemize}

  \item $P:A \to\wp(W)\to[0,1]$ assigns to each agent $a\in A$ a
    probability measure $P_a:\wp(W)\to[0,1]$ over the finite algebra
    $\wp(W)$ satisfying the property of \emph{full support\/}:
    $P_a(w)\neq0$ for each $w\in W$.
  \end{itemize}
  A \emph{pointed epistemic probability model} is a pair $(\M,w)$
  consisting of an epistemic probability model $\M=(W,R,V,P)$ and
  world $w\in W$ called the \emph{point}.
\end{definition}

Agent $a$'s uncertainty as to which world is the actual world
is given by the equivalence relation $R_a$. 
If $w$ is the actual world, then the probability agent
$a$ assigns to an event $X$ at $w$ is given by
\begin{equation}
  P_{a,w}(X) := \frac{P_a(X\cap[w]_a)}{P_a([w]_a)}\enspace.
  \label{eq:probability}
\end{equation}
In words: the probability agent $a$ assigns to event $X$ at world $w$
is the probability she assigns to $X$ conditional on her knowledge at
$w$. Slogan: subjective probability is always conditioned, and the
most general condition is the given by the knowledge of the
agent. This makes sense because the right side of
\eqref{eq:probability} is just $P_a(X|[w]_a)$, the probability of $X$
conditional on $[w]_a$.  Note that $P_{a,w}(X)$ is always
well-defined: we have $w\in[w]_a$ by the reflexivity of $R_a$ and
hence $0<P_a(w)\leq P_a([w]_a)$ by full support, so the denominator on
the right side of \eqref{eq:probability} is nonzero.

\begin{example}[Horse Racing] 
  \label{ExampleHorseRacing}
  Three horses compete in a race.  For each $i\in\{1,2,3\}$, horse
  $h_i$ wins the race in world $w_i$.  Neither agent can distinguish
  between these three possibilities. Agent $a$ assigns the horses
  winning chances of $3{:}2{:}1$, while $b$ assigns chances
  $1{:}2{:}1$.  We represent this situation in the form of an
  epistemic probability model $\M_{\ref{ExampleHorseRacing}}$
  pictured as follows:
  \begin{center}
    \begin{mytikz}
      %%
      %% nodes
      %%
      \node[w,label={below:$w_1$}] (w1) {$h_1$};

      \node[w,right of=w1,label={below:$w_2$}] (w2) {$h_2$};

      \node[w,right of=w2,label={below:$w_3$}] (w3) {$h_3$};

      %%
      %% edges
      %%
      \path (w1) edge[<->] node{$a,b$} (w2);

      \path (w2) edge[<->] node{$a,b$} (w3);
    \end{mytikz}
    \begin{eqnarray*}
      P_a &=& \textstyle\{w_1:\frac 36, w_2:\frac 26, w_3:\frac 16\} \\
      P_b &=& \textstyle\{w_1:\frac 14, w_2:\frac 24, w_3:\frac 14\}
    \end{eqnarray*}

    $\M_{\ref{ExampleHorseRacing}}$
  \end{center}
  When we picture epistemic probability models, the arrows of each
  individual agent are to be closed under reflexivity and
  transitivity.  With this convention in place, it is not difficult to
  verify the following.
  \begin{enumerate}
  \item $P_{a,w_1} (\{w_1,w_3\}) = \frac 23$.

    At $w_1$, agent $a$ assigns probability $\frac 23$ to the event
    that the winner is horse $1$ or horse $3$.

  \item $P_{b,w_1} (\{w_1,w_3\}) = \frac 12$.

    At $w_1$, agent $b$ assigns probability $\frac 12$ to the event
    that the winner is horse $1$ or horse $3$.
  \end{enumerate}
\end{example} 

The property of full support says
 that each world is probabilistially possible.
Therefore, in order to represent a situation in which
agent $a$ is certain that horse $3$ can never win, we simply make the
$h_3$-worlds inaccessible via $R_a$.

\begin{example}[Certainty of impossibility]
  \label{ExampleHorseRacing2}
  We modify Example~\ref{ExampleHorseRacing} by eliminating the
  $a$-arrow between worlds $w_2$ and $w_3$.
  \begin{center}
    \begin{mytikz}
      %%
      %% nodes
      %%
      \node[w,label={below:$w_1$}] (w1) {$h_1$};

      \node[w,right of=w1,label={below:$w_2$}] (w2) {$h_2$};

      \node[w,right of=w2,label={below:$w_3$}] (w3) {$h_3$};

      %%
      %% edges
      %%
      \path (w1) edge[<->] node{$a,b$} (w2);

      \path (w2) edge[<->] node{$b$} (w3);
    \end{mytikz}
    \begin{eqnarray*}
      P_a &=& \textstyle\{w_1:\frac 36, w_2:\frac 26, w_3:\frac 16\} \\
      P_b &=& \textstyle\{w_1:\frac 14, w_2:\frac 24, w_3:\frac 14\}
    \end{eqnarray*}

    $\M_{\ref{ExampleHorseRacing2}}$
  \end{center}
  At world $w_1$ in this picture, there is no $a$-accessible world at
  which horse $3$ wins.  Therefore, at world $w_1$, agent $a$ assigns
  probability $0$ to the event that horse $3$ wins:
  $P_{a,w_1}(w_3)=0$.
\end{example}

We define a language $\Lang$ for reasoning about epistemic probability
models.

\begin{definition}
  The language $\Lang$ of \emph{multi-agent probability logic} is
  defined by the following grammar.
  \begin{eqnarray*}
    \phi & ::= & 
    \top \mid p \mid \neg\phi \mid \phi\land\phi \mid
    t_a\geq 0
    \\
    t_a   & ::= & q \mid q \cdot P_a(\phi) \mid t_a+t_a
    \\
    &&
    \text{\footnotesize 
      $p\in\Prop$,
      $q\in\Rat$,
      $a\in A$
    }
  \end{eqnarray*}
  We adopt the usual abbreviations for Boolean connectives.  We define
  the relational symbols $\leq$, $>$, $<$, and $=$ in terms of $\geq$
  as usual.  For example, $t=s$ abbreviates $(t\geq s)\land(s\geq t)$.
  We also use the obvious abbreviations for writing general linear
  inequalities.  For example, $P_a(p)\leq 1-q$ abbreviates
  $1+(-q)+(-1)\cdot P_a(p)\geq 0$.
\end{definition}

\begin{definition} 
  Let $\M=(W,R,V,P)$ be an epistemic probability model.  We define
  a binary truth relation $\modelsp$ between a pointed epistemic
  probability model $(\M,w)$ and $\Lang$-formulas as follows.
  \[
  \renewcommand{\arraystretch}{1.3}
  \begin{array}{lcl}
    \M,w\modelsp\top 
    \\
    \M,w\modelsp p & \text{iff} & 
    p \in V(w) 
    \\
    \M,w\modelsp\neg\phi & \text{iff} &
    \M,w\not\modelsp\phi
    \\
    \M,w\modelsp\phi\land\psi & \text{iff} &
    \M,w\modelsp\phi \text{ and } \M,w\modelsp\psi
    \\
    \M,w\modelsp t_a\geq 0 & \text{iff} &
    \sem{t_a}_w\geq 0
  \end{array}
  \]
  \begin{eqnarray*}
    \semp{\phi} & := &
    \{ u \in W\mid \M,u\modelsp\phi \}
    \\
    P_{a,w}(X) & := & 
    \displaystyle\frac{P_a(X\cap[w]_a)}{P_a([w]_a)}
    \\
    \sem{q}_w & := & q
    \\
    \sem{q\cdot P_a(\phi)}_w & := & 
    q\cdot P_{a,w}(\semp{\phi}) 
    \\
    \sem{t_a+t_a'}_w & := &
    \sem{t_a}_w + \sem{t_a'}_w
  \end{eqnarray*}
  Validity of $\phi\in\Lang$ in epistemic probability model $\M$,
  written $\M\modelsp\phi$, means that $\M,w\modelsp\phi$ for each
  world $w\in W$.  Validity of $\phi\in\Lang$, written $\modelsp\phi$,
  means that $\M\modelsp\phi$ for each epistemic probability model
  $\M$.
\end{definition} 

\section{Certainty and Belief} 
\label{Section:CB} 

\cite{Eijck2013:lap} formulates and proves a ``certainty theorem''
relating certainty in epistemic probability models to knowledge in a
version of these models in which the probabilistic information is
removed.  This motivates the following definition.

\begin{definition}[Knowledge as Certainty]
  We adopt the following abbreviations.
  \begin{itemize}
  \item $K_a\phi$ abbreviates $P_a(\phi)=1$. 

    We read $K_a\phi$ as ``agent $a$ knows $\phi$.''

  \item $\check K_a\phi$ abbreviates $\lnot K_a\lnot\phi$.

    We read $\check K_a\phi$ as ``$\phi$ is consistent with agent
    $a$'s knowledge.''
  \end{itemize}
\end{definition}

\begin{theorem}[\cite{Eijck2013:lap}]
  \label{theorem:knowledge}
  $K_a$ is an $\mathsf{S5}$ modal operator:
  \begin{enumerate}
  \item $\modelsp \phi$ for each $\Lang$-instance $\phi$ of a scheme
    of classical propositional logic.

    Axioms of classical propositional logic are valid.

  \item $\modelsp K_a(\phi\to\psi)\to(K_a\phi\to K_a\psi)$
    
    Knowledge is closed under logical consequence.

  \item $\modelsp K_a\phi\to \phi$

    Knowledge is veridical.
    
  \item $\modelsp K_a\phi\to K_aK_a\phi$

    Knowledge is positive introspective:  it is known what is known.
    
  \item $\modelsp \lnot K_a\phi\to K_a\lnot K_a\phi$

    Knowledge is negative introspective: it is known what is not
    known.
    
  \item $\modelsp\phi$ implies $\modelsp K_a\phi$

    All validities are known.

  \item $\modelsp\phi\to\psi$ and $\modelsp\phi$ together imply
    $\modelsp\psi$.

    Validities are closed under the rule of Modus Ponens.
  \end{enumerate}
\end{theorem}

We define belief in a proposition $\phi$ as willingness to take bets
on $\phi$ with the odds being better than some rational number
$c\in(0,1)\cap\Rat$.  This leads to a number of degrees of belief, one
for each threshold $c$.

\begin{definition}[Belief as Willingness to Bet]
  \label{definition:belief}
  Fix a threshold $c\in(0,1)\cap\Rat$.
  \begin{itemize}
  \item $B_a^c\phi$ abbreviates $P_a(\phi)>c$.

    We read $B_a^c\phi$ as ``agent $a$ believes $\phi$ with threshold
    $c$.''

  \item $\check B_a^c\phi$ abbreviates $\lnot B_a^c\lnot\phi$.

    We read $\check B_a\phi$ as ``$\phi$ is consistent with agent
    $a$'s threshold-$c$ beliefs.''
  \end{itemize}
  If the threshold $c$ is omitted (either in the notations $B_a^c\phi$
  and $\check B_a^c\phi$ or in the informal readings of these
  notations), it is assumed that $c=\frac 12$.
\end{definition}

This notion of belief comes from subjective probability
\cite{Jeffrey2004:sptrt}.  In particular, fix a threshold
$c=p/q\in(0,1)\cap\Rat$.  Suppose that agent $a$ believes $\phi$ with
threshold $c=p/q$; that is, $P_a(\phi)>p/q$.  If the agent wagers $p$
dollars for a chance to win $q-p$ dollars on a bet that $\phi$ is
true, then she expects to win
\[
(q-p)\cdot P_a(\phi) - p\cdot(1-P_a(\phi)) = q\cdot P_a(\phi) - p
\]
dollars on this bet.  This is a positive number of dollars if and only
if $q\cdot P_a(\phi)>p$.  But notice that the latter is guaranteed by
the assumption $P_a(\phi)>p/q$.  Therefore, it is rational for agent
$a$ to take this bet.  Said in the parlance of the subjective
probability literature, ``If agent $a$ stakes $p$ to win $q-p$ in a
bet on $\phi$, then her winning expectation is positive in case she
believes $\phi$ with threshold $c= p/q$.''  Or in a short motto:
``Belief is willingness to bet.''

\begin{remark}
  Belief based on threshold $c=0$ or $c=1$ is trivial to express in
  terms of negation, $K_a$, and falsehood $\bot$.  So we do not
  consider these thresholds here.  Beliefs based on low-thresholds
  $c\in(0,\frac 12)\cap\Rat$ have unintuitive and unusual features.
  First, low-threshold beliefs unintuitively permit inconsistency of
  the kind that an agent can believe both $\phi$ and $\lnot\phi$ while
  avoiding inconsistency of the kind that the agent can believe a
  single contradictory formula such as $\bot$. (This suggests some
  connection with paraconsistent logic.)  Second, the dual of a
  low-threshold belief implies the belief at that threshold (i.e.,
  $\check B_a^c\phi\to B_a^c\phi$), which is unusual if we assign the
  usual ``consistency'' reading to dual operators (i.e., ``$\phi$ is
  consistent with the agent's beliefs implies $\phi$ is believed'' is
  unusual).  Since low-threshold $c\in(0,\frac12)\cap\Rat$ beliefs
  have these unintuitive and unusual features, we leave their study
  for future work, focusing instead on thresholds $c\in[\frac
  12,1)\cap\Rat$.
\end{remark}

The following lemma provides a useful characterization of the dual
$\check B^c_a\phi$.

\begin{lemma}
  \label{lemma:dual}
  Let $\M=(W,R,V,P)$ be an epistemic probability model.
  \begin{enumerate}
  \item \label{item:dual} $\M,w\modelsp\check B_a^c\phi$ iff
    $\M,w\modelsp P_a(\phi)\geq 1-c$.

  \item \label{item:dual-half} $\M,w\modelsp\check B_a^{\frac 12}\phi$ iff
    $\M,w\modelsp P_a(\phi)\geq \frac 12$.

 % \item \label{item:dual-low} For $c\in(0,\frac 12)\cap\Rat$:
 %   \[
 %   \M,w\modelsp\check B_a^c\phi \text{ implies } \M,w\modelsp
 %   P_a(\phi)>c\enspace\text{.}
 %   \]
  \end{enumerate}
\end{lemma}
\begin{proof}
  For Item~\ref{item:dual}, we have the following:
  \[
  \renewcommand{\arraystretch}{1.3}
  \begin{array}{lll}
    &
    \M,w\modelsp\check B^c_a\phi
    \\
    \text{iff} &
    \M,w\modelsp\lnot B^c_a\lnot\phi 
    & \text{by definition of $\check B^c_a\phi$}
    \\
    \text{iff} &
    P_{a,w}(\semp{\lnot\phi})\not>c
    & \text{by definition of $B^c_a\phi$ and $\modelsp$}
    \\
    \text{iff} &
    P_{a,w}(\semp{\lnot\phi})\leq c
    & \text{since $\Rat$ is totally ordered}
    \\
    \text{iff} &
    P_{a,w}(\semp{\phi})\geq 1-c
    & \text{since $\semp{\lnot\phi}=W-\semp{\phi}$}
  \end{array}
  \]
  For Item~\ref{item:dual-half}, apply Item~\ref{item:dual} 
  with $c =\frac 12$. 
  % For Item~\ref{item:dual-low}, we
  % observe that $c\in(0,\frac 12)$ implies $1-c>c$, so the result
  % follows by Item~\ref{item:dual}. 
\end{proof}

%Note that Item~\ref{item:dual-low} of Lemma~\ref{lemma:dual} implies that for
%low-threshold belief $c$, $\check B_a^c \phi$ implies $B_a^c \phi$.

We now consider a simple example.

\begin{example}[Non-normality]
  \label{example:non-normality}
  In this single-agent variation, all horses have equal chances of winning and
  agent $a$ knows this.
  \begin{center}
    \begin{mytikz}
      %%
      %% nodes
      %%
      \node[w,label={below:$w_1$}] (w1) {$h_1$};

      \node[w,right of=w1,label={below:$w_2$}] (w2) {$h_2$};

      \node[w,right of=w2,label={below:$w_3$}] (w3) {$h_3$};

      %%
      %% edges
      %%
      \path (w1) edge[<->] node{$a$} (w2);

      \path (w2) edge[<->] node{$a$} (w3);
    \end{mytikz}

    $P_a=\{w_1:\frac 13, w_2:\frac 13, w_3:\frac 13$\}

    \bigskip
    $\M_{\ref{example:non-normality}}$
  \end{center}
  Recalling that an omitted threshold $c$ is implicitly
  assumed to be $\frac 12$,
  the following are readily verified.
  \begin{enumerate}
  \item $\M_{\ref{example:non-normality}}\modelsp B_a(h_1\lor h_2\lor
    h_3)$.

    Agent $a$ believes the winning horse is among the three.

    (Agent $a$ is willing to bet that the winning horse is among the three.)

  \item $\M_{\ref{example:non-normality}}\modelsp B_a(h_1\lor h_2)\land
    B_a(h_1\lor h_3)\land B_a(h_2\lor h_3)$.

    Agent $a$ believes the winning horse is among any two.

    (Agent $a$ is willing to bet that the winning horse is among any two.)

  \item \label{item:conjuncts} $\M_{\ref{example:non-normality}}\modelsp
    B_a\lnot h_1\land B_a\lnot h_2\land B_a\lnot h_3$.

    Agent $a$ believes the winning horse is not any particular one.

    (Agent $a$ is willing to bet that the winning horse is not any
    particular one.)

  \item \label{item:conjunction}
    $\M_{\ref{example:non-normality}}\modelsp \lnot B_a(\lnot h_1\land \lnot
    h_2)$.

    Agent $a$ does not believe that both horses $1$ and $2$ do not
    win.

    (Agent $a$ is not willing to bet that both horses $1$ and $2$ do
    not win.)
  \end{enumerate}
\end{example} 

It follows from Items~\ref{item:conjuncts} and \ref{item:conjunction}
of Example~\ref{example:non-normality} that the present notion of
belief is not closed under conjunction.  This is discussed as part of
the literature on the ``Lottery Paradox''
\cite{Kyburg1961:patlorb}.\footnote{The usual formulation of the
  Lottery Paradox: it is paradoxical for an agent to believe that one
  of $n$ lottery tickets will be a winner (i.e., ``some ticket is a
  winner'') without believing of any particular ticket that it is the
  winner (i.e., ``for each $i\in\{1,\dots,n\}$, ticket $i$ is not a
  winner'').}  However, there is no reason in general that it is
paradoxical to assign a conjunction $\phi \land \psi$ a lower
probability than either of its conjunctions.  Indeed, if $\phi$ and
$\psi$ are independent, then the probability of their conjunction
equals the product of their probabilities, so unless one of $\phi$ or
$\psi$ is certain or impossible, the probability of $\phi \land \psi$
will be less than the probability of $\phi$ and less than the
probability of $\psi$.

We set aside philosophical arguments for or against closure of belief
under conjunction and instead turn our attention to the study of the
properties of the present notion of belief.  One of these is a
complicated but useful property due to Scott \cite{Sco64:JMP} that
makes use of notation due to Segerberg \cite{Segerberg1971:qpiams}.

\begin{definition}[Segerberg notation; \cite{Segerberg1971:qpiams}]
  \label{definition:segerberg-notation}
  Fix a positive integer $m\in\Int^+$ and formulas
  $\phi_1,\dots,\phi_m$ and $\psi_1,\dots,\psi_m$.  The expression
  \begin{equation}
    (\phi_1,\dots,\phi_m\mathbb{I}_a\psi_1,\dots,\psi_m)
    \label{eq:segerberg}
  \end{equation}
  abbreviates the formula
  \[
  K_a(C_0\lor C_1\lor C_2 \lor \cdots \lor C_m)\enspace,
  \]
  where $C_i$ is the disjunction of all conjunctions
  \[
  d_1\phi_1\land\cdots\land d_m\phi_m \land
  e_1\psi_1\land\cdots\land e_m\psi_m
  \]
  satisfying the property that \emph{exactly} $i$ of the $d_k$'s are the
  empty string, \emph{at least} $i$ of the $e_k$'s are the empty string, and
  the rest of the $d_k$'s and $e_k$'s are the negation sign $\lnot$.
  We may write $(\phi_i\mathbb{I}_a\psi_i)_{i=1}^m$ as an
  abbreviation for \eqref{eq:segerberg}.  Finally, let
  \[
  (\phi_i\mathbb{E}_a\psi_i)_{i=1}^m
  \quad\text{abbreviate}\quad
  (\phi_i\mathbb{I}_a\psi_i)_{i=1}^m\land(\psi_i\mathbb{I}_a\phi_i)_{i=1}^m
  \enspace.
  \]
  We also allow the use of $\mathbb{E}_a$ in a notation similar to 
  \eqref{eq:segerberg}.
\end{definition}

The formula $(\phi_i\mathbb{I}_a\psi_i)_{i=1}^m$ says that agent $a$
knows that the number of true $\phi_i$'s is less than or equal to the
number of true $\psi_i$'s.  Put another way,
$(\phi_i\mathbb{I}_a\psi_i)_{i=1}^m$ is true if and only if every one
of $a$'s epistemically accessible worlds satisfies at least as many
$\psi_i$'s as $\phi_i$'s.  The formula
$(\phi_i\mathbb{E}_a\psi_i)_{i=1}^m$ says that every one of $a$'s
epistemically accessible worlds satisfies exactly as many $\psi_i$'s
as $\phi_i$'s.

\begin{definition}[Scott scheme; \cite{Sco64:JMP}]
  \label{definition:scott-schemes}
  We define the following scheme:
  \[
  \begin{array}{cl}
    \textstyle [
    %(\phi_1,\dots,\phi_m\mathbb{I}_a\psi_1,\dots,\psi_m)
    (\phi_i\mathbb{I}_a\psi_i)_{i=1}^m
    \land B_a^{c} \phi_1 \land \bigwedge_{i=2}^m \check B_a^{c} \phi_i] \to
    \bigvee_{i=1}^m B_a^{c}\psi_i
    &
    \text{(Scott)}
  \end{array}
  \]
  If $m=1$, then $\bigwedge_{i=2}^m \check B_a^c\phi_i$ is $\top$.
  Note that (Scott)
  is meant to encompass the indicated scheme for each positive integer
  $m\in\Int^+$.
\end{definition}

(Scott) says that if agent $a$ knows the number of true $\phi_i$'s is
less than or equal to the number of true $\psi_i$'s, agent $a$ believes $\phi_1$
with threshold $c$, and the remaining $\phi_i$'s are each consistent
with agent $a$'s threshold-$c$ beliefs, then agent $a$ believes one of
the $\psi_i$'s with threshold $c$.  Adapting a proof of Segerberg
\cite{Segerberg1971:qpiams}, we show that belief with threshold
$c=\frac12$ satisfies (Scott).

We report this result along with a number of other properties
in the following proposition.

\begin{theorem}[Properties of Belief]
  \label{theorem:belief}
  For $c\in(0,1)\cap\Rat$, we have:
  \begin{enumerate}
  \item \label{item:B-not-normal} $\not\modelsp
    B_a^c(\phi\to\psi)\to(B_a^c\phi\to B_a^c\psi)$.

    Belief is not closed under logical consequence.

    (So $B_a^c$ is not a normal modal operator.)

  \item \label{item:B-not-T} $\not\modelsp B_a^c\phi\to\phi$.

    Belief is not veridical.

  \item \label{item:B-C} $\modelsp K_a\phi\to B_a^c\phi$.

    What is known is believed.

  \item \label{item:B-B} $\modelsp\lnot B^c_a\bot$.

    The propositional constant $\bot$ for falsehood is not believed.

  \item \label{item:B-N} $\modelsp B_a^c\top$.

    The propositional constant $\top$ for truth is believed.

  \item \label{item:B-Ap} $\modelsp B_a^c\phi\to K_aB_a^c\phi$.

    What is believed is known to be believed.

  \item \label{item:B-An} $\modelsp \lnot B_a^c\phi\to K_a\lnot
    B_a^c\phi$.

    What is not believed is known to be not believed.

  \item \label{item:B-M} $\modelsp K_a(\phi\to\psi)\to(B_a^c\phi\to B_a^c\psi)$.

    Belief is closed under known logical consequence.

  \item \label{item:B-D} $\modelsp B_a^c\phi\to \check B_a^c\phi$.

    Belief is consistent: belief in $\phi$ implies
    disbelief in $\lnot\phi$.

  \item \label{item:B-SC} $\modelsp
    \check{B}_a^{\frac 12} \phi \land \check{K}_a(\neg \phi \land \psi)
    \rightarrow B_a^{\frac 12} (\phi \lor \psi)$.
    
    For mid-threshold belief, if $\phi$ is consistent
    with agent $a$'s beliefs and $\lnot\phi\land\psi$ is
    consistent with agent $a$'s knowledge, then agent $a$ believes
    $\phi\lor\psi$.  

  \item \label{item:B-Len}
    $
    \textstyle \modelsp
    [(\phi_i\mathbb{I}_a\psi_i)_{i=1}^m \land
    B_a^{\frac 12}\phi_1 \land \bigwedge_{i=2}^m \check B_a^{\frac 12}\phi_i] \to
    \bigvee_{i=1}^m B_a^{\frac 12}\psi_i
    $.

    Mid-threshold belief satisfies (Scott).
  \end{enumerate}
\end{theorem}
\begin{proof}
  We consider each item in turn.
  \begin{enumerate}
  \item Given $c\in(0,1)\cap\Rat$ and integers $p$ and $q$ such that
    $p/q=c$, we define $\M$ as the modification of the model
    $\M_{\ref{example:non-normality}}$ of
    Example~\ref{example:non-normality} obtained by changing
    $P_a$ as follows:
    \[
    P_a := \left\{ w_1:\frac{q-p}{2q},\; w_2:\frac pq,\; w_3:\frac{q-p}{2q}
    \right\}\enspace.
    \]
    Since $0<p<q$, it follows that
    \begin{eqnarray*}
      P_{a,w_1}(\semp{\lnot h_1\to h_2}) &=&
      P_{a,w_1}(\{w_1,w_2\}) =
      \frac{q+p}{2q}>\frac {2p}{2q}=\frac pq,
      \\
      P_{a,w_1}(\semp{\lnot h_1}) &=&
      P_{a,w_1}(\{w_2,w_3\}) = \frac{q+p}{2q}>\frac {2p}{2q}=\frac pq,\text{ and}
      \\
      P_{a,w_1}(\semp{h_2})&=& 
      P_{a,w_1}(w_2) = \frac pq\enspace.
    \end{eqnarray*}
    Therefore, we have
    \[
    \M,w_1\modelsp B^c_a(\lnot h_1\to h_2)\land B^c_a\lnot h_1\land
    \lnot B^c_ah_2\enspace.
    \]
    
  \item For $\M$ defined in the proof of Item~\ref{item:B-not-normal},
    we have
    \[
    \M,w_1\modelsp h_1\land B_a^c\lnot h_1\enspace.
    \]
    
  \item $\M,w\modelsp K_a\phi$ implies $P_{a,w}(\semp{\phi})=1 > c$.
    Hence $\M,w\modelsp B_a^c\phi$.

  \item $P_{a,w}(\semp{\bot})=0<c$.  Hence
    $\M,w\models\lnot B^c_a\bot$.

  \item $P_{a,w}(\semp{\top})=1>c$. Hence
    $\M,w\modelsp B_a^c\top$.

  \item $\M,w\modelsp B_a^c\phi$ implies $P_{a,w}(\semp{\phi})>c$.  To show that
    $\M,w\modelsp K_aB_a^c\phi$, we must argue that
    \[
    P_{a,w}(\semp{B_a^c\phi})= 
    \frac{P_a(\semp{B_a^c\phi}\cap[w]_a)}{P_a([w]_a)}
    =1\enspace.
    \]
    To show this, we prove that
    $\semp{B_a^c\phi}\cap[w]_a=[w]_a$.
    So choose $u\in[w]_a$.  Since $R_a$ is an equivalence relation,
    we have
    \begin{equation*}
      P_{a,u}(\semp{\phi})=
      \frac{P_a(\semp{\phi}\cap[u]_a)}{P_a([u]_a)}=
      \frac{P_a(\semp{\phi}\cap[w]_a)}{P_a([w]_a)}=
      P_{a,w}(\semp{\phi})>c\enspace,
    \end{equation*}
    which implies $u\in\semp{B_a^c\phi}$.  The result follows.
    
  \item The argument is similar to that for Item~\ref{item:B-Ap},
    though we note that $\M,w\modelsp\lnot B_a^c\phi$ implies
    $P_{a,w}(\semp{\phi})\leq c$.
    
  \item We assume that $\M,w\modelsp K_a(\phi\to\psi)$ and
    $\M,w\modelsp B_a^c\phi$.  This means that
    $P_{a,w}(\semp{\phi\to\psi})=1$ and
    $P_{a,w}(\semp{\phi})>c$.  But then it follows that
    $P_{a,w}(\semp{\psi})>c$ as well, which is what it means to
    have $\M,w\modelsp B_a^c\psi$.

  % \item By Lemma~\ref{lemma:dual}.

  \item Assume $c\in[\frac 12,1)\cap\Rat$ and $\M,w\modelsp B_a^c\phi$.  Then $P_{a,w}(\semp{\phi})>c
    \geq 1 - c$. So $P_{a,w}(\semp{\phi})\geq 1-c$.  The result
    therefore follows by Lemma~\ref{lemma:dual}.

  \item We prove something more general. 
    Assume $c\in(0,\frac 12]\cap\Rat$ and $\M,w\modelsp\check B^c_a\phi$.  By
    Lemma~\ref{lemma:dual}, it follows that
    $P_{a,w}(\semp{\phi})\geq c$.  Let us assume further that
    $\M,w\modelsp\check K_a(\lnot\phi\land\psi)$.  This means
    \[
    1\neq P_{a,w}(\semp{\lnot(\lnot\phi\land\psi)})=
    \frac{P_a(\semp{\lnot(\lnot\phi\land\psi)}\cap[w]_a)}{P_a([w]_a)}
    \enspace,
    \]
    which implies there exists $v\in
    \semp{\lnot\phi\land\psi}\cap[w]_a$.  Since $P_a(v)>0$ by full support, it follows that
    \begin{eqnarray*}
    P_{a,w}(\semp{\phi\lor\psi}) &=&
    \frac{ P_a(\semp{\phi\lor\psi}\cap [w]_a) }{ P_a([w]_a) }
    \\
    &=&
    \frac{ P_a(\semp{\phi}\cap [w]_a) }{ P_a([w]_a) } +
    \frac{ P_a(\semp{\lnot\phi\land\psi}\cap [w]_a) }{ P_a([w]_a) }
    \\
    &\geq&
    \frac{ P_a(\semp{\phi}\cap[w]_a) }{P_a([w]_a)} + \frac{ P_a(v) }{P_a([w]_a)}
    \\
    &=&
    P_{a,w}(\semp{\phi})+\frac{ P_a(v) }{P_a([w]_a)}
    \\
    &\geq&
    c + \frac{ P_a(v) }{P_a([w]_a)}
    > c\enspace.
    \end{eqnarray*}
    That is,
    $\M,w\modelsp B_a^c(\phi\lor\psi)$.

  \item Again, we prove something more general. 
  We assume $c\in(0,\frac 12]\cap\Rat$ plus the following:
    \begin{eqnarray}
      &&
      \M,w\modelsp (\phi_i\mathbb{I}_a\psi_i)_{i=1}^m
      \label{eq:S}
      \\
      &&
      \M,w\modelsp B^c_a\phi_1
      \label{eq:Bphi1}
      \\
      &&
      \textstyle \M,w\modelsp \bigwedge_{i=2}^m\check B^c_a\phi_i
      \label{eq:checkB}
    \end{eqnarray}
    We recall the meaning of
    \eqref{eq:S}: for each $v\in[w]_a$, the number of
    $\phi_i$'s true at $v$ is less than or equal to the number of $\psi_k$'s true at $v$.
    It therefore follows from \eqref{eq:S} that
    \begin{equation}
      P_{a,w}(\semp{\phi_1})+\cdots+P_{a,w}(\semp{\phi_m})\leq
      P_{a,w}(\semp{\psi_1})+\cdots+P_{a,w}(\semp{\psi_m})\enspace.
      \label{eq:sum}
    \end{equation}
    Outlining an argument due to Segerberg 
    \cite[pp.~344--346]{Segerberg1971:qpiams},
    the reason for this is as follows: we think of each world $v\in[w]_a$ as being
    assigned a ``weight'' $P_{a,w}(v)$.  A member $P_{a,w}(\semp{\phi_i})$ of the
    sum on the left of \eqref{eq:sum} is just a total of the
    weight of every $v\in[w]_a$ that satisfies
    $\phi_i$; that is,
    \[
    P_{a,w}(\semp{\phi_i})=\sum\{P_{a,w}(v)\mid v\in\semp{\phi_i}\cap[w]_a\}\enspace.
    \]
    Assumption \eqref{eq:S} tells us that for each $v\in[w]_a$, the
    number of totals $P_{a,w}(\semp{\phi_i})$ on the left of
    \eqref{eq:sum} to which $v$ contributes its weight is less than or equal to the
    number of totals $P_{a,w}(\semp{\psi_k})$ on the right of
    \eqref{eq:sum} to which $v$ contributes its weight.  But then the
    sum of totals on the left must be less than or equal to the sum of totals on the
    right.  Hence \eqref{eq:sum} follows.

    Having established \eqref{eq:sum}, we now proceed further with the
    overall proof.  By \eqref{eq:Bphi1}, we have
    $P_{a,w}(\semp{\phi_1})>c$.  Applying \eqref{eq:checkB}
    and Lemma~\ref{lemma:dual}, we have $P_{a,w}(\phi_i)\geq
    c$ for each $i\in\{2,\dots,m\}$.  Hence
    \[
    P_{a,w}(\semp{\psi_1})+\cdots+P_{a,w}(\semp{\psi_m})\geq
    P_{a,w}(\semp{\phi_1})+\cdots+P_{a,w}(\semp{\phi_m})> mc\enspace.
    \]
    That is, the sum of the $P_{a,w}(\semp{\psi_k})$'s must exceed
    $mc$.  Since each member of this $m$-member sum is
    non-negative, it follows that at least one member must exceed
    $c$.  That is, there exists $j\in\{0,\dots,m\}$ such
    that $P_{a,w}(\semp{\psi_j})>c$.  Hence
    $\M,w\modelsp\bigvee_{j=1}^mB_a^c\psi_j$. \qedhere
  \end{enumerate}
\end{proof}

\section{Epistemic Neighborhood Models}
\label{Section:ENM}

The modal formulas $K_a\phi$ and $B_a^c\phi$ were taken as
abbreviations in the language $\Lang$ of multi-agent probability
logic.  We wish to consider a propositional modal language that has
knowledge and belief operators as primitives.

\begin{definition}
  The language $\Lang_\KB$ of \emph{multi-agent knowledge and belief}
  is defined by the following grammar.
  \begin{eqnarray*}
    \phi & ::= & 
    \top \mid p \mid \neg\phi \mid \phi\land\phi \mid
    K_a\phi \mid B_a\phi
    \\
    &&
    \text{\footnotesize 
      $p\in\Prop$,
      $a\in A$
    }
  \end{eqnarray*}
  We adopt the usual abbreviations for other Boolean connectives and
  define the dual operators $\check K_a:=\lnot K_a\lnot$ and $\check
  B_a:=\lnot B_a\lnot$.  Finally,
  the $\Lang_\KB$-formula
  \[
  (\phi_1,\dots,\phi_m\mathbb{I}_a\psi_1,\dots,\psi_m)
  \]
  and its abbreviation
  $(\phi_i\mathbb{I}_a\psi_i)_{i=1}^m$
  are given
  as in Definition~\ref{definition:segerberg-notation} except that
  all formulas are taken from the language $\Lang_\KB$.
\end{definition}

Our goal will be to develop a possible worlds semantics for
$\Lang_\KB$ that links with the probabilistic setting by making the
following translation truth-preserving.

\begin{definition}[Translation]
  \label{definition:translation}
  For $c\in[\frac 12,1)\cap\Rat$, we define $c: \Lang_\KB \to \Lang$
  as follows.
  \[
  \renewcommand{\arraystretch}{1.2}
  \begin{array}{ccl@{\qquad}l}
    \top^c & := & \top
    \\
    p^c & := & p 
    \\
    (\neg \phi)^c & := & \neg \phi^c 
    \\
    (\phi \land \psi)^c & := & \phi^c \land \psi^c 
    \\
    (K_a\phi)^c & := & P_a(\phi^c) = 1
    & (= K_a\phi^c \text{ in } \Lang)
    \\
    (B_a\phi)^c & := & \textstyle P_a(\phi^c) > c
    & (= B_a^c\phi^c \text{ in } \Lang)
  \end{array}
  \]
\end{definition} 

Since we have seen that the probabilistic belief operator $B_a^c$ is
not a normal modal operator
(Theorem~\ref{theorem:belief}\eqref{item:B-not-normal}), we opt for a
neighborhood semantics for $\Lang_\KB$ \cite[Ch.~7]{Chellas:ml} with
an epistemic twist.

\begin{definition} 
  An \emph{epistemic neighborhood model} is a structure
  \[
  \M=(W,R,V,N)
  \]
  satisfying the following.
  \begin{itemize}
  \item $(W,R,V)$ is a finite multi-agent $\mathsf{S5}$ Kripke model
    (as in Definition~\ref{definition:epistemic-probability-model}).
    As before, we let 
    \[
    [w]_a:=\{v\in W\mid wR_av\}
    \]
    denote the $a$-equivalence class of world $w$.  This is the
    set of worlds $a$ cannot distinguish from $w$.

  \item $N : A\times W \to \wp(\wp(W))$ is a \emph{neighborhood
      function} that assigns to each agent $a\in A$ and world $w\in W$
    a collection $N_a(w)$ of sets of worlds---each such set called a
    \emph{neighborhood} of $w$---subject to the following conditions.
    \begin{description}
    \item[(kbc)] $\forall X \in N_a(w) : X \subseteq [w]_a$.

    \item[(kbf)] $\emptyset\notin N_a(w)$.
      
    \item[(n)] $[w]_a\in N_a(w)$.
      
    \item[(a)] $\forall v \in [w]_a : N_a(v) = N_a(w)$.

    \item[(kbm)] $\forall X \subseteq Y \subseteq [w]_a : 
      \text{ if } X \in N_a(w) \text{, then } Y \in N_a(w)$.
   \end{description}
  \end{itemize}
  A \emph{pointed epistemic neighborhood model} is a pair $(\M,w)$
  consisting of an epistemic neighborhood model $\M$ and a world $w$
  in $M$. 
\end{definition}

An epistemic neighborhood model is a variation of a neighborhood model that
includes an epistemic component $R_a$ for each agent $a$.
Intuitively, $[w]_a$ is the set of worlds agent $a$ knows to be
possible at $w$ and each $X\in N_a(w)$ represents a proposition that
the agent believes at $w$.  The condition that $R_a$ be an equivalence
relation ensures that knowledge is closed under logical consequence,
veridical (i.e., only true things can be known), positive
introspective (i.e., the agent knows what she knows), and negative
introspective (i.e., the agent knows what she does not know).

Property (kbc) ensures that the agent does not believe a proposition
$X\subseteq W$ that she knows to be false: if $X$ contains a world in
$w'\in(W-[w]_a)$ that the agent knows is not possible with respect to
the actual world $w$, then she knows that $X$ cannot be the case and
hence she does not believe $X$.  Property (kbf) ensures that no
logical falsehood is believed, while Property (n) ensures that every
logical truth is believed.  Property (a) ensures that $X$ is believed
if and only if it is known that $X$ is believed. Property (kbm) says
that belief is monotonic: if an agent believes $X$, then she believes
all propositions $Y\supseteq X$ that follow from $X$.

We now turn to the definition of truth for the language $\Lang_\KB$.

\begin{definition} 
  Let $\M = (W,R,V,N)$ be an epistemic neighborhood model.  We define
  a binary truth relation $\modelsn$ between a pointed epistemic
  neighborhood model $(\M,w)$ and $\Lang_\KB$-formulas and a function
  $\semn{\cdot}^\M:\Lang_\KB\to \wp(W)$ as follows.
  \begin{eqnarray*} 
    \semn{\phi}^\M & := & \{v\in W\mid \M,v\modelsn\phi\}
    \\
    \M, w \modelsn p & \text{ iff } & p \in V(w) 
    \\
    \M, w \modelsn \neg \phi & \text{ iff } & \M, w \not\modelsn \phi 
    \\
    \M, w \modelsn \phi\land\psi  & \text{ iff } 
    & \M, w \modelsn \phi \text{ and } \M, w \modelsn \psi
    \\
    \M, w \modelsn K_a \phi  & \text{ iff } & 
    [w]_a\subseteq\semn{\phi}^\M
    \\
    \M, w \modelsn B_a \phi  & \text{ iff } &
    [w]_a\cap \semn{\phi}^\M \in N_a(w)
  \end{eqnarray*}
  Validity of $\phi\in\Lang_\KB$ in an epistemic neighborhood model
  $\M$, written $\M\modelsn\phi$, means that $\M,w\modelsn\phi$ for
  each world $w\in W$.  Validity of $\phi\in\Lang_\KB$, written
  $\modelsn\phi$, means that $\M\modelsn\phi$ for each epistemic
  neighborhood model $\M$.  For a class $\mathcal{C}$ of epistemic
  neighborhood models, we write $\mathcal{C}\modelsn\phi$ to mean that
  $\M\modelsn\phi$ for each $\M\in\mathcal{C}$.
\end{definition}

Intuitively, $K_a\phi$ is true at $w$ iff $\phi$ holds at all worlds
epistemically possible with respect to $w$, and $B_a\phi$ holds at $w$
iff the epistemically possible $\phi$-worlds make up a neighborhood of
$w$.  Note that it follows from this definition that the dual for
belief $\check{B}_a \phi$ is true at $w$ iff
$[w]_a\cap\semn{\neg\phi}^\M\notin N_a(w)$.  The latter says that the
epistemically possible $\lnot\phi$-worlds do not make up a
neighborhood of $w$.

\subsection{Neighborhood and Probability Model Agreement}

Epistemic neighborhood models describe agent knowledge and belief.
Epistemic probability models can be used for the same purpose along
the lines we have discussed above once we establish a belief threshold
$c\in(0,1)\cap\Rat$.  This gives rise to a natural question: is there
some sense in which these two models for knowledge and belief can be
seen to agree?

\begin{definition}[Model Agreement]
  Let $\M=(W,R,V,N)$ be an epistemic neighborhood model. For an agent
  $a\in A$ and a threshold $c\in(0,1)\cap\Rat$, to say that a
  probability measure $P_a:\wp(W)\to[0,1]$ \emph{agrees with $\M$ for
    threshold $c$ (and agent $a$)} means we have the following:
  \begin{itemize}
  \item $P_a$ satisfies full support (i.e., $P_a(w)\neq0$ for each
    $w\in W$); and

  \item for each $w\in W$ and $X\subseteq[w]_a$, we have
    \[
    X\in N_a(w) \quad\text{iff}\quad
    P_{a,w}(X):=P_a(X|[w]_a)>c\enspace.
    \]
  \end{itemize}
  To say that an epistemic probability model $\M'=(W',R',V',P')$
  \emph{agrees with $\M$ for threshold $c$} means that
  $(W',R',V')=(W,R,V)$ and $P'_a$ agrees with $\M$ for threshold $c$
  for each agent $a\in A$.  If the threshold $c$ is not mentioned, it
  is assumed that $c=\frac 12$.
\end{definition}

Agreement for threshold $c$ between an epistemic neighborhood model
and an epistemic probability model makes the translation
$c:\Lang_\KB\to\Lang$ (Definition~\ref{definition:translation})
truth-preserving.

\begin{theorem}[Agreement]
  \label{theorem:agreement}\label{BettingTheorem}
  Fix $c\in(0,1)\cap\Rat$, an epistemic neighborhood model $\M$, and
  an epistemic probability model $\M'$. If $\M$ and $\M'$ agree for
  threshold $c$, then we have for each $\phi \in \Lang_\KB$ that
  \[
  \M,w\modelsn\phi \quad\text{iff}\quad
  \M',w\modelsp\phi^c\enspace.
  \]
\end{theorem}
\begin{proof}
  Induction on the structure of $\phi\in\Lang_\KB$. The non-modal
  cases are obvious.

  We first consider knowledge formulas. Assume $\M,w \modelsn
  K_a\psi$.  This means $[w]_a\subseteq\semn{\psi}^{\M}$. Applying the
  induction hypothesis, this is equivalent to
  $[w]_a\subseteq\semp{\psi^c}^{\M'}$.  By full support, the latter
  holds if and only if
  \[
  P_{a,w}(\semp{\psi^c}^{\M'}) =
  \frac{P_a(\semp{\psi^c}^{\M'}\cap[w]_a)}{P_a([w]_a)}=1 \enspace,
  \]
  which is what it means to have $\M',w\modelsp P_a(\psi^c)=1$.  Since
  $P_a(\psi^c)=1$ is what is abbreviated by $(K_a\psi)^c$, the result
  follows.

  Now we move to belief formulas. Assume $\M,w \modelsn B_a\psi$.
  This means that $[w]_a\cap\semn{\psi}^{\M}\in N_a(w)$.  Since $\M'$
  agrees with $\M$, the latter holds iff
  $P_{a,w}([w]_a\cap\semp{\psi^c}^{\M'})>c$.  But this is equivalent
  to $P_{a,w}(\semp{\psi^c}^{\M'})>c$, which is what it means to have
  $\M',w \modelsp P_a(\psi^c)>c$.  Since $P_a(\psi^c)>c$ is what is
  abbreviated by $(B_a\psi)^c$, the result follows.
\end{proof}


\subsection{Probability Measures on Epistemic Neighborhood Models}

In this subsection, we take up the question of agreement between
epistemic probability models and epistemic neighborhood models from
the point of view of the latter: given an epistemic neighborhood model
and a threshold $c$, can we find an agreeing epistemic probability
model for this threshold? As we will see, we have a full answer only
for the case $c=\frac 12$.  The case for $c\neq\frac 12$ is open,
though we will have some comments on this in the conclusion of the
paper.

To begin, we adapt an example due to Walley and Fine
\cite{WalleyFine1979:vomacp} to show that not every epistemic
neighborhood model gives rise to an agreeing probability measure.

\begin{theorem}[\cite{WalleyFine1979:vomacp}]
  \label{theorem:walleyfine}
  There exists an epistemic neighborhood model $\M$ that has no
  agreeing probability measure for any threshold $c\in(0,1)\cap\Rat$.
\end{theorem}
\begin{proof}
  We adapt Example~2 from \cite[pp.~344-345]{WalleyFine1979:vomacp} to
  the present setting.  Fix $c\in(0,1)\cap\Rat$.  Let
  $\Prop:=\{a,b,c,d,e,f,g\}$ and $A=\{0\}$.  Assume a single agent
  $0$.  Define:
\[
   \mathcal{X} := \{efg, abg, adf, bde, ace, cdg, bcf\}. 
\]
\[
   \mathcal{X'} := \{abcd, cdef, bceg, acfg, bdfg, abef, adeg \}. 
\]
Notation:  $xyz$ for $\{x,y,z\}$.
\[
  \mathcal{Y}  := \{ Y \mid \exists X \in  \mathcal{X}: X \leq Y \leq W \}. 
\]
Let $\M:=(W,R,V,N)$ be defined by $W:=\Prop$, $R_0=W\times W$,
$V(w)=\{w\}$, and for all $w \in W$, $N_0(w) = \mathcal{Y}$. 
Check that   $\mathcal{X'} \cap \mathcal{Y} = \emptyset$. 
So $\M$ is a neighborhood model. 

Toward a contradiction, suppose there exists a probability function $P$
that agrees with $\M$. Since each letter $p\in W$ occurs in exactly
three of the seven members of $\mathcal{X}$, we have:
\[
   \sum_{X \in \mathcal{X}} P_0(X) = \sum_{p\in W}3\cdot P_0(\{p\}). 
\]
Since each letter $p\in W$ occurs in exactly four of the seven
members of $\mathcal{X'}$, we have:
\[
\sum_{X \in \mathcal{X'}} P_0(X) = \sum_{p\in W} 4\cdot P_0(\{p\}). 
\]
On the other hand, from the fact that $P_0 (X) > P_o (W-X)$ for all members
$X$ of $\mathcal{X}$ we get: 
\[
   \sum_{X \in \mathcal{X}} P_0 (X) >  
   \sum_{X \in \mathcal{X}} P_0 (W - X) = \sum_{X \in \mathcal{X'}} P_0 (X). 
\]
Contradiction. So no such $P_0$ exists.
\end{proof}

%The model $\M$ from the proof of
%Theorem~\ref{theorem:KB-probability-incompleteness} allows us to
%divide the space into overlapping pieces in two ways: one way
%$\mathcal{X}$ in which an agreeing measure must assign probability
%above threshold to each piece in $\mathcal{X}$ and another way
%$\mathcal{Y}$ in which the same agreeing measure must assign
%probability at or below threshold to the pieces in $\mathcal{Y}$.
%Since the pieces of $\mathcal{X}$ and those of $\mathcal{Y}$ use each
%world in $\M$ the same number of times, the sum of the probabilities
%of the pieces must be the same for $\mathcal{X}$ as it is for
%$\mathcal{Y}$.  So we see that the neighborhoods in $\M$ are arranged
%in a manner that is fundamentally inconsistent with an agreeing
%probability measure.

Question: what are the additional restrictions on the neighborhood
function that one must impose in order to guarantee the existence of
an agreeing probability measure for a given threshold
$c\in(0,1)\cap\Rat$?  For $c=\frac 12$, the restrictions are known.
For thresholds $c\neq\frac 12$, the question is open.

The restrictions needed for $c=\frac 12$ were studied first in the
form of a purely probabilistic semantics (i.e., something like
epistemic probability models and not something like our epistemic
neighborhood models).  To our knowledge, Lenzen \cite{Lenzen1980:gwuw}
is the first complete study of the restrictions needed in such a
purely probabilistic framework over a unary modal language similar to
$\Lang_\KB$. The conditions Lenzen proposed are targeted to satisfy
the conditions of Scott's theorem, which is the key result that gives
rise to a probability measure in the completeness proof for Lenzen's
logic.  Here we state the required restrictions in the language of our
epistemic neighborhood models.  Later we will make the link with
Lenzen's axiomatic system when we consider axiomatic theories in the
language $\Lang_\KB$ targeted to our epistemic neighborhood models.

\begin{definition}[Extra Properties for ``Mid-Threshold'' Models]
  \label{definition:extra-properties}
  Let $\M=(W,R,V,N)$ be an epistemic neighborhood model.  
  For $m\in\Int^+$ and sets
  of worlds $X_1,\dots,X_m$ and $Y_1,\dots,Y_m$,
  we write
  \begin{equation}
    X_1,\dots,X_m\mathbb{I}_aY_1,\dots,Y_m
    \label{eq:semantic-lenzen}
  \end{equation}
  to mean that for each $v\in W$, the number of $X_i$'s containing $v$
  is less than or equal to the number of $Y_i$'s containing $v$. This
  is the semantic counterpart of the formula from
  Definition~\ref{definition:segerberg-notation}.  We may write
  $(X_i\mathbb{I}_aY_i)_{i=1}^m$ as an abbreviation for
  \eqref{eq:semantic-lenzen}.  Also, we write
  $(X_i\mathbb{E}_aY_i)_{i=1}^m$ to mean that both
  $(X_i\mathbb{I}_aY_i)_{i=1}^m$ and $(Y_i\mathbb{I}_aX_i)_{i=1}^m$
  hold, and we allow the notation with $\mathbb{E}_a$ to be used in a
  form as in \eqref{eq:semantic-lenzen}.  The following is a list of
  properties that $\M$ may satisfy.
  \begin{description}
  \item[(d)] $\forall X \in N_a(w): [w]_a - X \notin  N_a(w)$.

    % \item[(lt)] $\forall \phi\in\Lang_\KB:\text{ if }
    %   [w]_a-\semn{\phi}\notin
    %   N_a(w) \text{, then } \semn{\phi}\cap[w]_a\in N_a(w)$.
 
  \item[(sc)] $\forall X,Y\subseteq[w]_a$: if $[w]_a-X\notin N_a(w)$
    and $X\subsetneq Y$, then $Y\in N_a(w)$.

  \item[(scott)] $\forall m\in\Int^+,\forall
    X_1,\dots,X_m,Y_1,\dots,Y_m\subseteq[w]_a:$
    \[
    \renewcommand{\arraystretch}{1.3}
    \begin{array}{ll}
      \text{if }
      &
      \begin{array}[t]{l}
        X_1,\dots,X_m\mathbb{I}_aY_1,\dots,Y_m\quad\text{and}
        \\
        X_1\in N_a(w)\quad\text{and}
        \\
        \forall i\in\{2,\dots,m\}:
        [w]_a-X_i\notin N_a(w) \enspace\text{,}
      \end{array}
      \\
      \text{then }
      &
      \exists j\in\{1,\dots,m\}: Y_j\in N_a(w)\enspace\text{.}
    \end{array}
    \]
  \end{description}
  To say an epistemic neighborhood model is \emph{mid-threshold} means
  it satisfies (d), (sc), and (scott).  We may drop the word
  ``epistemic'' in referring to mid-threshold epistemic neighborhood
  models.  Pointed versions of mid-threshold neighborhood models are
  defined in the obvious way.
\end{definition}

Property (d) ensures that beliefs are consistent in the sense that the
agent does not believe both $X$ and its complement $[w]_a-X$.
Property (sc) is a form of ``strong commitment'': if the agent does
not believe the complement $[w]_a-X$, then she must believe any
strictly weaker $Y$ implied by $X$.  Property (scott) is a version of
the syntactic scheme (Scott) from
Definition~\ref{definition:scott-schemes}.

By inspection of the model $\M$ from the proof of
Theorem~\ref{theorem:KB-probability-incompleteness}, one may verify
that that no $X\in\mathcal{X}$ is contained in the complement $W-X'$
of some $X'\in\mathcal{X}$. Numbering the members of $\mathcal{X}$ as
$X_1,\dots,X_7$ and the members of $\mathcal{Y}$ as $Y_1,\dots,Y_7$,
we see that $\M$ satisfies
\[
(X_i\mathbb{I}_0Y_i)_{i=1}^7,\enspace X_1\in N_0(a)\text{,
  and}\enspace \forall i\in\{2,\dots,7\}:W-X_i\notin N_0(a)\enspace,
\]
which is the antecedent of property (scott) from
Definition~\ref{definition:extra-properties}.  However, $\M$ does not
satisfy
\[
\exists j\in\{1,\dots,7\}:Y_j\in N_0(a)\enspace,
\]
which is the corresponding consequent of the indicated instance of
(scott).  So we see that if we were to restrict ourselves to the class
of epistemic neighborhood models satisfying this property, we would no
longer able to use $\M$ as a counterexample to the claim that not
every epistemic neighborhood model gives rise to an agreeing
probability measure.  Of course ruling out $\M$ as a counterexample to
this claim does not prove the claim.  However, utilizing (scott) in
conjunction with (d) and (sc), we are able to prove the claim.  This
proof makes crucial use of Scott's theorem.

In preparation for the statement of Scott's theorem, we recall some
well-known notions from linear algebra.  For a nonempty set $S$, let
$L(S)$ denote the $S$-dimensional real vector space whose vectors
consist of functions $x:S\to\mathbb{R}$ and whose operations of vector
addition and scalar multiplication are defined coordinate-wise: given
vectors $x,y:S\to\mathbb{R}$ and a scalar real $r\in\mathbb{R}$, the
vector $(x+y):S\to\mathbb{R}$ is defined by $(x+y)(s):=x(s)+y(s)$ for
each coordinate $s\in S$ and the vector $(r\cdot x):S\to\mathbb{R}$ is
defined by $(r\cdot x)(s):=r\cdot x(s)$ for each coordinate $s\in S$.
Note that we have just used the usual notational clash wherein the $+$
or $\cdot$ symbol on one side of an equation refers to the vector
operation and the same symbol on the other side of the same equation
refers to the operation in $\mathbb{R}$. Other common notational
abbreviations such as omission of $\cdot$'s and writing $-x$ for
$(-1)\cdot x$ will be used. To say that a vector $x:S\to\mathbb{R}$ is
\emph{rational} means that all of its coordinates (i.e., values) are
rational numbers.  To say a set $X\subseteq L(S)$ of vectors is
rational means that every vector in $X$ is rational, and to say that
$X$ is \emph{symmetric} means that $X=-X:=\{-x\mid x\in X\}$.  A
\emph{linear functional on $L(S)$} is a function $f:L(S)\to\mathbb{R}$
satisfying the following property of \emph{linearity\/}: for each
$r_1,r_2\in\mathbb{R}$ and $x,y\in L(S)$, we have
$f(r_1x+r_2y)=r_1\cdot f(x)+r_2\cdot f(y)$.

\begin{theorem}[{\cite[Theorem 1.2]{Sco64:JMP}}]
  \label{theorem:scott}
  Let $S$ be a finite nonempty set and $X$ be a finite, rational,
  symmetric subset of $L(S)$. For each $N\subseteq X$, there exists a
  linear functional $f$ on $L(S)$ that \emph{realizes $N$}, meaning
  \[
  N = \{x\in X\mid f(x)\geq 0\}\enspace,
  \]
  if and only if the following conditions are satisfied:
  \begin{itemize}
  \item for each $x\in X$, we have $x\in N$ or $-x\in N$; and

  \item for each integer $n\geq 0$ and $x_0,\dots,x_n\in N$, we have
  \[
  \sum_{i=0}^n x_i = 0
  \quad\Rightarrow\quad
  -x_0\in N\enspace.
  \]
  \end{itemize}
\end{theorem}

We use Scott's theorem to show that mid-threshold models always give
rise to an agreeing probability measure.  That is, the neighborhood
function of mid-threshold models picks out exactly those neighborhoods
that may be assigned a probability exceeding $\frac 12$.  Many of the
key ideas of the proof of the following result are due to Lenzen
\cite{Lenzen1980:gwuw}. However, the argument we present here is in a
streamlined, modern form and in the language of our epistemic
neighborhood models. Despite this difference (and the necessary work
we had to undertake to translate these results into this modern form),
we are happy to credit Professor Lenzen for the following result.

\begin{theorem}[\cite{Lenzen1980:gwuw}]
  \label{theorem:lenzen}
  Let $\M=(W,R,V,N)$ be a mid-threshold epistemic neighborhood model.
  For each agent $a\in A$, there exists a probability measure
  $P_a:\wp(W)\to[0,1]$ agreeing with $\M$ for threshold $\frac 12$ and
  agent $a$; that is,
  \begin{itemize}
  \item $P_a$ satisfies full support (i.e., $P_a(w)\neq0$ for each
    $w\in W$); and

  \item for each $w\in W$ and $X\subseteq[w]_a$, we have
    \[
    \textstyle X\in N_a(w) \quad\text{iff}\quad
    P_{a,w}(X):=P_a(X|[w]_a)>\frac 12\enspace.
    \]
  \end{itemize}
\end{theorem}
\begin{proof}
  We credit Lenzen \cite{Lenzen1980:gwuw} for this proof, though we
  herein provide an original reformulation of his work within the
  setting of the epistemic neighborhood models introduced in this
  paper.  Proceeding, for $a\in A$ and $w\in W$, define
  $S_{a,w}:=[w]_a$.  For each $X\subseteq S_{a,w}$, define the
  relative complement $X':=S_{a,w}-X$ and let
  $\iota(X):S_{a,w}\to\{0,1\}$ be the characteristic function of $X$:
  \[
  \iota(X)(s):=\begin{cases}
    1 & \text{if } s\in X, \\
    0 & \text{otherwise.}
  \end{cases}
  \]
  We consider the following finite subsets of $L(S_{a,w})$:
  \begin{eqnarray*}
    \mathcal{A}_{a,w} &:=& 
    \{\iota(X)\mid X\subseteq S_{a,w}\}\enspace, 
    \\
    \mathcal{B}_{a,w} &:=& 
    \{\iota(X)-\iota(X')\mid X\subseteq S_{a,w} 
    \text{ \& } X'\notin N_a(w)\}\enspace,
    \\
    \mathcal{N}_{a,w} &:=&
    \mathcal{A}_{a,w}\cup\mathcal{B}_{a,w}\enspace,
    \\
    \mathcal{X}_{a,w} &:=&
    \mathcal{N}_{a,w}\cup(-\mathcal{N}_{a,w})\enspace.
  \end{eqnarray*}
  It is easy to see that $\mathcal{N}_{a,w}\subseteq\mathcal{X}_{a,w}$
  and that $\mathcal{X}_{a,w}$ is a finite, rational, and symmetric
  subset of $L(S_{a,w})$.  We wish to show that $\mathcal{N}_{a,w}$
  and $\mathcal{X}_{a,w}$ satisfy the conditions of
  Theorem~\ref{theorem:scott}. First, we note that
  $x\in\mathcal{X}_{a,w}$ implies $x\in\mathcal{N}_{a,w}$ or
  $-x\in\mathcal{N}_{a,w}$ by the definition of $\mathcal{X}_{a,w}$.

  For the second condition of Theorem~\ref{theorem:scott}, suppose we
  are given an integer $n\geq 0$ such that
  $x_0,\dots,x_n\in\mathcal{N}_{a,w}$ and $\sum_{i=0}^n x_i=0$.  We
  wish to show that $-x_0\in\mathcal{N}_{a,w}$.  Proceeding, there
  exists an integer $\ell$ satisfying:
  \begin{eqnarray*}
    0\leq i\leq\ell & \text{implies} &
    x_i=\iota(X_i)-\iota(X_i')\in\mathcal{B}_{a,w} 
    \enspace,\text{and}
    \\
    \ell<i\leq n & \text{implies} &
    x_i=\iota(X_i)\in\mathcal{A}_{a,w} \enspace.
  \end{eqnarray*}
  Toward a contradiction, assume there exists $i>\ell$ with $x_i\neq
  0$.  Then for $x^*:=\sum_{i=\ell+1}^n x_i$, we have $x^*(s)\geq 0$
  for all $s\in S_{a,w}$, and there exists $s^*\in S_{a,w}$ with
  $x^*(s^*)>0$. Hence
  \[
  \textstyle \sum_{i=0}^\ell x_i=
  \sum_{i=0}^\ell\bigl(\iota(X_i)-\iota(X'_i)\bigr)=-x^*\enspace,
  \]
  where $x^*(s^*)<0$ and $-x^*(s)\leq 0$ for all $s\in S_{a,w}$.  So
  for each $s\in S_{a,w}$, the number of the sets in the list
  $X_0',\dots,X_\ell'$ containing $s$ is greater than or equal to the
  number of the sets in the list $X_0,\dots,X_\ell$ containing $s$.
  Further, $s^*$ is a member of strictly more sets in the former list
  than those in the latter.  By renumbering, we may assume that
  $s^*\in X_0'-X_0$.  Then we have
  \[
  X_0\cup\{s^*\},X_1,\dots,X_\ell \mathbb{I}_a
  X_0',X_1',\dots,X_\ell'\enspace.
  \]
  Since $X_0',\dots,X_\ell'\notin N_a(w)$, it follows by (scott) that
  $X_0\cup\{s^*\}\notin N_a(w)$.  But $X_0\subsetneq
  X_0\cup\{s^*\}\notin N_a(w)$ and $X_0'\notin N_a(w)$, which violates
  (sc).  Conclusion: $i>\ell$ implies $x_i=0$.  But then we have
  $\sum_{i=0}^n x_i=\sum_{i=0}^\ell x_i$.  Since
  $x_i=\iota(X_i)-\iota(X_i')$ for $i\leq\ell$, it follows that
  $\sum_{i=0}^\ell\iota(X_i)=\sum_{i=0}^\ell\iota(X_i')$. But the
  latter is what it means to have $(X_i\mathbb{E}_aX_i')_{i=0}^\ell$.
  Since $X_i'\notin N_a(w)$ for $i\leq\ell$ by the definition of
  $\mathcal{B}_{a,w}$, it follows by (scott) that $X_0\notin N_a(w)$.
  But then $\iota(X'_0)-\iota(X_0)=-x_0\in\mathcal{B}_{a,w} \subseteq
  \mathcal{N}_{a,w}$, as desired.

  So we may apply Theorem~\ref{theorem:scott}: there exists a linear
  functional $f_{a,w}$ on $L(S_{a,w})$ that realizes
  $\mathcal{N}_{a,w}$.  That is,
  \[
  \mathcal{N}_{a,w}=\{x\in\mathcal{X}_{a,w}\mid f_{a,w}(x)\geq
  0\}\enspace.
  \]
  Define $g_{a,w}:\wp(S_{a,w})\to\mathbb{R}$ by the composition
  $g_{a,w}(X):=f_{a,w}(\iota(X))$.  This function satisfies a few
  important properties.
  \begin{enumerate}
  \item \label{prop:XinN} $X\in N_a(w)$ iff $g_{a,w}(X)>g_{a,w}(X')$.

    Suppose $X\in N_a(w)$. Then $X'\notin N_a(w)$ by (d).  Hence
    $\iota(X)-\iota(X')\in\mathcal{B}_{a,w}$ and
    $\iota(X')-\iota(X)\notin\mathcal{B}_{a,w}$.  Since $S_{a,w}\in
    N_a(w)$ by (n), it follows that $X\neq\emptyset=S_{a,w}'$.  But
    then the coordinates of $\iota(X')-\iota(X)$ contain at least one
    $1$ and at least one $-1$.  Since every $x\in\mathcal{A}_{a,w}$
    has coordinates that are $1$'s or $0$'s only, it follows that
    $\iota(X')-\iota(X)\notin\mathcal{N}_{a,w}$.  As
    $\iota(X)-\iota(X')\in\mathcal{B}_{a,w}\subseteq
    \mathcal{N}_{a,w}$ and $f_{a,w}$ is linear and realizes
    $\mathcal{N}_{a,w}$, it follows that $g_{a,w}(X)\geq g_{a,w}(X')$
    and $g_{a,w}(X')\ngeq g_{a,w}(X)$.  That is,
    $g_{a,w}(X)>g_{a,w}(X')$.

    Conversely, suppose $g_{a,w}(X)>g_{a,w}(X')$. Since $f_{a,w}$ is
    linear and realizes $\mathcal{N}_{a,w}$, it follows that
    $\iota(X')-\iota(X)\notin\mathcal{N}_{a,w}\supseteq
    \mathcal{B}_{a,w}$. Applying the definition of
    $\mathcal{B}_{a,w}$, we have $X\in N_a(w)$.

  \item \label{prop:emptyS} $g_{a,w}(S_{a,w})>g_{a,w}(\emptyset)=0$.

    We have $g_{a,w}(\emptyset)=f_{a,w}(0)=0$ by the linearity of
    $f_{a,w}$.  Since $S_{a,w}\in N_a(w)$ by (n), it follows that
    $g_{a,w}(S_{a,w})>g_{a,w}(\emptyset)$ by property \ref{prop:XinN}.

  \item \label{prop:zerotoS} If $0\leq g_{a,w}(X)\leq
    g_{a,w}(S_{a,w})$.
    
    Since $\iota(X)\in\mathcal{A}_{a,w}\subseteq \mathcal{N}_{a,w}$
    and $f_{a,w}$ realizes $\mathcal{N}_{a,w}$, we have
    $g_{a,w}(X)\geq 0$.  So each $X\subseteq S_{a,w}$ satisfies
    $g_{a,w}(X)\geq 0$.  From this it follows by the linearity of
    $f_{a,w}$ that for each $X\subseteq S_{a,w}$, we have
    \[
    \textstyle g_{a,w}(X)=\sum_{v\in X}g_{a,w}(\{v\}) \leq\sum_{v\in
      S_{a,w}}g_{a,w}(\{v\}) =g_{a,w}(S_{a,w})\enspace.
    \]

  \item \label{prop:additivity} If $X,Y\subseteq S_{a,w}$ and $X\cap
    Y=\emptyset$, then $g_{a,w}(X\cup Y)=g_{a,w}(X)+g_{a,w}(Y)$.
    
    By the linearity of $f_{a,w}$.

  \item \label{prop:fullsupport} $\emptyset\neq X\subseteq S_{a,w}$
    implies $g_{a,w}(X)>0$.

    Suppose $\emptyset\neq X\subseteq S_{a,w}$. By property
    \ref{prop:emptyS}, it suffices to prove the result for $X\neq
    S_{a,w}$. Toward a contradiction, assume $g_{a,w}(X)=0$ for
    $\emptyset\subsetneq X\subsetneq S_{a,w}$.  By property
    \ref{prop:additivity}, we have
    $g_{a,w}(S_{a,w})=g_{a,w}(X)+g_{a,w}(X')=g_{a,w}(X')$.  Since
    $f_{a,w}$ is linear and realizes $\mathcal{N}_{a,w}$ and
    \[
    \iota(X')-\iota(S_{a,w}) = -(\iota(S_{a,w})-\iota(X')) =
    -\iota(X)\in\mathcal{X}_{a,w}\enspace,
    \]
    we obtain $-\iota(X)\in\mathcal{N}_{a,w}$.  But
    $\emptyset\subsetneq X\subsetneq S_{a,w}$ implies that $-\iota(X)$
    has coordinates containing at least one $-1$ and at least one $0$.
    Since members of $\mathcal{A}_{a,w}$ have coordinates made up of
    $0$'s and $1$'s, members of $\mathcal{B}_{a,w}$ have coordinates
    made up of $-1$'s and $1$'s, and
    $\mathcal{N}_{a,w}=\mathcal{A}_{a,w} \cup \mathcal{B}_{a,w}$, it
    cannot be the case that $-\iota(X)\in\mathcal{N}_{a,w}$.
    Contradiction.  Conclusion: $g_{a,w}(X)>0$.
  \end{enumerate}

  Now take $v\in[w]_a$.  Since $N_a(v)=N_a(w)$ by (a), it follows that
  $g_{a,w}$ also realizes $\mathcal{N}_{a,v}$.  So, letting $[W]_a$ be
  the set $\{[w]_a\mid w\in W\}$ of $a$-equivalence classes, let
  $h:[W]_a\to W$ be a choice function that selects for each class
  $[w]_a\in[W]_a$ a representative $h([w]_a)\in[w]_a$. Using a
  notational clash that ought to be harmless, we define a new function
  $h_{a,w}:\wp([w]_a)\to\mathbb{R}$ by setting
  $h_{a,w}(X):=g_{a,h([w]_a)}(X)$.  Obviously, $v\in[w_a]$ implies
  $h_{a,v}=h_{a,w}$.  Finally, we define $P_a:\wp(W)\to[0,1]$ by
  \[
  P_a(X):= \sum_{[w]_a\in[W]_a} \frac {h_{a,w}(X\cap[w]_a)}
  {h_{a,w}([w]_a)}\enspace.
  \]
  Note that by property \ref{prop:emptyS}, the denominator
  $h_{a,w}([w]_a)$ is always nonzero.

  We prove that $P_a$ is a probability measure on $\wp(W)$ satisfying
  full support. First, $P_a$ satisfies the Kolmogorov axioms over the
  finite algebra $\wp(W)$: we have $P_a(X)\geq 0$ by property
  \ref{prop:zerotoS}, $P_a(W)=1$ by property \ref{prop:emptyS} and the
  definition of $P_a$, and $P_a(X\cup Y)=P_a(X)+P_a(Y)$ for disjoint
  $X$ and $Y$ by property \ref{prop:additivity} and the definition of
  $P_a$.  Second, full support follows by property
  \ref{prop:fullsupport}.

  Finally, for $X\subseteq[w]_a$, we have by property \ref{prop:XinN}
  that $X\in N_a(w)$ iff $h_{a,w}(X)>h_{a,w}(X')$.  But the latter
  holds iff we have (making use of property \ref{prop:additivity})
  that
  \[
  2\cdot h_{a,w}(X)>h_{a,w}(X)+h_{a,w}(X')=h_{a,w}([w]_a)\enspace.
  \]
  By property \ref{prop:emptyS}, the definition of $P_a$, and the fact
  that $X\subseteq[w]_a$, the above inequality holds iff
  \[
  P_a(X)=\frac{h_{a,w}(X)}{h_{a,w}([w]_a)}>\textstyle \frac
  12\enspace.
  \]
\end{proof}

\begin{corollary}
  \label{corollary:lenzen}
  Let $\M=(W,R,V,N)$ be a mid-threshold epistemic neighborhood model.
  There exists an epistemic probability model $\N=(W,R,V,P)$ that
  agrees with $\M$ for threshold $\frac 12$.
\end{corollary}
\begin{proof} 
  For each $a\in A$, let $P_a$ be the measure given by
  Theorem~\ref{theorem:lenzen}.
\end{proof}

\subsection{Epistemic Neighborhood Models from Probability Measures}
\label{Section:BeliefBet}

In the last subsection, we investigated the question of whether an
epistemic neighborhood model gives rise to an agreeing epistemic
probability model.  In this section, we look at this question the
other way around: given an epistemic probability model and a threshold
$c$, is there an agreeing epistemic neighborhood model?  As we will
see, the answer is always ``yes.''

\begin{definition}
  Given an epistemic probability model $\M = (W,R,V,P)$ and a
  threshold $c\in[\frac 12,1)\cap\Rat$, we define the structure $\M^c
  := (W,R,V,N^c)$ by setting
  \[
  N^c_a(w) := \{X\subseteq[w]_a\mid P_{a,w}(X)>c\}\enspace.
  \]
\end{definition}

Intuitively, agent $a$ believes a proposition $X$ at world $w$ (i.e.,
$X\in N^c_a(w)$) if and only if $X$ is epistemically possible (i.e.,
$X\subseteq[w]_a$) and the probability $a$ assigns to $X$ at world $w$
exceeds the threshold (i.e., $P_{a,w}(X)>c$).

\begin{lemma}[Correctness]
  \label{lemma:correctness}
  Fix $c\in(0,1)\cap\Rat$.  If $\M$ is an epistemic probability model,
  then $\M^c$ is an epistemic neighborhood model.  Furthermore,
  $\M^{\frac 12}$ is a mid-threshold neighborhood model.
\end{lemma}
\begin{proof}
  We verify that $N^c_a$ satisfies the required properties.
  \begin{itemize}
  \item For (kbc), $X\in N^c_a(w)$ implies $X\subseteq[w]_a$ by
    definition.

  \item For (kbf), $P_{a,w}(\emptyset)=0<c$, so $\emptyset\notin N^c_a(w)$.

  \item For (n), $P_{a,w}([w]_a) = 1 > c$, so
    $[w]_a \in N^c_a(w)$.

  \item For (a), suppose $X \in N^c_a(w)$ and $v\in [w]_a$.  Then
    $P_{a,w}(X)>c$.  Since $v\in [w]_a$ implies $[w]_a = [v]_a$, we
    have
    \[
    P_{a,w}(X) = 
    \frac{P_a(X\cap[w]_a)}{P_a([w]_a)} =
    \frac{P_a(X\cap[v]_a)}{P_a([v]_a)} =
    P_{a,v}(X) \enspace.
    \]
    Hence $P_{a,v}(X)>c$, so $X \in N^c_a(v)$.

  \item For (kbm), suppose $X \in N^c_a(w)$.  Then $P_{a,w}(X)>c$.
    Hence if $Y$ satisfies $X \subseteq Y \subseteq [w]_a$, we have
    $P_{a,w}(Y)>c$ and so $Y \in N^c_a(w)$.
  \end{itemize}
  So $\M^c$ is an epistemic neighborhood model.  We now show that
  $\M^{\frac 12}$ satisfies the additional required properties.
  \begin{itemize}
  \item For (d), assume $c\in[\frac 12,1)\cap\Rat$ and $X \in
    N^c_a(w)$.  Then $P_{a,w}(X) > c$, and therefore $P_{a,w}([w]_a -
    X) \leq 1-c\leq c$. Hence $[w]_a - X \notin N^c_a(w)$.

  \item For (sc), assume $X':=[\Gamma]_a-X\notin N^{\frac 12}_a(w)$
    and $X\subsetneq Y\subseteq[\Gamma]_a$.  From the first
    assumption, we have $P_{a,w}(X') \leq \frac 12$, and therefore
    that $P_{a,w}(X)\geq \frac 12$.  Applying the second assumption,
    $P_{a,w}(Y) > P_{a,w}(X)\geq \frac 12$, and hence $X\in N^{\frac
      12}_a(w)$.

  \item For (scott), we assume $c\in(0,\frac 12]\cap\Rat$ along with the
    following:
    \begin{eqnarray}
      &&
      (X_i\mathbb{I}_aY_i)_{i=1}^m
      \label{eq:prop-l:E2} 
      \\ &&
      X_1^{w,a}\in N^c_a(w,i) 
      \label{eq:prop-l:X1} 
      \\ && 
      \forall i\in\{2,\dots,m\}: [\Gamma]_a-X\notin N^c_a(w) 
      \label{eq:prop-l:Xcs} 
    \end{eqnarray}
    From \eqref{eq:prop-l:E2} it follows that
    \begin{equation}
      P_{a,w} (X_1)+\cdots+P_{a,w}(X_m)\leq
      P_{a,w}(Y_1)+\cdots+ P_{a,w}(Y_m)
      \label{eq:sums-eq}
    \end{equation}
    The argument for this is similar to an argument for \eqref{eq:sum}
    in proof of Theorem~\ref{theorem:belief}\eqref{item:B-Len}.  From
    \eqref{eq:prop-l:X1}, we have $P_{a,w}(X_1)>c$.  From
    \eqref{eq:prop-l:Xcs}, we have for each $i\in\{2,\dots,m\}$ that
    $P_{a,w}([w]_a-X_i)\leq c$ and therefore that $P_{a,w}(X_i)\geq
    1-c\geq c$ since $c\in(0,\frac 12]\cap\Rat$.  Hence the left side
    of \eqref{eq:sums-eq} exceeds $mc$.  Since every summand on the
    right side of the inequality is positive and $mc>0$, it follows
    that at least one member of the right side of \eqref{eq:sums-eq}
    must exceed $c$.  That is, there exists $j\in\{1,\dots,m\}$ such
    that $P_{a,w}(Y_j) > c$ and hence $Y_j\in N^c_a(w)$.  \qedhere
  \end{itemize}
\end{proof}

\begin{theorem}
  Let $c\in(0,1)\cap\Rat$ and $\M=(W,R,V,P)$ be an epistemic
  probability model. The epistemic neighborhood model
  $\M^c=(W,R,V,N^c)$ agrees with $\M$ for threshold $c$.
\end{theorem}
\begin{proof}
  By definition of $N^c$.
\end{proof}

\section{Calculi for Belief as Willingness to Bet}
\label{Section:Calculi}

We now consider the axiomatic link with both epistemic neighborhood
models and epistemic probability models.  We study two calculi: the
calculus $\KB$ of epistemic neighborhood models, and the calculus
$\KBeq$ of mid-threshold neighborhood models.  $\KB$ is sound but not
complete for the probability interpretation for any threshold.
$\KBeq$ is both sound and complete for the probability interpretation
with threshold $c=\frac 12$.

$\KBeq$ is our modern reformulation of Lenzen's \cite{Lenzen1980:gwuw}
calculus for the logic of knowledge as probabilistic certainty and
belief as probability exceeding threshold $\frac 12$.  Lenzen's
intended semantic structures are something like epistemic probability
models.  Our intended semantic structures are our mid-threshold
neighborhood models, though there is a natural link with epistemic
probability models via Theorem~\ref{theorem:lenzen}.  In fact,
many of the main ideas of our proof of
Theorem~\ref{theorem:lenzen} are translations of Lenzen's ideas
into the language of our epistemic neighborhood models. Since we have
reworked his ideas using our own approach and modern modal notions, it
is difficult to determine whether we have introduced novel
mathematical results on top of Lenzen's existing work.  Therefore, we
are happy to credit Professor Lenzen for the soundness and
completeness of $\KBeq$ and for Theorem~\ref{theorem:lenzen}.
However, we do think that it is worth our effort to provide this
modern reformulation of his results.  In particular, we believe that
in using semantic structures more familiar to the modern modal
logician, our modern reformulation of Lenzen's results will make the
mathematical details of Lenzen's work more accessible to a modern
English-language audience.  We also hope that our use of the modal
neighborhood structures will suggest directions for further study of
qualitative probability via tools from modal logic.

\begin{definition}
  \label{definition:calculi}
  We define the following theories in the language $\Lang_\KB$.
  \begin{itemize}
  \item $\KB$ is defined in Table~\ref{table:KB}.

  \item $\KBeq$ is obtained from $\KB$ by adding (D), (SC), and (Scott)
    from Table~\ref{table:additional-schemes}.

  \item $\KBeqm$ is obtained from $\KBeq$ by omitting (BF) and (KBM).
  \end{itemize}
\end{definition}

We will see later in Theorem~\ref{theorem:KBminus} that $\KBeq$ and
$\KBeqm$ derive the same theorems.

\begin{table}[ht]
  \begin{center}
    \textsc{Axiom Schemes}\\[.4em]
    \renewcommand{\arraystretch}{1.3}
    \begin{tabular}[t]{cl}
      (CL) &
      Schemes of Classical Propositional Logic
      \\
      (KS5) &
      $\mathsf{S5}$ axiom schemes for each $K_a$
      \\
      (BF) &
      $\lnot B_a\bot$
      \\
      (N) &
      $B_a\top$
      \\
      (Ap) &
      $B_a\phi\to K_aB_a\phi$
      \\
      (An) &
      $\lnot B_a\phi\to K_a\lnot B_a\phi$
      \\
      (KBM) &
      $K_a(\phi\to\psi)\to(B_a\phi\to B_a\psi)$
    \end{tabular}
    \renewcommand{\arraystretch}{1.0}
    \\[1em]
    \textsc{Rules}\vspace{-.5em}
    \[
    \begin{array}{c}
      \phi\to\psi \quad \phi
      \\\hline
      \psi
    \end{array}\;\text{\footnotesize(MP)}
    \qquad
    \begin{array}{c}
      \phi
      \\\hline
      K_a\phi
    \end{array}\;\text{\footnotesize(MN)}
    \]
  \end{center}
  \caption{The theory $\KB$}
  \label{table:KB}
\end{table}

\begin{table}[ht]
  \begin{center}
    \renewcommand{\arraystretch}{1.3}
    \begin{tabular}[t]{cl}
      % (LT) &
      % $\check B_a\phi\to B_a\phi$
      % \\
      (D) &
      $B_a\phi\to \check B_a\phi$
      \\
      (SC) &
      $\check B_a\phi \land 
      \check K_a(\lnot\phi\land\psi) \to 
      B_a(\phi\lor\psi)$
      \\
      (Scott) &
      $\textstyle [(\phi_i\mathbb{I}_a\psi_i)_{i=1}^m
      \land B_a\phi_1 \land \bigwedge_{i=2}^m \check B_a\phi_i] \to
      \bigvee_{i=1}^m B_a\psi_i$
    \end{tabular}
  \end{center}
  \caption{Additional axiom schemes for the theory $\KBeq$}
  \label{table:additional-schemes}
\end{table}

\subsection{Results for the Basic Calculus \texorpdfstring{$\KB$}{KB}}

The following result shows that if we restrict attention to provable
statements whose only modality is single-agent belief $B_a\phi$, then
$\KB$ is an extension of the minimal modal logic $\mathsf{EMN45}+\lnot
B_a\bot=\mathsf{EMN45}+(\text{BF})$ obtained by adding
$\mathsf{S5}$-knowledge and the knowledge-belief connection principles
(Ap), (An), and (KBM).\footnote{$\mathsf{EMN45}+(\text{BF})$ is the
  logic of single-agent belief (without knowledge) having Schemes (CL)
  (Table~\ref{table:KB}), M
  (Theorem~\ref{theorem:KBgt-derivables}\eqref{derivables:Band-andB}), (N)
  (Table~\ref{table:KB}), 4
  (Theorem~\ref{theorem:KBgt-derivables}\eqref{derivables:pos-belief}), 5
  (Theorem~\ref{theorem:KBgt-derivables}\eqref{derivables:neg-belief}), and
  (BF) (Table~\ref{table:KB}) along with Rules (MP)
  (Table~\ref{table:KB}) and RE
  (Theorem~\ref{theorem:KBgt-derivables}\eqref{derivables:RE}). This is a
  ``monotonic'' system of modal logic satisfying positive and negative
  belief introspection (4 and 5) and the property (BF) that falsehood
  $\bot$ is not believed. See \cite[Ch.~8]{Chellas:ml} for details on
  naming minimal modal logics.} The modal theory $\KBeq$ is therefore
a similar knowledge-inclusive extension of
$\mathsf{EMND45}+(\text{BF})+(\text{Scott})$ that adds the additional
connection principle
(SC).\footnote{$\mathsf{EMND45}+(\text{BF})+(\text{Scott})$ is
  $\mathsf{EMN45}+(\text{BF})$ plus Schemes (D) and (Scott) from
  Table~\ref{table:additional-schemes}.}  In
Section~\ref{section:kbeq}, we will show that $\KBeq$ is the modal
logic for probabilistic belief with threshold $c=\frac 12$.

\begin{theorem}[$\KB$ Derivables]
  \label{theorem:KBgt-derivables}
  We have each of the following.
  \begin{enumerate}
  \item $\KB\vdash K_a\phi\to B_a\phi$.
    \label{derivables:KBC}

    ``Knowledge implies belief.''

  \item $\KB\vdash B_a(\phi\land\psi)\to(B_a\phi\land B_a\psi)$.
    \label{derivables:Band-andB}

    This is ``Scheme M'' \cite[Ch.~8]{Chellas:ml}.

  \item $\KB\vdash K_a\phi\land B_a\psi\to B_a(\phi\land\psi)$.
    \label{derivables:andB-Band}

    If the antecedent $K_a\phi$ were replaced by $B_a\phi$, then we
    would obtain ``Scheme C'' \cite[Ch.~8]{Chellas:ml}.  So we do not
    have Scheme C outright but instead a knowledge-weakened version:
    in order to conclude belief of a conjunction from belief of one of
    the conjuncts, the other conjunct must be known (and not merely
    believed, as is required by the stronger, non-$\KB$-provable
    Scheme C).

  \item $\KB\vdash K_a(\phi\to\psi)\to(\check B_a\phi\to\check B_a\psi)$.
    \label{derivables:check-M}

    This is the dual version of our (KBM).

    \item $\KB\vdash B_a\phi\to B_aB_a\phi$.
    \label{derivables:pos-belief}

    This is ``Scheme 4'' \cite[Ch.~8]{Chellas:ml}.

  \item $\KB\vdash \lnot B_a\phi\to B_a\lnot B_a\phi$.
    \label{derivables:neg-belief}

    This is ``Scheme 5'' \cite[Ch.~8]{Chellas:ml}.

  \item $\KB\vdash B_a\phi\leftrightarrow K_aB_a\phi$.
    \label{derivables:B-KB}

    This says that belief and knowledge of belief are equivalent.

  \item $\KB\vdash \lnot B_a\phi\leftrightarrow K_a\lnot B_a\phi$.
    \label{derivables:nB-KnB}

    This says that non-belief and knowledge of non-belief are equivalent.

  \item $\KB\vdash\phi$ implies $\KB\vdash B_a\phi$.
    \label{derivables:B-nec}

    This is the rule of Modus Ponens (or Modal Necessitation),
    sometimes called ``Rule RN'' \cite[Ch.~8]{Chellas:ml}.

  \item $\KB\vdash\phi\to\psi$ implies $\KB\vdash B_a\phi\to B_a\psi$.
    \label{derivables:Bimp}

    This is ``Rule RM'' \cite[Ch.~8]{Chellas:ml}.

  \item $\KB\vdash\phi\to\psi$ implies $\KB\vdash\check
    B_a\phi\to\check B_a\psi$.
    \label{derivables:check-Bimp}
    
    This is the dual version of RM.
    
  \item $\KB\vdash\phi\leftrightarrow\psi$ implies
    $\KB\vdash B_a\phi\leftrightarrow B_a\psi$.
    \label{derivables:RE}

    This is ``Rule RE'' \cite[Ch.~8]{Chellas:ml}.

  \item $\KB\vdash\phi\to\bot$ implies $\KB\vdash\lnot
    B_a\phi$. \label{derivables:GBF}

    This says that no self-contradictory sentence is believed.  This
    may be viewed as a certain generalization of (BF)
    (Table~\ref{table:KB}).
  \end{enumerate}
\end{theorem}
\begin{proof}
  We reason in $\KB$. For \ref{derivables:KBC}, we have $K_a\phi\to
  K_a(\top\to\phi)$ by elementary modal reasoning.  But then from
  this, $B_a\top$ by (N), and $K_a(\top\to\phi)\to(B_a\top\to
  B_a\phi)$ by (KBM), it follows by classical reasoning that we have
  $K_a\phi\to B_a\phi$.

  For \ref{derivables:Band-andB}, we derive
  \begin{equation}
    K_a((\phi\land\psi)\to\phi)\to(B_a(\phi\land\psi)\to B_a\phi)
    \label{eq:derivables:Band-andB}
  \end{equation}
  by (KBM), and the antecedent of \eqref{eq:derivables:Band-andB} by (CL) and (MN).
  Therefore, the consequent of \eqref{eq:derivables:Band-andB} is derivable by (MN).
  By a similar argument,  $B_a(\phi\land\psi)\to B_a\psi$ is derivable.  By classical
  reasoning,
  \ref{derivables:Band-andB} is derivable.
  
  For \ref{derivables:andB-Band}, we derive
  \begin{eqnarray}
    &&
    K_a\phi\to K_a(\psi\to(\phi\land\psi)) \enspace\text{and}
    \label{eq:derivables:andB-Band1}
    \\
    &&
    K_a(\psi\to(\phi\land\psi))\to(B_a\psi\to B_a(\phi\land\psi)) \enspace\text{.}
    \label{eq:derivables:andB-Band2}
  \end{eqnarray}
  \eqref{eq:derivables:andB-Band1} follows by $\mathsf{S5}$ reasoning.
  \eqref{eq:derivables:andB-Band2} follows by (KBM).  Applying classical reasoning
  to \eqref{eq:derivables:andB-Band1} and \eqref{eq:derivables:andB-Band2},
  we obtain
  \[
  K_a\phi\to (B_a\psi\to B_a(\phi\land\psi))\enspace,
  \]
  from which \ref{derivables:andB-Band} follows by classical reasoning.

  For \ref{derivables:check-M}, we derive
  \begin{eqnarray}
    &&
    K_a(\phi\to\psi)\to K_a(\lnot\psi\to\lnot\phi) \enspace\text{and}
    \label{eq:derivables:check-M1}
    \\
    &&
    K_a(\lnot\psi\to\lnot\phi)\to(B_a\lnot\psi\to B_a\lnot\phi) \enspace\text{.}
    \label{eq:derivables:check-M2}
  \end{eqnarray}
  \eqref{eq:derivables:check-M1} follows by $\mathsf{S5}$ reasoning.
  \eqref{eq:derivables:check-M2} follows by (KBM).  Applying classical reasoning
  to \eqref{eq:derivables:check-M1} and \eqref{eq:derivables:check-M2},
  we obtain
  \[
  K_a(\phi\to\psi)\to(B_a\lnot\psi\to B_a\lnot\phi)\enspace,
  \]
  from which \ref{derivables:check-M} follows by classical reasoning
  (just contrapose the consequent).

  \ref{derivables:pos-belief} follows by (Ap) and \ref{derivables:KBC}.
  \ref{derivables:neg-belief} follows by (An) and \ref{derivables:KBC}.
  \ref{derivables:B-KB} follows by by (Ap) for the right-to-left and (KS5) for the left-to-right.
  \ref{derivables:nB-KnB} follows by (An) for the right-to-left and (KS5) for the left-to-right.
  \ref{derivables:B-nec} follows by (MN) and \ref{derivables:KBC}.
  \ref{derivables:Bimp} follows by (MN) and (KBM).
  \ref{derivables:check-Bimp} follows by contraposition, (MN), (KBM), and contraposition.
  \ref{derivables:RE} follows from \ref{derivables:Bimp} by classical reasoning.

  For \ref{derivables:GBF}, we have
  \begin{equation}
    K_a(\phi\to\bot)\to(B_a\phi\to B_a\bot)
    \label{eq:GBF}
  \end{equation}
  by (KBM).  Therefore, if $\phi\to\bot$ is provable, it follows by
  (MN) that the antecedent of \eqref{eq:GBF} is as well.  By (MP), the
  consequent $B_a\phi\to B_a\bot$ is provable.  Applying (BF) and
  classical reasoning, it follows by contraposition that $\lnot
  B_a\phi$ is provable.
\end{proof}

\begin{theorem}[$\KB$ Neighborhood Soundness and Completeness]
  \label{theorem:KB-neighborhood-soundness}\label{theorem:KB-neighborhood-completeness}
  $\KB$ is sound and complete with respect to the class $\mathcal{C}$
  of epistemic neighborhood models:
  \[
  \forall\phi\in\Lang_\KB:\quad\KB\vdash\phi
  \quad\Leftrightarrow\quad
  \mathcal{C}\modelsn\phi
  \enspace.
  \]
\end{theorem}
\begin{proof}
  By induction on the length of derivation.  We first
  verify soundness of the axioms.
  \begin{itemize}
  \item Validity of (CL) immediate. Validity of (KS5) follows because
    the $R_a$'s are equivalence relations \cite{BlaRijVen:ml}.

  \item Scheme (BF) is valid: $\modelsn\lnot B_a\bot$.

    $\semn{\bot}=\emptyset\notin N_a(w)$ by (kbf).  Hence
    $\M,w\not\modelsn B_a\bot$.

  \item Scheme (N) is valid: $\modelsn B_a\top$.

    $\semn{\top}\cap[w]_a=[w]_a\in N_a(w)$ by (n).  Hence
    $\M,w\modelsn B_a\top$.

  \item Scheme (Ap) is valid: $\modelsn B_a\phi\to K_a B_a\phi$.

    Suppose $\M,w\modelsn B_a\phi$. Then $[w]_a\cap\semn{\phi}\in
    N_a(w)$.  Take $v\in[w]_a$.  We have $[v]_a=[w]_a$ because $R_a$
    is an equivalence relation, and we have $N_a(v)=N_a(w)$ by (a).
    Hence $[v]_a\cap\semn{\phi}\in N_a(v)$; that is, $\M,v\modelsn
    B_a\phi$.  Since $v\in [w]_a$ was chosen arbitrarily, we have
    shown that $[w]_a\subseteq\semn{B_a\phi}$.  Hence $\M,w\modelsn
    K_a B_a\phi$.

  \item Scheme (An) is valid: $\modelsn \lnot B_a\phi\to K_a\lnot
    B_a\phi$.

    Replace $B_a\phi$ by $\lnot B_a\phi$ and $\in$ by $\notin$ in the
    argument for the previous item.

  \item Scheme (KBM) is valid: $\modelsn
    K_a(\phi\to\psi)\to(B_a\phi\to B_a\psi)$.

    Suppose $\M,w\modelsn K_a(\phi\to\psi)$ and $\M,w\modelsn
    B_a\phi$.  This means $[w]_a\subseteq\semn{\phi\to\psi}$ and
    $[w]_a\cap\semn{\phi}\in N_a(w)$. But then
    \[
    [w]_a\cap\semn{\phi}\subseteq
    [w]_a\cap\semn{\phi}\cap\semn{\phi\to\psi}\subseteq
    [w]_a\cap\semn{\psi}\enspace.
    \]
    Hence $[w]_a\cap\semn{\psi}\in N_a(w)$ by (kbm).  That is,
    $\M,w\modelsn B_a\psi$.
  \end{itemize}
  That validity is closed under applications of the rules MP and MN
  follows by the standard arguments \cite{BlaRijVen:ml}.  This
  completes the proof of soundness.

  Before we prove completeness, we first prove an important result
  that we will use tacitly throughout the completeness proof
  proper. For each $a\in A$, let $M_a$ be the set of all
  $\Lang_\KB$-formulas having one of the forms $K_a\varphi$, $\lnot
  K_a\varphi$, $B_a\varphi$, or $\lnot B_a\varphi$.  We prove the
  following \emph{Modal-Assumption Deduction Theorem\/}: for each
  finite $F\subseteq M_a$, we have
  \begin{center}
    $F\vdash_\KB\varphi$ \quad{}iff\quad $\vdash_\KB(\bigwedge
    F)\to\varphi$\enspace.
  \end{center}
  The right-to-left direction straightforward. The proof of the
  left-to-right direction is by induction on the length of derivation.
  All cases are standard except for the induction step in which (MN)
  is applied, so we focus on this case.  Suppose $F\vdash_\KB
  K_a\varphi$ is derived by (MP) from $\varphi$ such that
  $F\vdash_\KB\varphi$. By the induction hypothesis, we have
  $\vdash_\KB (\bigwedge_{\chi\in F}\chi)\to\varphi$.  By (MN) and
  $\mathsf{K}$ reasoning, we have $\vdash_\KB (\bigwedge_{\chi\in
    F}K_a\chi)\to K_a\varphi$.  However, it also follows by
  $\mathsf{S5}$ reasoning (using schemes $\mathsf{4}$ and
  $\mathsf{5}$), scheme (Ap), scheme (An), and the fact that
  $F\subseteq M_a$ that we have $\vdash_\KB\chi\to K_a\chi$ for each
  $\chi\in F$.  Hence $\vdash_\KB(\bigwedge_{\chi\in
    F}\chi)\to(\bigwedge_{\chi\in F}K_a\chi)$. Conclusion:
  $\vdash_\KB(\bigwedge_{\chi\in F}\chi)\to K_a\varphi$.

  To prove completeness, it suffices to show that
  $\KB\nvdash\lnot\theta$ implies $\theta$ is satisfiable at a pointed
  epistemic neighborhood model. For two sets $F$ and $F'$ of
  $\Lang_\KB$-formulas, to say that $F$ is \emph{maxcons in $F'$}
  means that $F\subseteq F'$, the set $F$ is $\KB$-consistent (i.e.,
  for no finite $G\subseteq F$ do we have $\vdash_\KB (\bigwedge
  G)\to\bot$), and adding any formula $\psi\in F'$ not already in $F$
  will produce a $\KB$-inconsistent (i.e., not $\KB$-consistent) set.
  Let $S$ be the set of subformulas of $\theta$, including $\theta$
  itself.  Let $C$ be the Boolean closure of $S$; that is, $C$ is the
  smallest extension of $S$ that contains the propositional constants
  $\top$ (truth) and $\bot$ (falsehood), or their abbreviations in
  $\Lang_\KB$ if they are not primitive, and is closed under the
  Boolean connectives (e.g., negation, conjunction, disjunction)
  definable in the language.  For each $a\in A$, define:
  \begin{eqnarray*}
    W &:=& 
    \{w\subseteq C \mid w \text{ is maxcons in $C$}\}, 
    \\{}
    w_a &:=& w\cap M_a \text{ for } w\in W,
    \\{}
    [\varphi] &:=& \{w\in W\mid \varphi\in w\}
    \text{ for } \varphi\in C,
    \\
    R_a &:=& \{(w,v)\in W^2\mid w_a=v_a\},
    \\{}
    V(w) &:=& \Prop\cap w,
    \\
    N_a(w) &:=&
    \{ X\subseteq[w]_a \mid 
    \exists\varphi\in C:(X=[\varphi]\cap[w]_a
    \text{ \& }
    w_a\nvdash_\KB \lnot B_a\varphi)
    \}.
  \end{eqnarray*}

  Define $S':=S\cup\{\lnot\chi\mid\chi\in S\}\cup\{\bot,\top\}$, and
  note that $S'$ is finite.  In what follows, we may make tacit use of
  the following \emph{Identity Lemma\/}: for each $u,v\in W$, if
  $u\cap S'=v\cap S'$, then $u=v$.  The proof is by a normal form
  argument: every $\chi\in C$ is a Boolean combination of members of
  $S$, and every such Boolean combination is $\KB$-provably equivalent
  to a disjunction of conjunctions of maxcons subsets of $S'$.  We may
  also make tacit use of the following \emph{Definability Lemma\/}:
  for each $w\in W$ and each $X\subseteq[w]_a$, defining
  \[
  \textstyle
  X^d :=\bigvee_{v\in X}\bigwedge_{\chi\in v\cap S'} \chi\enspace,
  \]
  it follows that $X^d\in C$ and $[X^d]\cap[w]_a=X$. For the proof,
  first note that $X^d\in C$ because $C$ is closed under Boolean
  operations and $S'\subseteq C$.  So assume $u\in[X^d]\cap[w]_a$,
  which implies $X^d\in u$.  Since $u$ is maxcons in $C\supseteq S'$,
  we have for some $v\in X$ that $\bigwedge_{\chi\in v\cap S'}\chi\in
  u$ and hence $v\cap S'\subseteq u$.  Since $u$ is maxcons in $C$ and
  hence maxcons in $S'$ and $S'$ is closed under the operation
  ${\sim}:\Lang_\KB\to\Lang_\KB$ defined by
  \[
  {\sim}\varphi :=
  \begin{cases}
    \phantom{\lnot}\psi & \text{if }\varphi=\lnot\psi\\
    \lnot\varphi & \text{otherwise,} \\
  \end{cases}
  \]
  it follows that $u\cap S'=v\cap S'$, and hence $u=v\in X$ by the
  Identity Lemma.  Conversely, suppose $u\in X\subseteq[w]_a$.  By the
  definition of $X^d$, we have $\KB\vdash(\bigwedge_{\chi\in u\cap
    S'}\chi)\to X^d$ and therefore $X^d\in u$ because $u$ is maxcons
  in $C$. Hence $u\in[X^d]\cap[w]_a$ because $u\in X\subseteq[w]_a$.

  Our definitions above specify the structure $\M=(W,R,V,N)$.  $W$ is
  nonempty because $\theta$ is consistent and so may be extended to a
  maxcons $w_\theta\in W$.  Since $S'$ is finite, it follows by the
  Identity Lemma that $W$ is finite.  Further, $R_a$ is an equivalence
  relation. So to conclude that $M$ is an epistemic neighborhood
  model, all that remains is for us to show that $N_a$ satisfies the
  neighborhood function properties.
  \begin{description}
  \item[(kbc)] $X\in N_a(w)$ implies $X\subseteq[w]_a$.

    By definition.

  \item[(bf)] $\emptyset\notin N_a(w)$.

    Choose $\varphi\in C$ satisfying $[\varphi]\cap[w]_a=\emptyset$.
    It follows that $w_a\vdash_\KB\varphi\to\bot$, since otherwise we
    could extend $w_a\cup\{\varphi\}$ to some
    $v\in[\varphi]\cap[w]_a$, which would contradict
    $[\varphi]\cap[w]_a=\emptyset$.  So by (MN), we have
    $w_a\vdash_\KB K_a(\varphi\to\bot)$ and hence $w_a\vdash_\KB
    B_a\varphi\to B_a\bot$ by (KBM).  Since $w_a\vdash\lnot B_a\bot$
    by (BF), it follows that $w_a\vdash_\KB \lnot B_a\varphi$.  So we
    have shown that $w_a\vdash_\KB \lnot B_a\varphi$ for each
    $\varphi\in C$ satisfying $[\varphi]\cap[w]_a=\emptyset$.
    Conclusion: $\emptyset\notin N_a(w)$.

  \item[(n)] $[w]_a\in N_a(w)$.

    $w_a\vdash_\KB B_a\top$ by (N), so $w_a\nvdash_\KB\lnot B_a\top$.
    Hence $[\top]\cap[w]_a=[w]_a\in N_a(w)$.

  \item[(a)] $v\in[w]_a$ implies $N_a(v)=N_a(w)$.

    $v\in[w]_a$ implies $[v]_a=[w]_a$ and $v_a=w_a$.  Therefore for
    each $X\subseteq[v]_a=[w]_a$, we have $\varphi\in C$ satisfying
    $X=[\varphi]\cap[w]_a$ and $w_a\nvdash_\KB \lnot B_a\varphi$ iff
    $X=[\varphi]\cap[v]_a$ and $v_a\nvdash_\KB \lnot B_a\varphi$.
    Hence $N_a(v)=N_a(w)$.

  \item[(kbm)] If $X\subseteq Y\subseteq[w]_a$ and $X\in N_a(w)$, then
    $Y\in N_a(w)$.

    Suppose $X\subseteq Y\subseteq[w]_a$ and $X\in N_a(w)$. Then there
    is $\varphi\in C$ satisfying $X=[\varphi]\cap[w]_a$ and
    $w_a\nvdash_\KB \lnot B_a\varphi$.  Since $X\subseteq Y$, it
    follows that $[\varphi]\cap[w]_a\subseteq[Y^d]\cap[w]_a$. From
    this we obtain that $w_a\vdash_\KB\varphi\to Y^d$, since otherwise
    we could extend $w_a\cup\{\varphi,\lnot Y^d\}$ to some
    $v\in[\varphi]\cap[\lnot Y^d]\cap[w]_a$, which would contradict
    $[\varphi]\cap[w]_a\subseteq[Y^d]\cap[w]_a$.  Hence $w_a\vdash_\KB
    K_a(\varphi\to Y^d)$ by (MN) and so $w_a\vdash_\KB B_a\varphi\to
    B_a Y^d$ by (KBM).  Since $w_a\nvdash_\KB \lnot B_a\varphi$, we
    have $w_a\nvdash_\KB \lnot B_aY^d$.  Hence $Y\in N_a(w)$.
  \end{description}
  So $M$ is indeed and epistemic neighborhood model. To complete our
  overall argument, it suffices to prove the \emph{Truth Lemma\/}: for
  each $\varphi\in C$ and $w\in W$, we have $\varphi\in w$ iff
  $\M,w\modelsn\varphi$. The argument is by induction on the
  construction of $\varphi\in C$.  Boolean cases are straightforward,
  so we restrict our attention to the modal cases: formulas
  $B_a\varphi$ and $K_a\varphi$ in $C$.  Note that by the definition
  of $C$ as the Boolean closure of the set $S$ of subformulas of
  $\theta$, either of $B_a\varphi\in C$ or $B_a\varphi\in C$ implies
  $\varphi\in C$.

  Suppose $B_a\varphi\in w$. Then $w_a\vdash_\KB B_a\varphi$ and
  therefore $w_a\nvdash_\KB\lnot B_a\varphi$.  Hence
  $[\varphi]\cap[w]_a\in N_a(w)$ by the definition of $N_a$ and the
  fact that $\varphi\in C$. Applying the induction hypothesis,
  $[\varphi]=\semn{\varphi}^\M$, so $\semn{\varphi}^\M\cap[w]_a\in
  N_a(w)$.  But this is what it means to have $\M,w\modelsn
  B_a\varphi$.

  Conversely, assume $\M,w\modelsn B_a\varphi$ for $B_a\varphi\in
  C$. This means $\semn{\varphi}^\M\cap[w]_a\in N_a(w)$.  By the
  induction hypothesis and the fact that $\varphi\in C$, we have
  $[\varphi]=\semn{\varphi}^\M$, so $[\varphi]\cap[w]_a\in N_a(w)$.
  By the definition of $N_a$, there exists $\psi\in C$ such that
  $w_a\nvdash_\KB \lnot B_a\psi$ and
  $[\varphi]\cap[w]_a=[\psi]\cap[w]_a$.  But then
  $w_a\vdash_\KB\psi\to\varphi$, for otherwise we could extend
  $w_a\cup\{\psi,\lnot\varphi\}$ to some $v\in[w]_a$ such that
  $v\in[\psi]\cap[w]_a$ and $v\notin[\varphi]\cap[w]_a$, contradicting
  $[\varphi]\cap[w]_a=[\psi]\cap[w]_a$.  Applying (MN), we have
  $w_a\vdash_\KB K_a(\psi\to\varphi)$ and hence $w_a\vdash_\KB
  B_a\psi\to B_a\varphi$ by (KBM).  Since $w_a\nvdash_\KB \lnot
  B_a\psi$, it follows that $w_a\nvdash_\KB \lnot B_a\varphi$.  Hence
  $\lnot B_a\varphi\notin w$. But since $B_a\varphi\in C$ and $w$ is
  maxcons in $C$, we conclude that $B_a\varphi\in w$.

  Now suppose $K_a\varphi\in w$.  Then for each $v\in [w]_a$, we have
  that $K_a\varphi\in v$ and therefore $\varphi\in v$ by $\mathsf{S5}$
  reasoning (using scheme $\mathsf{T}$) and the fact that $v$ is
  maxcons in $C$.  But then we have shown that
  $[w]_a\subseteq[\varphi]$.  Since $\varphi\in C$, it follows by the
  induction hypothesis that $[w]_a\subseteq\semn{\varphi}^\M$, which
  is what it means to have $\M,w\modelsn K_a\varphi$.

  Conversely, assume $\M,w\modelsn K_a\varphi$ for $K_a\varphi\in
  C$. It follows that $[w]_a\subseteq\semn{\varphi}^\M$.  By the
  induction hypothesis, $[w]_a\subseteq[\varphi]$.  But then
  $w_a\vdash_\KB\varphi$, for otherwise we could extend
  $w_a\cup\{\lnot\varphi\}$ to some $v\in[w]_a$ satisfying
  $v\notin[\varphi]$, contradicting $[w]_a\subseteq[\varphi]$.  By
  (MN), we have $w_a\vdash_\KB K_a\varphi$.  Since $K_a\varphi\in C$
  and $w$ is maxcons in $C$, it follows that $K_a\varphi\in w$.
\end{proof}

Since $\KB$ is sound and complete with respect to the class of
epistemic neighborhood models, we would expect that in light of
Theorem~\ref{theorem:walleyfine} that $\KB$ is at most sound for the
probability interpretation.

\begin{theorem}[$\KB$ Probability Soundness]
  $\KB$ is sound for any threshold $c\in(0,1)\cap\Rat$ with respect to
  the class of epistemic probability models:
  \[
  \textstyle \forall c\in(0,1)
  \cap\Rat,\forall\phi\in\Lang_\KB:\quad
  \KB\vdash\phi
   \quad\Rightarrow\quad
  {}\modelsp\phi^c \enspace.
  \]
\end{theorem}
\begin{proof}
  Theorems~\ref{theorem:knowledge} and \ref{theorem:belief}.
\end{proof}

\begin{theorem}[$\KB$ Probability Incompleteness]
  \label{theorem:KB-probability-incompleteness}
  $\KBc$ is incomplete for all thresholds $c\in(0,1)\cap\Rat$ with
  respect to the class of epistemic probability models:
  \[
  \textstyle 
  \exists\phi\in\Lang_\KB,
  \forall c\in(0,1)
  \cap\Rat:\quad
  {}\modelsp\phi^c
  \quad\text{and}\quad
  \KB\nvdash\phi\enspace.
  \]
\end{theorem}
\begin{proof}
  Take $\M$ as in the proof of Theorem~\ref{theorem:walleyfine}.  Let
  $\sigma$ be the modal formula describing $(\M,a)$: informally (and
  easily formalizable),
  \[
  \textstyle \sigma \quad:=\quad a\bar{b}\cdots\bar{g}\land
  KW\land(\bigwedge_{Z\in N(a)} BZ)\land
  (\bigwedge_{Z'\in\wp(W)-N_0(a)}\lnot BZ')\enspace.
  \]
  We have $\M,w\modelsn\sigma$ so that $\not\modelsn\lnot\sigma$ and
  therefore $\KB\nvdash\lnot\sigma$ by
  Theorem~\ref{theorem:KB-neighborhood-completeness}.  By the proof of
  Theorem~\ref{theorem:walleyfine}, there is no probability measure
  agreeing with $\M$ for any threshold.  Hence
  $\modelsp\lnot\sigma^c$. So $\phi:=\lnot\sigma$ gives us the desired
  formula.
\end{proof}

\subsection{Results for the Mid-Threshold Calculus \texorpdfstring{$\KBeq$}{KB.5}}
\label{section:kbeq}

We first show that the $\KB$ schemes (BF) and (KBM) are redundant in
the theory $\KBeq$.

\begin{theorem}
  \label{theorem:KBminus}
  $\KBeqm$ and $\KBeq$ derive the same theorems:
  \[
  \forall\phi\in\Lang_\KB:\quad
  \KBeqm\vdash\varphi \quad\Leftrightarrow\quad \KBeq\vdash\varphi\enspace.
  \]
\end{theorem}
\begin{proof}
  It suffices to prove that the schemes (BF) and (KBM) are derivable
  in $\KBeqm$. For (KBM), we have by
  Definition~\ref{definition:segerberg-notation} that the formula
  $\phi\mathbb{I}_a\psi$ is just
  \begin{equation}
    K_a\bigl(\;
    \underbrace{(\lnot\phi\land\lnot\psi) \lor (\lnot\phi\land\psi)}_{C_0} \lor
    \underbrace{(\phi\land\psi)}_{C_1}
    \;\bigr)\enspace,
    \label{eq:KaPhitoPsi}
  \end{equation}
  where we have explicitly indicated the subformulas $C_0$ and $C_1$
  used in the notation of
  Definition~\ref{definition:segerberg-notation}.  Semantically,
  \eqref{eq:KaPhitoPsi} says that in each of $a$'s accessible worlds,
  $\psi$ is true whenever $\phi$ is true.  Now reasoning within
  $\KBeqm$, it follows that $K_a(\phi\to\psi)$ is provably equivalent
  to $\phi\mathbb{I}_a\psi$.  But then from $K_a(\phi\to\psi)$ and
  $B_a\phi$, we may derive $\phi\mathbb{I}_a\psi$ and $B_a\phi$, from
  which we may derive $B_a\psi$ by (Scott). Hence (KBM) is derivable.

  We now consider (BF).  The formula $\bot\to\lnot\top$ is a classical
  tautology and hence $K_a(\bot\to\lnot\top)$ follows by (MN).  Hence
  by an instance of (KBM), which can be defined away in terms of
  axioms other than (BF) as above, it follows that $B_a\bot\to
  B_a\lnot\top$ and therefore that $\lnot B_a\lnot\top\to\lnot
  B_a\bot$.  Also by (N), (D), and (MP), we may derive $\lnot
  B_a\lnot\top$.  That is, (BF) is derivable.
\end{proof}

\begin{theorem}[$\KBeq$ Neighborhood Soundness and Completeness]
  \label{theorem:KBeq}
  $\KBeq$ is sound and complete with respect to the class $\Ceq$ of
  mid-threshold neighborhood models:
  \[
  \forall\phi\in\Lang_\KB:\quad
  \KBeq\vdash\phi \quad\Leftrightarrow\quad \Ceq\modelsn\phi \enspace.
  \]
\end{theorem}
\begin{proof}
  Soundness is by induction on the length of derivation.  Most cases
  are as in the proof of
  Theorem~\ref{theorem:KB-neighborhood-completeness}.  We only need
  consider the remaining axiom schemes.
  \begin{itemize}
  \item Scheme (D) is valid: $\modelsn B_a\phi\to \check B_a\phi$.

    Suppose $\M,w\modelsn B_a\phi$. This means
    $[w]_a\cap\semn\phi\in N_a(w)$.  By (d), 
    \[
    [w]_a\cap\semn{\lnot\psi}=[w]_a-\semn\phi =
    [w]_a-([w]_a\cap\semn\phi)\notin N_a(w)\enspace.
    \]
    But this is what it means to have $\M,w\modelsn\check B_a\phi$.

  \item Scheme (SC) is valid: $\modelsn \check B_a\phi \land \check
    K_a(\lnot\phi\land\psi) \to B_a(\phi\lor\psi)$.

    Suppose $\M,w\modelsn\check B_a\phi$ and $\M,w\modelsn \check
    K_a(\lnot\phi\land\psi)$.  It follows that
    \[
    [w]_a-([w]_a\cap\semn{\phi})=[w]_a\cap\semn{\lnot\phi}\notin
    N_a(w)
    \]
    and that there exists $v\in[w]_a$ satisfying
    $\M,v\models\lnot\phi\land\psi$.  But then
    $[w]_a\cap\semn{\phi\lor\psi}\supsetneq[w]_a\cap\semn{\phi}$ and
    therefore $[w]_a\cap\semn{\phi\lor\psi}\in N_a(w)$ by (sc). Hence
    $\M,w\models B_a(\phi\lor\psi)$.

  \item Scheme (Scott) is valid:
    \[
    \modelsn \textstyle [(\phi_i\mathbb{I}_a\psi_i)_{i=1}^m
    \land B_a\phi_1 \land \bigwedge_{i=2}^m \check B_a\phi_i] \to
    \bigvee_{i=1}^m B_a\psi_i\enspace.
    \]

    Suppose $(\M,w)$ satisfies the antecedent of scheme (Scott).  It
    follows that each $v\in[w]_a$ satisfies at least as many
    $\phi_i$'s as $\psi_i$'s, that $[w]_a\cap\semn{\psi_1}\in N_a(w)$,
    and that $[w]_a-\semn{\phi_k}\notin N_a(w)$ for each
    $k\in\{2,\dots,m\}$.  Hence
    \[
    [w]_a\cap\semn{\phi_1},\dots,[w]_a\cap\semn{\phi_m}\mathbb{I}_a
    [w]_a\cap\semn{\psi_1},\dots, [w]_a\cap\semn{\psi_m}\enspace,
    \]
    from which it follows by (scott) that $[w]_a\cap\semn{\psi_j}\in
    N_a(w)$ for some $j\in\{1,\dots,m\}$.  Hence $\M,w\modelsn
    B_a\psi_j$, and thus $\M,w\modelsn\bigvee_{i=1}^m B_a\psi_i$.
  \end{itemize}
  Soundness has been proved.  

  For completeness, it suffices to show that the model $\M$ defined as
  in the proof of
  Theorem~\ref{theorem:KB-neighborhood-completeness}---except that now
  derivability is always taken with respect to $\KBeq$---is a
  mid-threshold neighborhood model; the rest of the argument is as in
  that proof, \emph{mutatis mutandis}.  Most of the properties of $\M$
  are shown in that proof.  What remains is for us to show that $\M$
  also satisfies (d), (sc), and (scott).
  \begin{description}
  \item[(d)] $X\in N_a(w)$ implies $X'\notin N_a(w)$, where
    $X':=[w]_a-X$.

    Suppose $X\in N_a(w)$.  Then we have $\varphi\in C$ such that
    $X=[\varphi]\cap[w]_a$ and $w_a\nvdash_\KB \lnot B_a\varphi$.  By
    (D), it follows that $w_a\nvdash_\KB\lnot \check B_a\varphi$.
    Choose $\psi\in C$ satisfying $X'=[\psi]\cap[w]_a$.  We have
    $w_a\vdash_\KB\lnot\varphi\to\psi$, since otherwise we could
    extend $w_a\cup\{\lnot\varphi,\lnot\psi\}$ to a $v\in[w]_a$ such
    that $v\in[\lnot\varphi]\cap[w]_a=X'$ and
    $v\in[\lnot\psi]\cap[w]_a=X$, contradicting $X'\cap X=\emptyset$.
    By (MP), we have $w_a\vdash_\KB K_a(\lnot\varphi\to\psi)$ and
    therefore $w_a\vdash_\KB B_a\lnot\varphi\to B_a\psi$ by (KBM).
    Since $\lnot\check B_a\varphi=\lnot\lnot B_a\lnot\varphi$ and
    $w_a\nvdash_\KB\lnot \check B_a\varphi$, it follows that
    $w_a\nvdash_\KB\lnot B_a\psi$.  So we have shown that
    $w_a\nvdash_\KB\lnot B_a\psi$ for each $\psi\in C$ satisfying
    $X'=[\psi]\cap[w]_a$.  Conclusion: $X'\notin N_a(w)$.

  \item[(sc)] If $X'\notin N_a(w)$ and $X\subsetneq Y\subseteq[w]_a$,
    then $Y\in N_a(w)$.

    So assume that $X'\notin
    N_a(w)$ and $X\subsetneq Y\subseteq[w]_a$. Define
    \begin{eqnarray*}
      \varphi &:=&
      \textstyle \bigvee_{v\in X}\bigwedge_{\chi\in v\cap S'}\chi
      \enspace,\text{ and}
      \\
      \psi &:=&
      \textstyle \bigvee_{v\in Y-X}\bigwedge_{\chi\in v\cap S'}\chi\enspace.
    \end{eqnarray*}
    It follows that $\{\varphi,\psi\}\subseteq C$,
    $[\varphi]\cap[w]_a=X$, and $[\psi]\cap[w]_a=Y-X$.  Hence
    $\varphi\lor\psi\in C$ and $[\varphi\lor\psi]\cap[w]_a=Y$.  Since
    $\varphi\in C$ and $[\varphi]\cap[w]_a=X$, it follows that
    $\lnot\varphi\in C$ and $[\lnot\varphi]\cap[w]_a=X'$.  Hence by
    $X'\notin N_a(w)$, we have $w_a\nvdash_\KB B_a\lnot\varphi$.  For
    each $v\in[w]_a$,

  \end{description}

  % XXX
  \begin{itemize}

  \item For (sc), we must show $X':=[\Gamma]_a-X\notin N_a(\Gamma)$
    and $X\subsetneq Y\subseteq[\Gamma]_a$ together imply $Y\in
    N_a(\Gamma)$.  So suppose we have the antecedent of this
    implication.  It follows that $B_aX'\notin\Gamma$ and therefore
    $\lnot B_aX'\in\Gamma$ by the definition of $L^+$ and maximal
    consistency.  Notice that $\KBeq\vdash X'\leftrightarrow\lnot X$
    because $X'=[\Gamma]_a-X$ and therefore we have by
    Theorem~\ref{theorem:KBgt-derivables}\eqref{derivables:Bimp} (and
    the fact that $\KBeq$ extends $\KBgt$) that $\check
    B_aX\in\Gamma$.  Further, $\check K_a(\lnot X\land Y)\in L^+$, so
    since $X\subsetneq Y\subseteq[\Gamma]_a$, we have $\check
    K_a(\lnot X\land Y)\in\Gamma$ by maximal consistency.  Further,
    since $B_a(X\lor Y)\in L^+$, it follows by (SC) and maximal
    consistency that $B_a(X\lor Y)\in\Gamma$.  Since $K_a(X\lor Y\to
    Y)\in L^+$, we have by $X\subsetneq Y\subseteq[\Gamma]_a$ and
    maximal consistency that $K_a(X\lor Y\to Y)\in\Gamma$.  Hence by
    maximal consistency and (KBM), we have $B_aY\in\Gamma$ and
    therefore $Y\in N_a(\Gamma)$.

  \item For (scott), we must show $(X_i\mathbb{I}_aY_i)_{i=1}^m$,
    $X_1\in N_a(\Gamma)$, and $[\Gamma]_a-X_i\notin N_a(\Gamma)$ for
    all $i\neq 1$ together imply that $Y_j\in N_a(\Gamma)$ for some
    $j$.  Suppose we have the antecedent of this implication.  Then
    $B_aX_1\in\Gamma$.  Further, as in the argument for (sc), our
    assumptions imply $\check B_aX_i\in\Gamma$ for all $i\neq 1$.  By
    our argument for (wl), we have
    $(X_i\mathbb{I}_aY_i)_{i=1}^m\in\Gamma$.  Since
    $\bigvee_{i=1}^mY_i\in L^+$, it follows by (Scott) and maximal
    consistency that $\bigvee_{i=1}^mY_i\in\Gamma$ and therefore by
    maximal consistency that $Y_j\in\Gamma$ for some $j$.  But then
    $Y_j\in N_a(\Gamma)$.  \qedhere
  \end{itemize}
\end{proof}

Since $\KBeq$ is sound and complete with respect to mid-threshold
neighborhood models, we would expect from
Corollary~\ref{corollary:lenzen} that $\KBeq$ is sound and complete
with respect to the probability interpretation for threshold $c=\frac
12$.

\begin{theorem}[Due to \cite{Lenzen1980:gwuw}; $\KBeq$
  Probability Soundness and Completeness]
  $\KBeq$ is sound and complete for threshold $\frac12$ with respect
  to the class of epistemic probability models:
  \[
  \textstyle 
  \forall\phi\in\Lang_\KB:\quad
  \KBeq\vdash\phi
   \quad\Leftrightarrow\quad
  {}\modelsp\phi^{\frac 12} \enspace.
  \]
\end{theorem}
\begin{proof}
  Soundness is by Theorems~\ref{theorem:knowledge} and
  \ref{theorem:belief}.  Completeness is by Theorem~\ref{theorem:KBeq}
  and Corollary~\ref{corollary:lenzen}.
\end{proof}

\section{Conclusion} 
\label{SectionFRW}

\paragraph{Summary}

We have provided a study of the modal logic of certain knowledge and
``betting'' belief (i.e., belief of events greater than a rational
probability $c\geq\frac 12$).  Our study included both a probabilistic
semantics and a neighborhood semantics with a new epistemic twist.  We
formulated Lenzen's proof that $\KBeq$ is the logic of threshold
$c=\frac 12$.  We also proved completeness with respect to the new
neighborhood semantics.  This proof made use of an ingenious
neighborhood-completeness trick due to Segerberg (``logical
finiteness''), which we extended in a way that ensures all
epistemically possible propositions are definable using finitely many
formulas that are all candidates for agent belief in the canonical
model. Moving to a broder take on our results, we believe that our
work provides connections between probabilistic and neighborhood
semantics that may present interesting opportunities for future
cross-disciplinary work.

\paragraph{Open Questions for Future Work}

\begin{enumerate}
\item
The main open question is the following: given a ``high-threshold''
$c\in(\frac 12,1)\cap\Rat$, find the exact extension $\KB^c$ of $\KB$ that is
probabilistically sound and complete for threshold $c$ with respect to
the class of epistemic probability models, in the sense that we would have:
\[
\textstyle 
\forall\phi\in\Lang_\KB:\quad
\KB^c\vdash\phi
\quad\Leftrightarrow\quad
{}\modelsp\phi^c \enspace.
\]
Observing that (SC) and (Scott) are not valid for high-thresholds $c>\frac
12$, we conjecture that what is required are threshold-specific
variants of (SC) and (Scott) that will together guarantee probability
soundness and completeness.  Toward this end, we suggest the following
schemes as a starting point:
\begin{center}
  \renewcommand{\arraystretch}{1.3}
  \begin{tabular}[t]{cl}
    (SC$_0^s$) &
    $\textstyle(\check K_a\phi_0\land
    \bigwedge_{i=1}^s\check B_a\phi_i\land
    \bigwedge_{i\neq j=0}^s K_a(\phi_i\to\lnot \phi_j))\to 
    B_a(\bigvee_{i=0}^s \phi_i)$
    \\
    (SC$_1^s$) &
    $\textstyle(\bigwedge_{i=1}^s\check B_a\phi_i\land
    \bigwedge_{i\neq j=1}^s K_a(\phi_i\to\lnot \phi_j))
    \to B_a(\bigvee_{i=1}^s \phi_i)$
    \\
    (WL) &
    $\textstyle [(\phi_i\mathbb{I}_a\psi_i)_{i=1}^m
    \land \bigwedge_{i=1}^m B_a\phi_i] \to
    \bigvee_{i=1}^m B_a\psi_i$
  \end{tabular}
\end{center}
Observe that (SC) is just (SC$_0^1$).  Further, if we define
$s':=c/(1-c)$ and $s:=\text{ceiling}(s')$, then scheme (SC$_0^s$) is
probabilistically sound if $s=s'$ and scheme (SC$_1^s$) is
probabilistically sound if $s\neq s'$.  The reasoning for this is as
follows: $s'$ tells us the number of $(1-c)$'s that divide $c$.  In
particular, recall from Lemma~\ref{lemma:dual} that the probabilistic
interpretation of $\check B_a\phi$ is that $\phi$ is assigned
probability at least $1-c$.  Therefore, if we have $s$ disjoint
propositions that each have probability at least $1-c$, then the
probability of their disjunction will have probability
$s\cdot(1-c)\geq c$.  This inequality is strict if $s\neq s'$ and is
in fact an equality if $s=s'$.  Therefore, in the case $s\neq s'$,
scheme (SC$_1^s$) is sound: $s$ disjoint propositions each having
probability $1-c$ together sum to a probability exceeding the
threshold $c$.  And in case $s=s'$, scheme (SC$_0^s$) is sound: $s$
disjoint propositions each having probability $1-c$ together sum to a
probability that equals $c$, so adding some additional probability
from another disjoint proposition $\phi_0$ will yield a disjunction whose
probability again exceeds $c$.  In either case, exceeding probability
$c$ is what we equate with belief, so soundness is proved.  We note
that scheme (WL) can be shown to be sound by adapting the proof
Theorem~\ref{theorem:belief}\eqref{item:B-Len}.  The epistemic
neighborhood model versions of (SC$_0^s$), (SC$_1^s$), and (WL) are:
\begin{description}
\item[(sc$_0^s$)] $\forall X_1,\dots,X_s,Y\subseteq[w]_a$:
  if $[w]_a-X_1,\dots,[w]_a-X_s\notin N_a(w)$,
  the $X_i$'s are pairwise disjoint, and $Y\supsetneq\bigcup_{i=1}^sX_i$,
  then $Y\in N_a(w)$.

\item[(sc$_1^s$)] $\forall X_1,\dots,X_s\subseteq[w]_a$: if
  $[w]_a-X_1,\dots,[w]_a-X_s\notin N_a(w)$ and the $X_i$'s are
  pairwise disjoint, then $\bigcup_{i=1}^sX_i\in N_a(w)$.
 
\item[(wl)] $\forall m\in\Int^+,\forall
  X_1,\dots,X_m,Y_1,\dots,Y_m\subseteq[w]_a:$
  \[
  \renewcommand{\arraystretch}{1.3}
  \begin{array}{ll}
    \text{if }
    &
    \begin{array}[t]{l}
      X_1,\dots,X_m\mathbb{I}_aY_1,\dots,Y_m\quad\text{and}
      \\
      \forall i\in\{1,\dots,m\}:
      X_i\in N_a(w) \enspace\text{,}
    \end{array}
    \\
    \text{then }
    &
    \exists j\in\{1,\dots,m\}: Y_j\in N_a(w)\enspace\text{.}
  \end{array}
  \]
\end{description}
If $M$ is an epistemic neighborhood model, then a slight modification
of the proof of property (wl) in Lemma~\ref{lemma:correctness} shows
that $M^c$ satisfies (wl).  We presume that an adaptation of the proof
for the proof of property (sc) in the same lemma will show that $M^c$
satisfies (sc$_0^s$) if $s=s'$ and (sc$_1^s$) if $s\neq s'$.

We remark that (WL) is not threshold-specific, though it is sound for
all high-thresholds $c>\frac 12$.  We suspect that a
threshold-specific variant may be required in order to adapt Lenzen's
proof of $\KBeq$ probability soundness and completeness for threshold
$c=\frac 12$ (Theorem~\ref{theorem:lenzen}).  In particular, using
terminology and notation from that proof: take $c=p/q$ and redefine
set $M_2$ by setting
\[
M_2:= \{((q-p) V(X)- p V(\overline X)\in\mathbb{R}^n\mid
X\subseteq[w]_a,\overline{X}\notin N\}\enspace.
\]
Observe that we have
\[
(q-p) V(X)- p V(\overline X)) =
qV(X)-pV([w]_a)\enspace.
\]
So if we have an appropriate linear functional $\phi$ and
$\overline{X}\notin N$, then
$q\cdot\phi(V(X))-p\cdot\phi(V([w]_a))\geq 0$.  Since
$\phi(V([w]_a))=1$, it follows that $P_{a,w}(X)\geq p/q=c$.  The
argument for the case $\overline{X}\in N$ is similar.  However, we
note that the existence of $\phi$ seems to require a
threshold-specific version of (Scott) and it is not clear how this might
come about so as to follow the logic of the proof of
Theorem~\ref{theorem:lenzen}.  Perhaps some variant of (WL) that takes
into account the specific values of $p$ and $q$ is required.

\item Another open question is the exact relationship between
  Segerberg's comparitive operator $\phi\geq\psi$ (``$\phi$ is at
  least as probable as $\psi$'') \cite{Segerberg1971:qpiams} and
  betting belief.  $B_a\phi$ is equivalent to $\phi>\lnot\phi$, where
  the strict inequality $>$ is defined in the natural way.  However,
  it is not clear how the logics of these operators are related.

\item Yet another direction is the extension of our work to Bayesian
  updating.  Given a pointed epistemic probability model $(\M,w)$
  satisfying $\phi$, let
\[
\M[\phi]=(W[\phi],R[\phi],V[\phi],P[\phi])
\]
be defined by
\begin{eqnarray*}
  W[\phi] &:=& \semp{\phi}^\M \\
  R[\phi]_a &:=& R_a \cap (W[\phi]\times W[\phi]) \\
  V[\phi](w) &:=& V(w) \text{ for } w\in W[\phi] \\
  P[\phi]_a(w) &:=&
    \frac{ P_a(w) }{ P_a(\semp{\phi}^\M) }
\end{eqnarray*}
It is not difficult to see that $\M[\phi]$ is an epistemic probability model and
\[
P[\phi]_a(X) =
\frac{ P_a(X\cap\semp{\phi}^\M) }{ P_a(\semp{\phi}^\M) } = 
P[\phi]_a(X|\semp{\phi}^\M)\enspace,
\]
where the value on the right is the probability of $X$ conditional on
$\semp{\phi}^\M$.  It would be interesting to investigate the analog
of this operation in epistemic neighborhood models.  The operation may
also have a close relationship with the study of updates in
Probabilistic Dynamic Epistemic Logic
\cite{BenGerKoo09:SL,BalSme08:Synt}.
\end{enumerate}

\paragraph{Acknowledgements} 

Thanks to Alexandru Baltag, Jim Delgrande, Andreas Herzig, and Sonja
Smets for helpful comments and pointers to the literature.

\bibliographystyle{alpha}
\bibliography{vER-BEPL}  

\end{document} 
