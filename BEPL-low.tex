\listfiles
\documentclass[12pt]{article}

%%
%% page setup and comments:
%% - comment out the \def\comments{1} if you do not want comments
%% - leave the \def\comments{1} uncommented if you do want comments
%%

\def\comments{1}

%% leave the next few lines below alone

\ifx\comments\undefined
  % then do normal margins and ignore comments
  \usepackage[a4paper,margin=4cm,includefoot]{geometry}
  \newcommand{\XXXcomment}[1]{}
\else
  % then do big marginpar margins and insert comments
  \usepackage[a4paper,margin=.5in,right=2.5in,marginpar=2in,includefoot]{geometry}
  \newcommand{\XXXcomment}[1]{\marginpar{\color{blue}{\footnotesize #1}}}
\fi

%%
%% packages and package-specific settings
%%

\usepackage{latexsym,amssymb,amsmath,amscd,amsthm,alltt,stmaryrd,graphicx,mathrsfs}

\usepackage{times}

\usepackage{float}
\floatstyle{boxed}
\restylefloat{figure}
\restylefloat{table}

%%
%% tikz stuff, including custom environment
%%

\usepackage{tikz}
\usetikzlibrary{trees,arrows,positioning,patterns,automata,shapes,decorations,decorations.pathmorphing}

\newcommand{\mystretch}{\renewcommand{\arraystretch}{1.5}}
\newcommand{\normalstretch}{\renewcommand{\arraystretch}{1}}

\newlength{\mysep}
\setlength{\mysep}{2.7cm}
\newlength{\mysepFarther}
\setlength{\mysepFarther}{3cm}
\newlength{\mysepFarthest}
\setlength{\mysepFarthest}{4cm}

\newenvironment{mytikz}[1][0em]{
\begin{tikzpicture}[>=latex,auto,node distance=\mysep,
  baseline={([yshift=#1]current bounding box.east)}]

  \normalstretch{}

  \tikzstyle{w}=[draw,circle,thick,minimum size=3.8em]

  \tikzstyle{e}=[draw,minimum size=2.5em,node distance=3cm]
  
  \tikzstyle{every edge}=[draw,thick,font=\footnotesize]
  
  \tikzstyle{every label}=[font=\footnotesize]
  
  \tikzstyle{ev}=[anchor=west,node distance=2em]

  \tikzstyle{bigger}=[minimum size=4em]

  \tikzstyle{farther}=[node distance=\mysepFarther]

  \tikzstyle{farthest}=[node distance=\mysepFarthest]

  \tikzstyle{l}=[node distance=3.5em]
}{\mystretch{}\end{tikzpicture}}

%%
%% theorem environments
%%

\theoremstyle{definition}
\newtheorem{theorem}{Theorem}[section]
\newtheorem{proposition}[theorem]{Proposition}
\newtheorem{definition}[theorem]{Definition}
\newtheorem{lemma}[theorem]{Lemma}
\newtheorem{conjecture}[theorem]{Conjecture}
\newtheorem{example}[theorem]{Example}
\newtheorem{question}[theorem]{Question}
\newtheorem{exercise}[theorem]{Exercise}
\newtheorem{remark}[theorem]{Remark}

%%
%% custom commands
%%

\newcommand{\Nat}{\mathbb{N}}  % natural numbers
\newcommand{\Rat}{\mathbb{Q}}  % rational numbers
\newcommand{\Ree}{\mathbb{R}}  % real numbers
\newcommand{\Int}{\mathbb{Z}}  % integers

\newcommand{\pow}{{\cal P}}    % powerset

\newcommand{\M}{{\cal M}}      % caligraphic model M
\newcommand{\N}{{\cal N}}      % caligraphic model N

\newcommand{\Prop}{{\bf P}}    % propositional letters

\newcommand{\Lang}{{\cal L}}   % language

\newcommand{\conv}{\check{\ }} % dual modality

\newcommand{\KB}{{\mathsf{KB}}}                     % base theory
\newcommand{\KBlt}{{\mathsf{KB}^{\mathsf{<.5}}}}    % theory for c<.5
\newcommand{\KBeq}{{\mathsf{KB}^{\mathsf{=.5}}}}    % theory for c=.5
\newcommand{\KBgt}{{\mathsf{KB}^{\mathsf{>.5}}}}    % theory for c>.5

\newcommand{\Clt}{{\mathcal{C}^{\mathsf{<.5}}}}    % neigh. models for c<.5
\newcommand{\Ceq}{{\mathcal{C}^{\mathsf{=.5}}}}    % for c=.5
\newcommand{\Cgt}{{\mathcal{C}^{\mathsf{>.5}}}}    % for c>.5

\newcommand{\Rel}[1]{\stackrel{#1}{\Longrightarrow}}         % #1 over \Longrightarrow

\newcommand{\sem}[1]{\llbracket{#1}\rrbracket}               % [[ #1 ]]

\newcommand{\modelsn}{\models_{\mathsf{n}}}                  % \models_n
\newcommand{\semn}[1]{\llbracket{#1}\rrbracket_{\mathsf{n}}} % [[ #1 ]]_n

\newcommand{\modelsp}{\models_{\mathsf{p}}}                  % \models_p
\newcommand{\semp}[1]{\llbracket{#1}\rrbracket_{\mathsf{p}}} % [[ #1 ]]_p

%%
%% author, title, &c
%%

\newcommand{\ourtitle}{Belief as Willingness to Bet}

\newcommand{\jan}{Jan van Eijck}

\newcommand{\janAffiliation}{CWI \&\ ILLC, Amsterdam}

\newcommand{\bryan}{Bryan Renne}

\newcommand{\bryanAffiliation}{ILLC, University of Amsterdam}

\newcommand{\bryanFunding}{Funded by an Innovational Research
  Incentives Scheme Veni grant from the Netherlands Organisation for
  Scientific Research (NWO).}

\title{\ourtitle{}} 

\author{\jan{}\\{}{\small\janAffiliation{}} \and
  \bryan{}\footnote{\bryanFunding{}}\\{}{\small\bryanAffiliation{}}}

\usepackage[pdftex,
            bookmarks=true,
            bookmarksnumbered=true,
            pdfborder={0 0 0},
            plainpages=false,
            pdfpagelabels]{hyperref}

\hypersetup{ 
  pdfauthor={\jan{}, \bryan{}}, 
  pdftitle={\ourtitle{}}
}

%%
%% beginning of paper
%%

\begin{document}

\maketitle

\begin{abstract}
  We investigate the logic of the operator ``I am willing to bet on
  $\phi$'' by relating a semantics for this in terms of probability
  models to a more abstract semantics in terms of epistemic
  neighborhood models.  This relates the perspective on rational
  decision making of the probabilist with that of the logician, and
  can perhaps be viewed as a step toward reconciliation.  We show
  completeness of various calculi, with respect to a class of
  epistemic neighborhood models with respect to a strictly smaller
  class of committed epistemic neighborhood models, and with respect a
  class of probability models that is in turn properly contained in
  this class. This demonstrates at the same time the power and the
  weakness of the logical point of view: logic is able to talk about
  larger classes of models, but at the price of being able to say
  less.
\end{abstract}

\section{Introduction}

In thinking about rational decision making there are roughly two
schools of thought. The probabilist school holds that rational
decision making should have its foundation in probability theory. See
\cite{koerner2008naive} for an eloquent example. The AI school holds
that the reasoning that goes on in rational decision making is a mix
of doxastic and epistemic logic (perhaps with a dash of non-monotonic
logic added). See, e.g., \cite{KyburgTeng2012:tlorkr}.  The present
paper can be viewed as a fresh attempt to reconcile these two views.

Section \ref{Section:EPL} introduces epistemic probability logic and
probability models as a frame of reference. Section \ref{Section:CB}
defines modal operators as abbreviations in this language.  Section
\ref{Section:ENM} introduces epistemic minimal models for knowledge
and belief. Section \ref{Section:BeliefBet} formulates and proves a
Betting and Certainty theorem, relating interpretation in epistemic
minimal models to interpretation in epistemic probability models.
Section \ref{Section:TreeCalculi} introduces three classes of belief
models, relates the three classes, and gives a complete calculus for
each of them.  This shows that the epistemic minimal model semantics
can be viewed as an abstract version of the epistemic probability
semantics.  Section \ref{SectionFRW} concludes with plans for future
work and connections to the literature.

\section{Probability Models}
\label{Section:EPL} 

Following \cite{Eijck2013:lap}, we define probability models as finite
multi-agent $\mathsf{S5}$ Kripke models with a probabilistic component
given by a so-called ``lottery.''

\begin{definition}
  Given a finite nonempty set $W$ of ``worlds,'' a \emph{$W$-lottery}
  is a function $l:W\to\Rat^+$ mapping each world $w\in W$ to a
  positive rational number $l(w)$.  We let $\mathsf{Lot}(W)$ denote
  the set of $W$-lotteries.  A $W$-lottery will be called a
  ``lottery'' if the set $W$ ought to be clear from context.
\end{definition}

\begin{definition}
  We fix a finite nonempty set $A$ of ``agents'' and a set $\Prop$ of
  propositional letters.  An \emph{epistemic probability model} $\M$
  is a tuple $(W,V,R,n,l,E)$ satisfying the following.
  \begin{itemize} 
  \item $(W,R,V)$ is a finite multi-agent $\mathsf{S5}$ Kripke model:
    \begin{itemize}
    \item $W$ is a finite nonempty set of ``worlds.''
      
    \item $R:A\to\pow(W\times W)$ assigns an equivalence relation $R_a$
      on $W$ to each agent $a\in A$.

    \item $V:W\to\pow(\Prop)$ assigns a set $V(w)$ of propositional
      letters to each world $w\in W$.
    \end{itemize}

  \item $n\in\Int^+$ is a positive integer.  We let
    $\bar{n}=\{0,\dots,n-1\}$ denote the set of nonnegative integers
    less than $n$.

  \item $l:\bar{n}\to\mathsf{Lot}(W)$ maps each index $i\in\bar{n}$ to
    a $W$-lottery $l_i:W\to\Rat^+$.

  \item $E:A\to\pow(\bar{n}\times\bar{n})$ assigns an equivalence
    relation $E_a$ on $\bar{n}$ to each agent $a\in A$.
  \end{itemize}
  A \emph{pointed epistemic probability model} is a triple $(\M,w,i)$
  consisting of an epistemic probability model $\M=(W,V,R,n,l,E)$, a
  world $w\in W$, and an index $i\in\bar{n}$.
\end{definition}

Intuitively, a lottery $l$ assigns a positive rational-number
``weight'' $l(w)$ to each world $w$.  An epistemic probability model
$\M=(W,V,R,n,l,E)$ consists of $n$ lotteries, indexed by the set
$\bar{n}=\{0,\dots,n-1\}$.  Agent $a$'s uncertainty as to which
lottery is the actual lottery is given by the equivalence relation
$E_a$.  Agent $a$'s uncertainty as to which world is the actual world
is given by the equivalence relation $R_a$.  We construe an
\emph{event} as a set of world-index pairs.  If $w$ is the actual
world and $i$ is the index of the actual lottery, then the probability
agent $a$ assigns to an event $S$ is given by the ratio
\begin{equation}
  P^{\M}_{a,w,i}(S) = \frac{\sum \{l_j(u) \mid wR_au \text{ and } iE_aj
    \text{ and } (u,j)\in S \}} {\sum \{l_j(u) \mid wR_au \text{ and }
    iE_aj\}}\enspace.
  \label{eq:probability}
\end{equation}
In words: the probability agent $a$ assigns to event $S$ at
world-index $(w,i)$ is the fraction of total weight of
$a$-indistinguishable world-index pairs that come from event $S$.

Since there are finitely many worlds and finitely many lotteries, each
sum in \eqref{eq:probability} is finite.  Further, since lottery
weights are positive and each of $R_a$ and $E_a$ is reflexive, the
denominator in \eqref{eq:probability} is nonzero.  It follows that
$P^{\M}_{w,a,i}(S)$ is well-defined.  We will drop the superscript
$\M$ in this notation when doing so ought not cause confusion.  We
will sometimes abuse notation and write $P^{\M}_{w,a,i}(X)$ for an
object $X$ that is not an event (i.e., $X$ is not a set of world-index
pairs); in so doing, what is meant is $P^{\M}_{w,a,i}(S_X)$ for the
set $S_X$ made up of all world-index pairs at which $X$ holds.  When
we abuse notation in this way, the exact membership of $S_X$ will
always be clear from context.

\begin{example}[Horse Racing] 
  \label{ExampleHorseRacing}
  Three horses compete in a race.  For each $i\in\{1,2,3\}$, horse
  $h_i$ wins the race in world $w_i$.  Neither agent can distinguish
  between these three possibilities. The chances for the horses to win
  are either $3{:}2{:}1$ or equal. Agent $a$ can distinguish between
  these two possibilities, but agent $b$ cannot.  We picture this
  scenario as follows.
  \begin{center}
    \begin{mytikz}
      %%
      %% nodes
      %%
      \node[w,label={left:$w_1$}] (w1) {$h_1$};

      \node[w,below of=w1,label={left:$w_2$}] (w2) {$h_2$};

      \node[w,below of=w2,label={left:$w_3$}] (w3) {$h_3$};

      \node[right of=w1,farthest,yshift=-.5\mysep,label={right:$l_0$}] (l0)
      {$\{w_1:3,w_2:2,w_3:1\}$};

      \node[below of=l0,label={right:$l_1$}] (l1)
      {$\{w_1:2,w_2:2,w_3:2\}$};

      \node[right of=w3,yshift=-2em] 
      {$\M_{\ref{ExampleHorseRacing}}$};

      %%
      %% edges
      %%
      \path (w1) edge[<->] node{$a,b$} (w2);

      \path (w2) edge[<->] node{$a,b$} (w3);

      \path (l0) edge[<->] node{$b$} (l1);
    \end{mytikz}
  \end{center}
  When we picture epistemic probability models, the arrows of each
  individual agent are to be closed under reflexivity and
  transitivity.  With this convention in place, it is not difficult to
  verify the following.
  \begin{enumerate}
  \item $P_{a,w_1,0} (l_0) = 1$.

    At $(w_1,0)$, agent $a$ is certain that the winning chances are
    $3{:}2{:}1$.

  \item $P_{a,w_1,0} (h_1) = \frac{1}{2}$.

    At $(w_1,0)$, agent $a$ assigns probability $\frac 12$ to horse
    $1$ winning the race.

  \item $P_{b,w_1,0} (l_0) = \frac{1}{2}$.

    At $(w_1,0)$, agent $b$ assigns probability $\frac 12$ to the
    winning chances being $3{:}2{:}1$.

  \item $P_{b,w_1,0} (h_1) = \frac{5}{12}$.

    At $(w_1,0)$, agent $b$ assigns probability $\frac 5{12}$ to horse
    $1$ winning the race.
  \end{enumerate}
\end{example} 

In our models for epistemic probability logic, lottery values are
always positive. Therefore, in order to represent a situation in which
agent $a$ is certain that horse $3$ can never win, we simply make the
$h_3$-worlds inaccessible via $R_a$.

\begin{example}[Horse Racing, Once More]
  \label{ExampleHorseRacing2}
  Consider the following variation of our horse racing example.
  \begin{center}
    \begin{mytikz}
      %%
      %% nodes
      %%
      \node[w,label={left:$w_1$}] (w1) {$h_1$};

      \node[w,below of=w1,label={left:$w_2$}] (w2) {$h_2$};

      \node[w,below of=w2,label={left:$w_3$}] (w3) {$h_3$};

      \node[right of=w1,farthest,yshift=-.5\mysep,label={right:$l_0$}] (l0)
      {$\{w_1:3,w_2:2,w_3:1\}$};

      \node[below of=l0,label={right:$l_1$}] (l1)
      {$\{w_1:2,w_2:2,w_3:2\}$};

      \node[right of=w3,yshift=-2em] 
      {$\M_{\ref{ExampleHorseRacing2}}$};

      %%
      %% edges
      %%
      \path (w1) edge[<->] node{$a,b$} (w2);

      \path (w2) edge[<->] node{$b$} (w3);

      \path (l0) edge[<->] node{$b$} (l1);
    \end{mytikz}
  \end{center}
  At world $w_1$ in this picture, there is no $a$-accessible world at
  which horse $3$ wins.  Therefore, at world $w_1$, agent $a$ assigns
  probability $0$ to the event that horse $3$ wins: we have
  $P_{a,w_1,i}(h_3)=0$ for each $i\in\bar{2}$.
\end{example}

We define a language $\Lang$ for reasoning about multi-agent
probability logic over lotteries.

\XXXcomment{Changed language for simpler presentation.  I recall you
  mentioned this hurts the computational properties, but I wonder if
  it might be better to have this simpler presentation for this paper.
  (Later we could mention a variant of the language that has better
  computational properties, perhaps when these are
  studied?)\hfill{}--br}

\begin{definition}
  The language $\Lang$ of \emph{multi-agent probability logic} is
  defined by the following grammar.
  \begin{eqnarray*}
    \phi & ::= & 
    \top \mid p \mid \neg\phi \mid \phi\land\phi \mid
    l_j \mid @_j \phi \mid t\geq q
    \\
    t   & ::= & q \mid q \cdot P_a(\phi) \mid t+t
    \\
    &&
    \text{\footnotesize 
      $p\in\Prop$,
      $j\in\Nat$,
      $q\in\Rat$,
      $a\in A$
    }
  \end{eqnarray*}
  We adopt the usual abbreviations for Boolean connectives.  We define
  the relational symbols $\leq$, $>$, $<$, and $=$ in terms of $\geq$
  as usual.  For example, $t=s$ abbreviates $(t\geq s)\land(s\geq t)$.
\end{definition}

\begin{definition} 
  Let $\M=(W,R,V,n,l,E)$ be an epistemic probability model.  We define
  a binary truth relation $\modelsp$ between a pointed epistemic
  probability model $(\M,w)$ and $\Lang$-formulas as follows.
  \[
  \renewcommand{\arraystretch}{1.3}
  \begin{array}{lcl}
    \M,w,i\modelsp\top 
    \\
    \M,w,i\modelsp p & \text{iff} & 
    p \in V(w) 
    \\
    \M,w,i\modelsp\neg\phi & \text{iff} &
    \M,w,i\not\modelsp\phi
    \\
    \M,w,i\modelsp\phi\land\psi & \text{iff} &
    \M,w,i\modelsp\phi \text{ and } \M,w,i\modelsp\psi
    \\
    \M,w,i\modelsp l_j & \text{iff} & 
    i = j \;(\text{mod}\; n)
    \\
    \M,w,i\modelsp @_j \phi & \text{iff} &
    \M, w,(j \text{ mod } n)\modelsp \phi
    \\
    \M,w,i\modelsp t\geq s & \text{iff} &
    \sem{t}^{\M}_{w,i}\geq\sem{s}^{\M}_{w,i}
  \end{array}
  \]
  \[
  \renewcommand{\arraystretch}{1.3}
  \begin{array}{rcl}
    \semp{\phi}^{\M} & := &
    \{ (u,j) \in W\times\bar{n} \mid \M,u,j\modelsp\phi \}
    \\[1em]
    \sem{q}^{\M}_{w,i} & := & q
    \\
    \sem{q\cdot P_a(\phi)}^{\M}_{w,i} & := & 
    q\cdot P^{\M}_{a,w,i}(\semp{\phi}^{\M}) 
    \text{ with $P^{\M}_{a,w,i}$ given by \eqref{eq:probability}}
    \\
    \sem{t+s}^{\M}_{w,i} & := &
    \sem{t}^{\M}_{w,i} + \sem{s}^{\M}_{w,i}
  \end{array}
  \]
  Validity of $\phi\in\Lang$ in epistemic probability model $\M$,
  written $\M\modelsp\phi$, means that $\M,w\modelsp\phi$ for each
  world $w\in W$.  Validity of $\phi\in\Lang$, written $\modelsp\phi$,
  means that $\M\modelsp\varphi$ for each epistemic probability model
  $\M$.
\end{definition} 

$P_{a,w,i} (\phi)$ is the probability agent $a$ assigns to $\phi$ at
world $w$ under lottery $l_i$. According to \eqref{eq:probability},
this value is normalized for the set of all world-index pairs that
agent $a$ cannot distinguish from $(w,i)$.  Note that $P_{a,w,i} (l_i)
= 1$ if and only if agent $a$ does not confuse $l_i$ with any other
lottery.

\begin{example}
  Referring to the epistemic probability models
  $\M_{\ref{ExampleHorseRacing}}$ of Example~\ref{ExampleHorseRacing}
  and $\M_{\ref{ExampleHorseRacing2}}$ of
  Example~\ref{ExampleHorseRacing2}, the following are easily
  verified.
  \begin{enumerate}
  \item $\M_{\ref{ExampleHorseRacing}},w_1,0\modelsp P_a(l_0)= 1$.
    
    At $(w_1,0)$ in $\M_{\ref{ExampleHorseRacing}}$, agent $a$ is
    certain that $l_0$ is the actual lottery.
    
  \item $\M_{\ref{ExampleHorseRacing}},w_1,0\modelsp P_a(h_1) =
    \frac{1}{2}$.
    
    At $(w_1,0)$ in $\M_{\ref{ExampleHorseRacing}}$, agent $a$ assigns
    probability $\frac 12$ to $h_1$.
    
  \item $\M_{\ref{ExampleHorseRacing}},w_1,0\modelsp P_b(l_0) =
    \frac{1}{2}$.
    
    At $(w_1,0)$ in $\M_{\ref{ExampleHorseRacing}}$, agent $b$ assigns
    probability $\frac 12$ to the lottery being $l_0$.
    
  \item $\M_{\ref{ExampleHorseRacing}},w_1,0\modelsp P_b(h_1) = \frac{5}{12}$.

    At $(w_1,0)$ in $\M_{\ref{ExampleHorseRacing}}$, agent $b$ assigns
    probability $\frac 5{12}$ to $h_1$.

  \item $\M_{\ref{ExampleHorseRacing2}},w_1,i\modelsp P_a(h_3)=0$ for
    each $i\in\bar 2$.

    At world $w_1$ in $\M_{\ref{ExampleHorseRacing2}}$, agent $a$
    assigns probability $0$ to $h_3$.
  \end{enumerate}
\end{example}


\section{Certainty and Belief} 
\label{Section:CB} 

\cite{Eijck2013:lap} formulates and proves a ``certainty theorem''
relating certainty in epistemic probability models to knowledge in a
version of these models in which the lottery information is removed.
This motivates the following definition.

\begin{definition}[Knowledge as certainty]
  We adopt the following abbreviations.
  \begin{itemize}
  \item $K_a\phi$ abbreviates $P_a(\phi)=1$. 

    We read $K_a\phi$ as ``agent $a$ knows $\phi$.''

  \item $\check K_a\phi$ abbreviates $\lnot K_a\lnot\phi$.

    We read $\check K_a\phi$ as ``$\phi$ is consistent with agent
    $a$'s knowledge.''
  \end{itemize}
\end{definition}

\begin{proposition}[Properties of knowledge as certainty;
  \cite{Eijck2013:lap}]
  \label{prop:knowledge}
  $K_a$ is an $\mathsf{S5}$ modal operator:
  \begin{enumerate}
  \item $\modelsp \phi$ for each $\Lang$-instance $\phi$ of a scheme
    of classical propositional logic.

    Axioms of classical propositional logic are valid.

  \item $\modelsp K_a(\varphi\to\psi)\to(K_a\varphi\to K_a\psi)$
    
    Knowledge is closed under logical consequence.

  \item $\modelsp K_a\varphi\to \varphi$

    Knowledge is veridical.
    
  \item $\modelsp K_a\varphi\to K_aK_a\varphi$

    Knowledge is positive introspective:  it is known what is known.
    
  \item $\modelsp \lnot K_a\varphi\to K_a\lnot K_a\varphi$

    Knowledge is negative introspective: it is known what is not
    known.
    
  \item $\modelsp\varphi$ implies $\modelsp K_a\varphi$

    All validities are known.

  \item $\modelsp\varphi\to\psi$ and $\modelsp\varphi$ together imply
    $\modelsp\psi$.

    Validities are closed under the rule of Modus Ponens.
  \end{enumerate}
\end{proposition}

We define belief in a proposition $\phi$ as willingness to take bets
on $\phi$ with the odds being better than some rational number
$c\in[0,1]\cap\Rat$.  This leads to a number of degrees of belief, one
for each threshold $c$.

\begin{definition}[Belief as willingness to bet]
  Fix a threshold $c\in(0,1)\cap\Rat$.\footnote{We do not allow the
    thresholds $c=0$ or $c=1$ because many forthcoming results would
    require us to continually add the proviso that the threshold is
    neither $0$ nor $1$, and this would soon become tiresome.
    Furthermore, the two natural definitions of threshold-$1$ belief
    $B^1_a\phi$ (as $P_a(\phi)>1$ or as $P_a(\phi)=1$) are already
    expressible in the language (respectively as $\bot$ and as
    $K_a\phi$).  Similarly, the two natural definitions of
    threshold-$0$ belief $B^0_a\phi$ (as $P_a(\phi)>0$ or as
    $P_a(\phi)=0$) are also expressible (respectively as $\check
    K_a\phi$ and as $K_a\lnot\phi$).  This being said, we do not allow
    the thresholds $c=0$ or $c=1$ anywhere else in this paper.}
  \begin{itemize}
  \item $B_a^c\phi$ abbreviates $P_a(\phi)>c$.

    We read $B_a^c\phi$ as ``agent $a$ believes $\phi$ with threshold
    $c$.''

  \item $\check B_a^c\phi$ abbreviates $\lnot B_a^c\lnot\phi$.

    We read $\check B_a\phi$ as ``$\phi$ is consistent with agent
    $a$'s threshold-$c$ beliefs.''
  \end{itemize}
  If the threshold $c$ is omitted (either in the notations $B_a^c\phi$
  and $\check B_a^c\phi$ or in the informal readings of these
  notations), it is assumed that $c=\frac 12$.
\end{definition}

This notion of belief comes from subjective probability
\cite{Jeffrey2004:sptrt}.  In particular, fix a threshold
$c=p/q\in(0,1)\cap\Rat$.  Suppose that agent $a$ believes $\phi$ with
threshold $c=p/q$; that is, $P_a(\phi)>p/q$.  If the agent wagers $p$
dollars for a chance to win $q-p$ dollars on a bet that $\phi$ is
true, then she expects to win
\[
(q-p)\cdot P_a(\phi) - p\cdot(1-P_a(\phi)) = q\cdot P_a(\phi) - p
\]
dollars on this bet.  This is a positive number of dollars if and only
if $q\cdot P_a(\phi)>p$.  But notice that the latter is guaranteed by
the assumption $P_a(\phi)>p/q$.  Therefore, it is rational for agent
$a$ to take this bet.  Said in the parlance of the subjective
probability literature, ``If agent $a$ stakes $p$ to win $q-p$ in a
bet on $\phi$, then her winning expectation is positive in case she
believes $\phi$ with threshold $c$.''  Or in a short motto: ``Belief
is willingness to bet.''

In this paper, we will ignore ``low-threshold'' beliefs; that is, we
ignore thresholds $c\in(0,\frac 12)$.  Low-threshold beliefs are
studied in a separate paper \cite{EijRen13:Manu}.  Our present
interest will be in ``high-threshold'' beliefs: those with thresholds
$c\in[\frac 12,1)$.

The following lemma provides a useful characterization of the dual
$\check B^c_a\phi$.

\begin{lemma}
  \label{lemma:dual}
  Let $\M=(W,R,V,n,l,E)$ be an epistemic probability model.
  \begin{enumerate}
  \item \label{item:dual} $\M,w,i\modelsp\check B_a^c\phi$ iff
    $\M,w,i\modelsp P_a(\phi)\geq c'$, where $c':=1-c$.

  \item \label{item:dual-half} $\M,w,i\modelsp\check B_a^{1/2}\phi$ iff
    $\M,w,i\modelsp P_a(\phi)\geq \frac 12$.

  \item \label{item:dual-low} For $c\in(0,\frac 12)\cap\Rat$:
    \[
    \M,w,i\modelsp\check B_a^c\phi \text{ implies } \M,w,i\modelsp
    P_a(\phi)>c\enspace\text{.}
    \]
  \end{enumerate}
\end{lemma}
\begin{proof}
  For Item~\ref{item:dual}, we have the following:
  \[
  \renewcommand{\arraystretch}{1.3}
  \begin{array}{lll}
    &
    \M,w,i\modelsp\check B^c_a\phi
    \\
    \text{iff} &
    \M,w,i\modelsp\lnot B^c_a\lnot\phi 
    & \text{by definition of $\check B^c_a\phi$}
    \\
    \text{iff} &
    P^\M_{a,w,i}(\semp{\lnot\phi}^\M)\not>c
    & \text{by definition of $B^c_a\phi$ and $\models$}
    \\
    \text{iff} &
    P^\M_{a,w,i}(\semp{\lnot\phi}^\M)\leq c
    & \text{since $\Rat$ is totally ordered}
    \\
    \text{iff} &
    P^\M_{a,w,i}(\semp{\phi}^\M)\geq 1-c
    & \text{since $\semp{\lnot\phi}^\M=W-\semp{\phi}^\M$}
    \\
    \text{iff} &
    \M,w,i\modelsp P_a(\phi)\geq c'
    & \text{definition of $\models$ and $c'$}
  \end{array}
  \]
  For Item~\ref{item:dual-half}, we observe that $1-\frac 12=\frac 12$
  and apply Item~\ref{item:dual}.  For Item~\ref{item:dual-low}, we
  observe that $c\in(0,\frac 12)$ implies $c'=1-c>c$, so the result
  follows by Item~\ref{item:dual}.  Note that the converse of
  Item~\ref{item:dual-low} does not follow: since $c\in(0,\frac 12)$
  implies $c'=1-c>c$, it is not the case that $\M,w,i\modelsp
  P_a(\phi)>c$ implies $\M,w,i\modelsp P_a(\phi)\geq c'$.
\end{proof}

We now consider a simple example.

\begin{example}[Horse Racing, Single-Agent Version]
  \label{ExampleHorseRacing3}
  In this variation, all horses have equal chances of winning and
  agent $a$ knows this.
  \begin{center}
    \begin{mytikz}
      %%
      %% nodes
      %%
      \node[w,label={left:$w_1$}] (w1) {$h_1$};

      \node[w,below of=w1,label={left:$w_2$}] (w2) {$h_2$};

      \node[w,below of=w2,label={left:$w_3$}] (w3) {$h_3$};

      \node[right of=w1,farthest,yshift=-.5\mysep,label={right:$l_0$}] (l0)
      {$\{w_1:1,w_2:1,w_3:1\}$};

      \node[right of=w3,yshift=-2em] 
      {$\M_{\ref{ExampleHorseRacing3}}$};

      %%
      %% edges
      %%
      \path (w1) edge[<->] node{$a$} (w2);

      \path (w2) edge[<->] node{$a$} (w3);
    \end{mytikz}
  \end{center}
  The following are readily verified.
  \begin{enumerate}
  \item $\M_{\ref{ExampleHorseRacing3}}\modelsp B_a(h_1\lor h_2\lor
    h_3)$.

    Agent $a$ believes the winning horse is among the three.

    (Agent $a$ is willing to bet that the winning horse is among the three.)

  \item $\M_{\ref{ExampleHorseRacing3}}\modelsp B_a(h_1\lor h_2)\land
    B_a(h_1\lor h_3)\land B_a(h_2\lor h_3)$.

    Agent $a$ believes the winning horse is among any two.

    (Agent $a$ is willing to bet that the winning horse is among any two.)

  \item \label{item:conjuncts} $\M_{\ref{ExampleHorseRacing3}}\modelsp
    B_a\lnot h_1\land B_a\lnot h_2\land B_a\lnot h_3$.

    Agent $a$ believes the winning horse is not any particular one.

    (Agent $a$ is willing to bet that the winning horse is not any
    particular one.)

  \item \label{item:conjunction}
    $\M_{\ref{ExampleHorseRacing3}}\not\modelsp B_a(\lnot h_1\land
    \lnot h_2)$.

    Agent $a$ does not believe that both horses $1$ and $2$ do not
    win.

    (Agent $a$ is not willing to bet that both horses $1$ and $2$ do
    not win.)
  \end{enumerate}
\end{example} 

It follows from Items~\ref{item:conjuncts} and \ref{item:conjunction}
of Example~\ref{ExampleHorseRacing3} that the present notion of belief
is not closed under conjunction.  This is known as the ``Lottery
Paradox'' \cite{Kyburg1961:patlorb}.\footnote{The usual formulation of
  the Lottery Paradox: it is paradoxical for an agent to believe that
  one of $n$ lottery tickets will be a winner (i.e., ``some ticket is
  a winner'') without believing of any particular ticket that it is
  the winner (i.e., ``for each $i\in\{1,\dots,n\}$, ticket $i$ is not
  a winner'').}  However, there is no reason in general that it is
paradoxical to assign a conjunction $\phi \land \psi$ a lower
probability than either of its conjunctions.  Indeed, if $\phi$ and
$\psi$ are independent, then the probability of their conjunction
equals the product of their probabilities, so unless one of $\phi$ or
$\psi$ is certain or impossible, the probability of $\phi \land \psi$
will be less than the probability of $\phi$ and less than the
probability of $\psi$.

We set aside philosophical arguments for or against closure of belief
under conjunction and instead turn our attention to the study of the
properties of the present notion of belief.  One of these is a
complicated but useful property due to Lenzen \cite{Lenzen2003:kbasp}
that makes use of notation due to Segerberg
\cite{Segerberg1971:qpiams}.  Another is a similarly complicated
property related to the Lenzen property.

\begin{definition}[Segerberg notation; \cite{Segerberg1971:qpiams}]
  \label{definition:segerberg-notation}
  Fix a positive integer $m\in\Int^+$ and formulas
  $\phi_1,\dots,\phi_m$ and $\psi_1,\dots,\psi_m$.  The expression
  \begin{equation}
    (\phi_1,\dots,\phi_m\mathbb{E}_a\psi_1,\dots,\psi_m)
    \label{eq:segerberg}
  \end{equation}
  abbreviates the formula
  \[
  K_a(C_0\lor C_1\lor C_2 \lor \cdots \lor C_m)\enspace,
  \]
  where $C_i$ is the disjunction of all conjunctions
  \[
  d_1\phi_1\land\cdots\land d_m\phi_m \land
  e_1\psi_1\land\cdots\land e_m\psi_m
  \]
  satisfying the property that exactly $i$ of the $d_i$'s are the
  empty string, exactly $i$ of the $e_i$'s are the empty string, and
  the rest of the $d_i$'s and $e_i$'s are the negation sign $\lnot$.
  We may write $\overline\phi\mathbb{E}\overline\psi$ as an
  abbreviation for \eqref{eq:segerberg} when doing so ought not cause
  confusion.
\end{definition}

The formula $\overline\phi\mathbb{E}_q\overline\psi$ says that agent
$a$ knows that the number of true $\phi_i$'s is equal to the number of
true $\psi_i$'s.  Put another way,
$\overline\phi\mathbb{E}_q\overline\psi$ is true if and only if every
epistemically accessible world satisfies just as many $\phi_i$'s as
$\psi_i$'s.

\begin{definition}[Lenzen schemes; \cite{Lenzen2003:kbasp}]
  \label{definition:lenzen-schemes}
  We define the following schemes:
  \[
  \begin{array}{cl}
    \textstyle [(\phi_1,\dots,\phi_m\mathbb{E}_a\psi_1,\dots,\psi_m)
    \land B_a^c\phi_1 \land \bigwedge_{i=2}^m \check B_a^c\phi_i] \to
    \bigvee_{i=1}^m B_a^c\psi_i
    &
    \text{(Len)}
    \\[1em]
    \textstyle [(\phi_1,\dots,\phi_m\mathbb{E}_a\psi_1,\dots,\psi_m)
    \land \bigwedge_{i=1}^m B_a^c\phi_i] \to
    \bigvee_{i=1}^m B_a^c\psi_i
    &
    \text{(wLen)}
  \end{array}
  \]
  If $m=1$, then $\bigwedge_{i=2}^m \check B_a^c\phi_i$ is $\top$.
  Note that (Len) and (wLen) are meant to encompass the respectively
  indicated schemes for each positive integer $m\in\Int^+$.
\end{definition}

(Len) says that if agent $a$ knows the number of true $\phi_i$'s is
equal to the number of true $\psi_i$'s, agent $a$ believes $\phi_1$
with threshold $c$, and the remaining $\phi_i$'s are each consistent
with agent $a$'s threshold-$c$ beliefs, then agent $a$ believes one of
the $\psi_i$'s with threshold $c$.  Adapting a proof of Segerberg
\cite{Segerberg1971:qpiams}, we show that belief with threshold
$c\in(0,\frac 12]\cap\Rat$ satisfies (Len).

(wLen) is a weaker version of (Len), in the sense that its antecedent
has stronger requirements.  In particular, (wLen) says that if agent
$a$ knows the number of true $\phi_i$'s is equal to the number of true
$\psi_i$'s and agent $a$ believes all $\psi_i$'s with threshold $c$,
then agent $a$ believes one of the $\psi_i$'s with threshold $c$.  The
proof of Segerberg can also be adapted to show that belief with any
threshold $c\in(0,1)\cap\Rat$ satisfies (wLen).

We report these two results along with a number of other properties
in the following proposition.

\begin{proposition}[Properties of belief as willingness to bet]
  \label{prop:belief}
  We have:
  \begin{enumerate}
  \item \label{item:B-not-normal} $\not\modelsp
    B_a^c(\varphi\to\psi)\to(B_a^c\varphi\to B_a^c\psi)$.

    Belief is not closed under logical consequence.

    (So $B_a^c$ is not a normal modal operator.)

  \item \label{item:B-not-T} $\not\modelsp B_a^c\varphi\to\varphi$.

    Belief is not veridical.

  \item \label{item:B-C} $\modelsp K_a\phi\to B_a^c\phi$.

    What is known is believed.

  \item \label{item:B-B} $\modelsp\lnot B^c_a\bot$.

    The propositional constant $\bot$ for falsehood is not believed.

  \item \label{item:B-N} $\modelsp B_a^c\top$.

    The propositional constant $\top$ for truth is believed.

  \item \label{item:B-Ap} $\modelsp B_a^c\phi\to K_aB_a^c\phi$.

    What is believed is known to be believed.

  \item \label{item:B-An} $\modelsp \lnot B_a^c\phi\to K_a\lnot
    B_a^c\phi$.

    What is not believed is known to be not believed.

  \item \label{item:B-M} $\modelsp K_a(\phi\to\psi)\to(B_a^c\phi\to B_a^c\psi)$.

    Belief is closed under known logical consequence.

  \item \label{item:B-LT} If $c\in(0,\frac 12)$, then $\modelsp \check
    B_a^c\phi \to B_a^c\phi$.

    Low-threshold belief is decisive: disbelief in $\lnot\phi$ implies
    belief in $\phi$.

  \item \label{item:B-D} If $c\in[\frac 12,1)$, then $\modelsp
    B_a^c\phi\to \check B_a^c\phi$.

    High-threshold belief is consistent: belief in $\phi$ implies
    disbelief in $\lnot\phi$.

  \item \label{item:B-SC} If $c\in(0,\frac 12]$, then $\modelsp
    \check{B}_a^c \phi \land \check{K}_a(\neg \phi \land \psi)
    \rightarrow B_a^c (\phi \lor \psi)$.
    
    For each threshold $c\in(0,\frac 12]$, if $\phi$ is consistent
    with agent $a$'s threshold-$c$ beliefs and $\lnot\phi\land\psi$ is
    consistent with agent $a$'s knowledge, then agent $a$ believes
    $\phi\lor\psi$ with threshold $c$.

  \item \label{item:B-Len} If $c\in(0,\frac 12]$, then
    \[
    \textstyle \modelsp
    [(\phi_1,\dots,\phi_m\mathbb{E}_a\psi_1,\dots,\psi_m) \land
    B_a^c\phi_1 \land \bigwedge_{i=2}^m \check B_a^c\phi_i] \to
    \bigvee_{i=1}^m B_a^c\psi_i\enspace.
    \]

    Belief with threshold $c\in(0,\frac 12]$ satisfies (Len).

  \item \label{item:B-wLen}
    $\modelsp[(\phi_1,\dots,\phi_m\mathbb{E}_a\psi_1,\dots,\psi_m) \land
    \bigwedge_{i=1}^m B_a^c\phi_i] \to \bigvee_{i=1}^m B_a^c\psi_i$.

    Belief satisfies (wLen).
  \end{enumerate}
\end{proposition}
\begin{proof}
  We consider each item in turn.
  \begin{enumerate}
  \item Given $c\in(0,1)\cap\Rat$ and integers $p$ and $q$ such that
    $p/q=c$, we define $\M$ as the modification of the model
    $\M_{\ref{ExampleHorseRacing3}}$ of
    Example~\ref{ExampleHorseRacing3} obtained by changing lottery
    $l_0$ as follows:
    \[
    l_0 := \left\{ w_1:\frac{q-p}2,\; w_2:p,\; w_3:\frac{q-p}2
    \right\}\enspace.
    \]
    Since $1\leq p<q$, it follows that
    \begin{eqnarray*}
      P^\M_{a,w_1,0}(\semp{\lnot h_1\to h_2}^\M) &=& 
      (p+(q-p)/2)/q > p/q \enspace,
      \\
      P^\M_{a,w_1,0}(\semp{\lnot h_1}^\M) &=& 
      (p+(q-p)/2)/q > p/q 
      \enspace,\text{ and}
      \\
      P^\M_{a,w_1,0}(\semp{h_2}^M)&=&p/q\enspace.
    \end{eqnarray*}
    Therefore, we have
    \[
    \M,w_1,0\modelsp B^c_a(\lnot h_1\to h_2)\land B^c_a\lnot h_1\land
    \lnot B^c_ah_2\enspace.
    \]
    
  \item For $\M$ defined in the proof of Item~\ref{item:B-not-normal},
    we have
    \[
    \M,w_1,0\modelsp h_1\land B_a^c\lnot h_1\enspace.
    \]
    
  \item $\M,w,i\modelsp K_a\phi$ means $P^\M_{a,w,i}(\semp{\phi}^\M)=1$.
    Since $c\in(0,1)$, it follows that
    $P^\M_{a,w,i}(\semp{\phi}^\M)>c$, which is what it means to have
    $\M,w,i\modelsp B_a^c\phi$.

  \item $P^\M_{a,w,i}(\semp{\bot}^\M)=0<c\in(0,1)$.  Hence
    $\M,w,i\models\lnot B^c_a\bot$.

  \item $P^\M_{a,w,i}(\semp{\top}^\M)=1>c\in(0,1)$. Hence
    $\M,w,i\modelsp B_a^c\top$.

  \item We note that since $R_a$ and $E_a$ are equivalence relations,
    we have
    \begin{equation}
      wR_au \text{ and } iE_aj \text{ implies } 
      P^\M_{a,w,i}(\semp{\phi}^\M)= P^\M_{a,u,j}(\semp{\phi}^\M)\enspace.
      \label{eq:PeqP}
    \end{equation}
    So for Item~\ref{item:B-Ap}, it follows from $\M,w,i\modelsp
    B_a^c\phi$ that $P^\M_{a,w,i}(\semp{\phi}^\M)>c$.  To show that
    $\M,w,i\modelsp K_aB_a^c\phi$, we must argue that
    \[
    P^M_{a,w,i}(\semp{B_a^c\phi}^\M)= \frac{\sum\{l_j(u) \mid
      (u,j)\in[w,i]_a\cap \semp{B_a^c\phi}^\M\}} {\sum\{l_j(u) \mid
      (u,j)\in[w,i]_a\}} = 1\enspace.
    \]
    Choosing $(u,j)\in[w,i]_a$, it suffices to show that
    $(u,j)\in\semp{B_a^c\phi}^\M$.  By \eqref{eq:PeqP}, we have
    $P^\M_{a,u,j}(\semp{\phi}^\M)=P^\M_{a,w,i}(\semp{\phi}^\M)>c$, which
    implies $(u,j)\in\semp{B_a^c\phi}^\M$.  The result follows.
    
  \item The argument is similar to that for Item~\ref{item:B-Ap},
    though we note that $\M,w,i\modelsp\lnot B_a^c\phi$ implies
    $P^\M_{a,w,i}(\semp{\phi}^\M)\leq c$.
    
  \item Let us assume that $\M,w,i\modelsp K_a(\phi\to\psi)$ and
    $\M,w,i\modelsp B_a^c\phi$.  This means that
    $P^\M_{a,w,i}(\semp{\phi\to\psi}^\M)=1$ and
    $P^\M_{a,w,i}(\semp{\phi}^\M)>c$.  But then it follows that
    $P^\M_{a,w,i}(\semp{\psi}^\M)>c$ as well, which is what it means to
    have $\M,w,i\modelsp B_a^c\psi$.

  \item By Lemma~\ref{lemma:dual}.

  \item Assume $c\in[\frac 12,1)$ and $\M,w,i\modelsp B_a^c\phi$.
    Then $P^\M_{a,w,i}(\semp{\phi}^\M)>c$.  But $c\in[\frac 12,1)$
    implies $c\geq 1-c$, so $P^\M_{a,w,i}(\semp{\phi}^\M)\geq 1-c$.
    The result therefore follows by Lemma~\ref{lemma:dual}.

  \item Assume $c\in(0,\frac 12]$ and $\M,w,i\modelsp\check B^c_a\phi$.  By
    Lemma~\ref{lemma:dual}, it follows that
    $P^\M_{a,w,i}(\semp{\phi}^\M)\geq c$.  Let us assume further that
    $\M,w,i\modelsp\check K_a(\lnot\phi\land\psi)$.  This means
    \[
    P^\M_{a,w,i}(\semp{\lnot(\lnot\phi\land\psi)}^\M)\neq 1\enspace,
    \]
    which implies there exists $(v,k)\in[w,i]_a\cap
    \semp{\lnot\phi\land\psi}^\M$.  Since $l_k(v)>0$, it follows that
    \[
    \renewcommand{\arraystretch}{1.3}
    \begin{array}{cl}
      &
      \textstyle \sum\{l_j(u) \mid (u,j)\in[w,i]_a\cap
      \semp{\phi\lor\psi}^\M\}
      \\
      \geq &
      \textstyle \sum\{l_j(u) \mid (u,j)\in[w,i]_a\cap
      \semp{\phi}^\M\} + l_k(v)
      \\
      \geq &
      c+l_k(v) > c
    \end{array}
    \]
    Hence $P^\M_{a,w,i}(\semp{\phi\lor\psi}^\M)>c$.  That is,
    $\M,w,i\modelsp B_a^c(\phi\lor\psi)$.
    
  \item Given $c\in(0,\frac 12)$, we assume the following:
    \begin{eqnarray}
      &&
      \M,w,i\modelsp \phi_1,\dots,\phi_m\mathbb{E}_a\psi_1,\dots,\psi_m
      \label{eq:S}
      \\
      &&
      \M,w,i\modelsp B^c_a\phi_1
      \label{eq:Bphi1}
      \\
      &&
      \textstyle \M,w,i\modelsp \bigwedge_{i=2}^m\check B^c_a\phi_i
      \label{eq:checkB}
    \end{eqnarray}
    For each formula $\chi$, we define the set
    \[
    s(\chi):=\{l_j(u)\mid wR_au \text{ and } iE_aj \text{ and }
    M,u,j\modelsp\chi\}
    \]
    of all weights from world-index pairs that are epistemically
    accessible from $(w,i)$ and that satisfy $\chi$. Setting $A:=\sum
    s(\top)$, we observe that
    \[
    P^\M_{a,w,i}(\semp{\chi}^\M)=\frac{\sum s(\chi)}A\enspace.
    \]
    Adapting a proof due to Segerberg \cite{Segerberg1971:qpiams}, a
    subset $X\subseteq s(\top)$ is called a \emph{cell} if there
    exists an $r\in\{0,\dots,m\}$ such that
    \begin{equation}
      \textstyle X=s(\bigwedge_{i=1}^m d_i\phi_i \land \bigwedge_{i=1}^m
      e_i\psi_i)\enspace,
      \label{eq:cell}
    \end{equation}
    where exactly $r$ of the $d_i$'s are the empty string, exactly $r$
    of the $e_i$'s are the empty string, and the rest of the $d_i$'s
    and $e_i$'s are the negation sign $\lnot$.  We say that
    \emph{$d_i$ is empty} in a cell $X$ to mean that in the equation
    \eqref{eq:cell}, the symbol $d_i$ denotes the empty string.  The
    meaning of saying \emph{$e_i$ is empty} in a cell is defined
    analogously. We let $\mathcal{C}$ denote the set of cells.  As a
    consequence of \eqref{eq:S}, it follows that
    \begin{eqnarray*}
      \textstyle \bigcup\{X\in\mathcal{C}\mid\text{$d_i$ is empty in
        $X$}\} 
      &=&
      s(\phi_i) \enspace\text{and}
      \\
      \textstyle
      \bigcup\{X\in\mathcal{C}\mid\text{$e_i$ is empty in $X$}\}
      &=&
      s(\psi_i) \enspace\text{.}
    \end{eqnarray*}
    The cells are pairwise disjoint.  Hence
    \begin{eqnarray*}
      \textstyle
      \sum s(\phi_i) &=&
      \sum_{\text{$d_i$ is empty in $X\in\mathcal{C}$}} X \enspace\text{and}
      \\
      \textstyle
      \sum s(\psi_i) &=&
      \sum_{\text{$e_i$ is empty in $X\in\mathcal{C}$}} X \enspace\text{.}
    \end{eqnarray*}
    We define $S:=\sum_{i=1}^m\sum s(\phi_i)$ and $T:=\sum_{i=1}^m\sum
    s(\psi_i)$.  It follows that
    \begin{eqnarray*}
      S &=&
      \sum_{i=1}^m\sum_{\text{$d_i$ is empty in $X\in\mathcal{C}$}} X \enspace\text{and}
      \\
      T &=&
      \sum_{i=1}^m\sum_{\text{$e_i$ is empty in $X\in\mathcal{C}$}} X \enspace\text{.}
    \end{eqnarray*}
    Given $X\in\mathcal{C}$, the set $X$ will occur in the sum $S$
    exactly as many times as it occurs in the sum $T$.  Hence $S=T$.
    By \eqref{eq:Bphi1}, we have $(\sum s(\phi_1))/A>c$.  By
    \eqref{eq:checkB}, our assumption that $c\in(0,\frac 12)$, and
    Lemma~\ref{lemma:dual}, we have $(\sum s(\phi_i))/A\geq c$ for
    each $i\in\{2,\dots,m\}$.  Hence
    \[
    \frac SA = \frac{\sum s(\phi_1)}A + \sum_{i=2}^m\frac{\sum
      s(\phi_i)}A > c + (m-1)c = mc\enspace.
    \]
    It follows that $T/A>mc$; that is,
    \begin{equation}
      \sum_{i=1}^m \frac{\sum s(\psi_i)}A > mc\enspace.
      \label{eq:Len-sum}
    \end{equation}
    But since $\sum s(\psi_i)>0$ for each $i\in\{1,\dots,m\}$, each
    term $(\sum s(\psi_i))/A$ of the sum \eqref{eq:Len-sum} is
    positive.  Therefore, at least one of these terms must exceed $c$.
    Hence $\M,w,i\modelsp\bigvee_{i=1}^m B_a^c\psi_i$.

    \XXXcomment{Not sure I am completely happy with the clarity of the
      proof of Item~\ref{item:B-Len} (my adaptation of Segerberg's
      proof).  I wonder if it might be more straightforward to simply
      argue $ \sum s(\phi_1) + \cdots + \sum s(\phi_m) = \sum
      s(\psi_1) + \cdots + \sum s(\psi_m) $ directly, maybe using our
      notation $F(\phi)$ for $\sum s(\phi)$. \hfill --br}

  \item The proof is similar to that for the previous item, except
    that $c\in(0,1)$ and the assumption \eqref{eq:checkB} is replaced
    the by assumption
    \begin{equation}
      \textstyle
      \M,w,i\modelsp \bigwedge_{i=2}^m B^c_a\phi_i
      \label{eq:wLen-B}
    \end{equation}
    It follows from \eqref{eq:Bphi1} and \eqref{eq:wLen-B} that $(\sum
    s(\phi_i))/A>c$ for each $i\in\{1,\dots,m\}$ and hence we have
    $T/A=S/A>mc$ by the argument in the previous item.  It follows
    that $\M,w,i\modelsp\bigvee_{i=1}^m B_a^c\psi_i$.  \qedhere
  \end{enumerate}
\end{proof}

\section{Epistemic Neighborhood Models}
\label{Section:ENM}

The modal formulas $K_a\phi$ and $B_a^c\phi$ were taken as
abbreviations in the language $\Lang$ of multi-agent probability
logic.  We wish to consider a propositional modal language that has
knowledge and belief operators as primitives.

\begin{definition}
  The language $\Lang_\KB$ of \emph{multi-agent knowledge and belief}
  is defined by the following grammar.
  \begin{eqnarray*}
    \phi & ::= & 
    \top \mid p \mid \neg\phi \mid \phi\land\phi \mid
    K_a\phi \mid B_a\phi
    \\
    &&
    \text{\footnotesize 
      $p\in\Prop$,
      $a\in A$
    }
  \end{eqnarray*}
  We adopt the usual abbreviations for other Boolean connectives and
  define the dual operators $\check K_a:=\lnot K_a\lnot$ and $\check
  B_a:=\lnot B_a\lnot$.
\end{definition}

Our goal will be to develop a possible worlds semantics for
$\Lang_\KB$ that links with the probabilistic setting by making the
following translation truth-preserving.

\begin{definition}
  \label{definition:s}
  For each $c\in(0,1)\cap\Rat$, we define a translation $c:
  \Lang_\KB \to \Lang$ as follows.
  \[
  \renewcommand{\arraystretch}{1.3}
  \begin{array}{ccl@{\qquad}l}
    \top^c & := & \top
    \\
    p^c & := & p 
    \\
    (\neg \phi)^c & := & \neg \phi^c 
    \\
    (\phi \land \psi)^c & := & \phi^c \land \psi^c 
    \\
    (K_a\phi)^c & := & P_a(\phi^c) = 1
    & (= K_a\phi^c \text{ in } \Lang)
    \\
    (B_a\phi)^c & := & \textstyle P_a(\phi^c) > c
    & (= B_a^c\phi^c \text{ in } \Lang)
  \end{array}
  \]
\end{definition} 

Since we have seen that the probabilistic belief operator $B_a^c$ is
not a normal modal operator (Proposition~\ref{prop:belief}), we opt
for a neighborhood semantics for $\Lang_\KB$ \cite[Ch 7]{Chellas:ml}
with a new epistemic twist.

\begin{definition} 
  An \emph{epistemic neighborhood model} is a structure
  \[
  \M=(W,R,N,V)
  \]
  satisfying the following.
  \begin{itemize}
  \item $W$ is a nonempty set of ``worlds.''

  \item $R : A \to \pow (W \times W)$ is a function that assigns to each
    agent $a\in A$ an equivalence relation $R_a$ on $W$.  We let
    \[
    [w]^\M_a:=\{v\in W\mid wR_av\}
    \]
    denote the equivalence class of $w$ under $R_a$.  The superscript
    $\M$ may be dropped when doing so ought not cause confusion.

  \item $N : A\times W \to \pow(\pow(W))$ is a \emph{neighborhood
      function} that assigns to each agent $a\in A$ and world $w\in W$
    a collection $N_a(w)$ of sets of worlds---each such set called a
    \emph{neighborhood} of $w$---subject to the following conditions.
    \begin{description}
    \item[(c)] $\forall X \in N_a(w) : X \subseteq [w]_a$.

    \item[(f)] $\emptyset\notin N_a(w)$.
      
    \item[(n)] $[w]_a\in N_a(w)$.
      
    \item[(a)] $\forall v \in [w]_a : N_a(v) = N_a(w)$.

    \item[(m)] $\forall X \subseteq Y \subseteq [w]_a : 
      \text{ if } X \in N_a(w) \text{, then } Y \in N_a(w)$.
    \end{description}

  \item $V:W\to\pow(\Prop)$ assigns a set $V(w)$ of propositional
    letters to each world $w\in W$.
  \end{itemize}
  A \emph{pointed epistemic neighborhood model} is a pair $(\M,w)$
  consisting of an epistemic neighborhood model $\M$ and a world $w$
  in $M$.
\end{definition} 

An epistemic minimal model is a variation of a minimal model that
includes an epistemic component $R_a$ for each agent $a$.
Intuitively, $[w]_a$ is the set of worlds agent $a$ knows to be
possible at $w$ and each $X\in N_a(w)$ represents a proposition that
the agent believes at $w$.  The condition that $R_a$ be an equivalence
relation ensures that knowledge is closed under logical consequence,
veridical (i.e., only true things can be known), positive
introspective (i.e., the agent knows what she knows), and negative
introspective (i.e., the agent knows what she does not know).

Property (c) ensures that the agent does not believe a proposition
$X\subseteq W$ that she knows to be false: if $X$ contains a world in
$w'\in(W-[w]_a)$ that the agent knows is not possible with respect to
the actual world $w$, then she knows that $X$ cannot be the case and
hence she does not believe $X$.  Property (f) ensures that no logical
falsehood is believed, while Property (n) ensures that every logical
truth is believed.  Property (a) ensures that if $X$ is believed, then
it is known that $X$ is believed.  Property (m) says that belief is
monotonic: if an agent believes $X$, then she believes all
propositions $Y\supseteq X$ that follow from $X$.

Here are a few other properties that will arise later on.

\begin{definition}[Extra Properties]
  \label{definition:extra-properties}
  Let $\M=(W,R,N,V)$ be an epistemic neighborhood model.  The
  following are a list of properties that $\M$ may satisfy.
  \begin{description}
  \item[(lt)] $\forall X\subseteq[w]_a: \text{ if } [w]_a-X\notin
    N_a(w) \text{, then } X\in N_a(w)$.

  \item[(d)] $\forall X \in N_a(w): [w]_a - X \notin  N_a(w)$.

  \item[(sc)] $\forall X,Y\subseteq[w]_a: \text{ if } [w]_a-X\notin
    N_a(w) \text{ and } Y\supsetneq X \text{, then } Y\in N_a(w)$.

  \item[(l)] $\forall m\in\Int^+,\forall
    X_1,\dots,X_m,Y_1,\dots,Y_m\subseteq[w]_a:$
    \[
    \renewcommand{\arraystretch}{1.3}
    \begin{array}{ll}
      \text{if }
      &
      \begin{array}[t]{l}
        X_1,\dots,X_m\mathbb{E}_aY_1,\dots,Y_n \quad\text{and}
        \\
        X_1\in N_a(w) \quad\text{and}
        \\{}
        \forall i\in\{2,\dots,m\}:
        [w]_a-X_i\notin N_a(w) \text{,}
      \end{array}
      \\
      \text{then }
      &
      \exists j\in\{1,\dots,m\}: Y_j\in N_a(w)\text{.}
    \end{array}
    \]

    Notation: $X_1,\dots,X_m\mathbb{E}_aY_1,\dots,Y_n$ means that for
    each $v\in W$, the number of $X_i$'s containing $v$ is equal to
    the number of $Y_i$'s containing $v$.

  \item[(wl)] $\forall m\in\Int^+,\forall
    X_1,\dots,X_m,Y_1,\dots,Y_m\subseteq[w]_a:$
    \[
    \renewcommand{\arraystretch}{1.3}
    \begin{array}{ll}
      \text{if }
      &
      \begin{array}[t]{l}
        X_1,\dots,X_m\mathbb{E}_aY_1,\dots,Y_n \quad\text{and}
        \\
        \forall i\in\{1,\dots,m\}:
        X_i\in N_a(w) \text{,}
      \end{array}
      \\
      \text{then }
      &
      \exists j\in\{1,\dots,m\}: Y_j\in N_a(w)\text{.}
    \end{array}
    \]
  \end{description}
\end{definition}

Property (lt) says that if the agent does not believe the complement
of $X$, then she must believe $X$.  Property (d) ensures that beliefs
are consistent (in the sense that the agent does not believe both $X$
and its complement $[w]_a-X$). Property (sc) is a form of ``strong
commitment'': if the agent does not believe the complement of $X$,
then she must believe any strict superset $Y\supsetneq X$ that follows
from $X$.  Properties (l) and (wl) are versions of the Lenzen scheme
(Len) and the weak Lenzen scheme (wLen) from
Definition~\ref{definition:lenzen-schemes}.  We will come back to
these properties later.

We now turn to the definition of truth for the language $\Lang_\KB$.

\begin{definition} 
  Let $\M = (W,R,N,V)$ be an epistemic neighborhood model.  We define
  a binary truth relation $\modelsn$ between a pointed epistemic
  neighborhood model $(\M,w)$ and $\Lang_\KB$-formulas and a function
  $\semn{\cdot}^\M:\Lang_\KB\to \pow(W)$ as follows.
  \begin{eqnarray*} 
    \semn{\phi}^\M & := & \{v\in W\mid \M,v\modelsn\phi\}
    \\
    \M, w \modelsn p & \text{ iff } & p \in V(w) 
    \\
    \M, w \modelsn \neg \phi & \text{ iff } & \M, w \not\modelsn \phi 
    \\
    \M, w \modelsn \phi\land\psi  & \text{ iff } 
    & \M, w \modelsn \phi \text{ and } \M, w \modelsn \psi
    \\
    \M, w \modelsn K_a \phi  & \text{ iff } & 
    [w]_a\subseteq\semn{\phi}^\M
    \\
    \M, w \modelsn B_a \phi  & \text{ iff } &
    [w]_a\cap \semn{\phi}^\M \in N_a(w)
  \end{eqnarray*}
  Validity of $\phi\in\Lang_\KB$ in an epistemic neighborhood model
  $\M$, written $\M\modelsn\phi$, means that $\M,w\modelsn\phi$ for
  each world $w\in W$.  Validity of $\phi\in\Lang_\KB$, written
  $\modelsn\phi$, means that $\M\modelsn\varphi$ for each epistemic
  neighborhood model $\M$.  For a class $\mathcal{C}$ of epistemic
  neighborhood models, we write $\mathcal{C}\modelsn\phi$ to mean that
  $\M\modelsn\phi$ for each $\M\in\mathcal{C}$.
\end{definition}

Intuitively, $K_a\phi$ is true at $w$ iff $\phi$ holds at all worlds
epistemically possible with respect to $w$, and $B_a\phi$ holds at $w$
iff the epistemically possible $\phi$-worlds make up a neighborhood of
$w$.  Note that it follows from this definition that the dual for
belief $\check{B}_a \phi$ is true at $w$ iff
$[w]_a\cap\semn{\neg\phi}^\M\notin N_a(w)$.

\section{Relating Belief with Willingness to Bet}
\label{Section:BeliefBet}

We now relate the definition of belief in neighborhood models to the
notion of belief as willingness to bet. For every epistemic
probability model $\M$ and threshold $c\in(0,1)\cap\Rat$, we define
an epistemic neighborhood model $\M^c$ in which the lotteries are
eliminated.

\begin{definition} Given a probability model $\M = (W,R,V,n,l,E)$ and
  a threshold $c\in(0,1)\cap\Rat$, we define the structure
 \[
 \M^c = (W^c,R^c,N^c, V^c)
 \]
 as follows.
 \begin{itemize} 
 \item $W^c := W\times\bar{n}$.

 \item $R^c_a:=\{((w,i),(u,j))\in W^c\times W^c\mid wR_au \text{ and }
   iE_aj\}$.

   When it ought not cause confusion, we let $[w,i]_a$ denote the set
   \[
   [(w,i)]^{\M^c}_a:=\{(u,j)\in W^c\mid (w,i)R^c_a(u,j)\} \enspace.
   \]

 \item To define $N^c$, let $F:\pow(W\times \bar{n})\to \Rat^+$ be
   given by
   \[
   F(X) := 
   \sum \{ l_j(u) \mid (u,j) \in X \} \enspace,
   \]
   and then set
   \[
   \textstyle N^c_a (w,i) := \{ X \subseteq [w,i]_a
   \mid F(X) > c\cdot F([w,i]_a) \} \enspace.
   \]
   
 \item $V^c(w,i) := V(w)$.
 \end{itemize}
\end{definition}

The worlds of $\M^c$ are world-index pairs from $\M$, the worlds that
are epistemically indistinguishable from $(w,i)$ are determined
coordinatewise by the indistinguishability relations $R_a$ and $E_a$,
and the neighborhoods of $(w,i)$ consist of all collections
$X\subseteq [w]_a$ of epistemically indistinguishable worlds whose
total lottery value $F(X)$ exceeds $c$ times the total lottery value
$F([w,i]_a)$ of all epistemically indistinguishable worlds.
Intuitively, agent $a$ believes a proposition $X\in N^c_a(w,i)$ if and
only if $X$ is epistemically possible (i.e., $X\subseteq[w,i]_a$) and
the probability of $X$ is greater than $c$ (i.e.,
$F(X)/F([w,i]_a)>c$).

\begin{lemma}[Correctness]
  \label{lemma:correctness}
  If $\M$ is a probability model and $c\in(0,1)\cap\Rat$, then $\M^c$
  is an epistemic neighborhood model.  Furthermore:
  \begin{itemize}
  \item if $c\in(0,\frac 12)\cap\Rat$, then $\M^c$ satisfies (lt),
    (sc), and (l);

  \item $\M^{1/2}$ satisfies (d), (sc), and (l); and

  \item if $c\in(\frac 12,1)\cap\Rat$, then $\M^c$ satisfies (d) and
    (wl).
  \end{itemize}
\end{lemma}
\begin{proof}
  % XXX; check this!
  $W^c$ is nonempty because $W$ and $\bar{n}$ are nonempty.  $R^c_a$
  is an equivalence relation because it is the product of the
  equivalence relations $R_a$ and $E_a$.  We verify that $N^c_a$
  satisfies properties (c), (f), (n), (a), and (m).  Notation: let
  $A:=[w,i]_a$, let $X$ and $Y$ (possibly with subscripts) denote
  subsets of $A$, and let $X'$ denote $A-X$.
  \begin{itemize}
  \item For (c), $X\in N^c_a(w,i)$ implies $X\subseteq[w,i]_a$ by
    definition.

  \item For (f), we have $F(\emptyset)=0<c\cdot F(A)$ because
    $c\in(0,1)\cap\Rat$ and $F(A)>0$.  Hence $\emptyset\notin
    N^c_a(w,i)$.

  \item For (n), we have $F(A) > c\cdot F(A)$ because
    $c\in(0,1)\cap\Rat$, and therefore $[w,i]_a \in N^c_a(w,i)$.

  \item For (a), suppose $X \in N^c_a(w,i)$. Then $F(X) > c\cdot
    F([w,i]_a)$.  Let $(v,j) \in [w,i]_a$. Then $[w,i]_a = [v,j]_a$,
    and therefore $F(X) > c\cdot F([v,j]_a)$. Thus $X \in N^c_a(v,j)$.

  \item For (m), assume $X \in N^c_a(w,i)$.  Then $F(X) > c\cdot
    F(A)$. Hence for $Y$ satisfying $X \subseteq Y \subseteq A$, we
    have $F(Y) \geq F(X) > c\cdot F(A)$ and therefore that $Y \in
    N^c_a(w,i)$.
  \end{itemize}
  Now let us check the remaining properties conditional on the value
  of the threshold $c\in(0,1)\cap\Rat$.
  \begin{itemize}
  \item For (lt), we assume $c\in(0,\frac 12)\cap\Rat$ and $X'\notin
    N^c_a(w,i)$.  Then $F(X')\leq c\cdot F(A)$ and therefore $F(X)\geq
    (1-c)\cdot F(A)$.  Since $1-c>c$, it follows that $F(X)> c\cdot
    F(A)$ and hence that $X\in N^c_a(w,i)$.

  \item For (d), we assume $c\in[\frac 12,1)\cap\Rat$ and $X \in
    N^c_a(w,i)$. Then $F(X) > c\cdot F(A)$ and therefore $F(X') \leq
    (1-c)\cdot F(A)$. Since $1-c\leq c$, it follows that $F(X')\leq
    c\cdot F(A)$ and hence that $X' \notin N^c_a(w,i)$.

  \item For (sc), we assume $c\in(0,\frac 12]\cap\Rat$, $X'\notin
    N^c_a(w,i)$, and $Y\supsetneq X$.  Now the second assumption
    implies $F(X')\leq c\cdot F(A)$.  Hence $F(X)\geq(1-c)\cdot F(A)$.
    Since $1-c\geq c$ and $Y\supsetneq X$, we have $F(Y)>F(x)\geq
    c\cdot F(A)$ and hence that $Y\in N^c_a(w,i)$.

  \item For (l), we assume $c\in(0,\frac 12]\cap\Rat$ along with the
    following:
    \begin{eqnarray}
      &&
      X_1,\dots,X_m\mathbb{E}_aY_1,\dots,Y_n
      \label{eq:prop-l:E}
      \\
      &&
      X_1\in N^c_a(w,i) 
      \label{eq:prop-l:X1}
      \\
      &&
      \forall i\in\{2,\dots,m\}: X_i'\notin N^c_a(w,i)
      \label{eq:prop-l:Xcs}
    \end{eqnarray}
    From \eqref{eq:prop-l:E} it follows (by an argument similar to the
    argument that $S=T$ in the proof of Item~\ref{item:B-Len} of
    Proposition~\ref{prop:belief}) that \XXXcomment{If we drop the
      Segerberg argument in Prop.~\ref{prop:belief}, then we need to
      change the wording here.\hfill--br}
    \[
    F(X_1)+\cdots+F(X_m)=F(Y_1)+\cdots+F(Y_m)\enspace.
    \]
    Therefore,
    \begin{equation}
      \frac{F(X_1)}{F(A)}+\cdots+\frac{F(X_m)}{F(A)}=
      \frac{F(Y_1)}{F(A)}+\cdots+\frac{F(Y_n)}{F(A)}\enspace.
      \label{eq:sums-eq}
    \end{equation}
    From \eqref{eq:prop-l:X1}, we have $F(X_1)>c\cdot F(A)$.  From
    \eqref{eq:prop-l:Xcs}, we have for each $i\in\{2,\dots,m\}$ that
    $F(X_i')\leq c\cdot F(A)$ and therefore that
    \[
    F(X_i)\geq (1-c)\cdot F(A)\geq c\cdot F(A)\enspace,
    \]
    here making use of the fact that $c\in(0,\frac 12)$ implies
    $1-c\geq c$.  Hence the left-hand side of the sum
    \eqref{eq:sums-eq} is strictly greater than $mc$.  Since every
    summand on either side of the equality is positive, it follows
    that at least one member of the right-hand side of
    \eqref{eq:sums-eq} must exceed $c$.  That is, there exists
    $j\in\{1,\dots,m\}$ such that $F(Y_j)>c\cdot F(A)$ and hence
    $Y_i\in N^c_a(w,i)$.

  \item For (wl), we assume $c\in(\frac 12,1)\cap\Rat$,
    \eqref{eq:prop-l:E}, and the following:
    \begin{eqnarray}
      &&
      \forall i\in\{1,\dots,m\}: X_i\in N^c_a(w,i)
      \label{eq:prop-wl:Xis}
    \end{eqnarray}
    As in the argument for (l), we have \eqref{eq:sums-eq}.  Further,
    it follows from \eqref{eq:prop-wl:Xis} that $F(X_i)>c\cdot F(A)$
    for each $i\in\{1,\dots,m\}$.  But then the left-hand side of
    \eqref{eq:sums-eq} exceeds $mc$, and hence the right-hand side of
    this sum does as well. That is, there exists $j\in\{1,\dots,m\}$
    such that $F(Y_j)>c\cdot F(A)$ and hence $Y_i\in N^c_a(w,i)$.
    \qedhere
  \end{itemize}
\end{proof}

\begin{theorem}[Betting and Certainty]
  \label{BettingTheorem}
  For each epistemic probability model
  \[
  \M = (W,R,V,n,l,E)\enspace,
  \]
  pair $(w,i)\in W\times\bar{n}$, threshold $c\in(0,1)\cap\Rat$, and
  formula $\phi \in \Lang_\KB$, we have:
  \[
  \M^c,(w,i)\modelsn\phi \quad\text{iff}\quad \M,w,i\modelsp\phi^c
  \enspace.
  \]
\end{theorem}
\begin{proof}
  Induction on the structure of $\phi$. The non-modal cases are
  obvious.

  We first consider knowledge formulas. Assume $\M^c,(w,i)\modelsn
  K_a\psi$.  This means
  $[w,i]_a\subseteq\semn{\psi}^{\M^c}$. Applying the definition of
  $R^c_a$ and the induction hypothesis, this is equivalent to
  \[
  \{(u,j)\mid wR_au \text{ and } iE_aj\}\subseteq \{(u,j)\mid
  \M,u,j\modelsp \psi^c\} \enspace.
  \]
  Since $l_j(u)>0$ for each $(u,j)\in W\times\bar{n}$, this is
  equivalent to the statement that
  \[
  \frac{\sum\{l_j(u)\mid wR_au \text{ and } iE_aj\text{ and }
    \M,u,j\modelsp \psi^c\}} {\sum\{l_j(u)\mid wR_au\text{ and }
    iE_aj\}} = 1 \enspace.
  \]
  But the latter is what it means to have $\M,w,i\models
  P_a(\psi^c)=1$.

  Now we move to belief formulas. Assume $\M^c,(w,i)\modelsn
  B_a\psi$.  This means that
  \[
  [w,i]_a\cap\semn{\psi}^{\M^c}\in N^c_a(w,i) \enspace.
  \]
  According to the induction hypothesis, this is equivalent to: 
  \[
  [w,i]_a\cap \{(u,j)\mid \M,u,j\modelsp \psi^c\}\in N^c_a(w,i)
  \enspace.
  \]
  By the definition of $R^c_a$, this is in turn equivalent to:
  \begin{equation}
    \{(u,j)\mid wR_au \text{ and } iE_aj \text{ and } \M,u,j\modelsp\psi^c\}\in
    N^c_a(w,i) \enspace.
  \label{bettingstatement0}
  \end{equation}
  Since $X\in N^c_a(w,i)$ iff $F(X)> c\cdot F([w,i]_a)$ and
  \[
    F(X) = \sum \{l_j(u) \mid (u,j)\in X\} \enspace,
  \]
  it follows that \eqref{bettingstatement0} is equivalent to
  \[
  \frac{\sum\{l_j(u)\mid wR_au \text{ and } iE_aj\text{ and }
    \M,u,j\modelsp\psi^c\}} {\sum\{l_j(u)\mid wR_au\text{ and }
    iE_aj\}} > c \enspace.
  \]
  But this is what it means to have $\M,w,i\modelsp P_a(\psi^c)>c$.
\end{proof}

\section{Calculi for Belief as Willingness to Bet}

\begin{definition}
  The theory $\KB$ in the language $\Lang_\KB$ is defined in
  Table~\ref{table:KB}.  Using additional schemes from
  Table~\ref{table:other-schemes}, we also define the following
  theories in the language $\Lang_\KB$.
  \begin{itemize}
  \item $\KBlt$ is obtained from $\KB$ by adding (LT), (SC), and (L).

  \item $\KBeq$ is obtained from $\KB$ by adding (D), (SC), and (L).

  \item $\KBgt$ is obtained from $\KB$ by adding (D) and (WL).
  \end{itemize}
\end{definition}

\begin{table}[h]
  \begin{center}
    \textsc{Axiom Schemes}\\[.4em]
    \renewcommand{\arraystretch}{1.3}
    \begin{tabular}[t]{cl}
      (CL) &
      Schemes of Classical Propositional Logic
      \\
      (KS5) &
      $\mathsf{S5}$ axiom schemes for each $K_a$
      \\
      (C) &
      $K_a\phi\to B_a\phi$
      \\
      (F) &
      $\lnot B_a\bot$
      \\
      (N) &
      $B_a\top$
      \\
      (Ap) &
      $B_a\phi\to K_aB_a\phi$
      \\
      (An) &
      $\lnot B_a\phi\to K_a\lnot B_a\phi$
      \\
      (M) &
      $K_a(\phi\to\psi)\to(B_a\phi\to B_a\psi)$
    \end{tabular}
    \renewcommand{\arraystretch}{1.0}
    \\[1em]
    \textsc{Rules}\vspace{-.5em}
    \[
    \begin{array}{c}
      \varphi\to\psi \quad \varphi
      \\\hline
      \psi
    \end{array}\;\text{\footnotesize(MP)}
    \qquad
    \begin{array}{c}
      \varphi
      \\\hline
      K_a\varphi
    \end{array}\;\text{\footnotesize(MN)}
    \]
  \end{center}
  \caption{The theory $\KB$}
  \label{table:KB}
\end{table}

\begin{table}[h]
  \begin{center}
    \renewcommand{\arraystretch}{1.3}
    \begin{tabular}[t]{cl}
      (LT) &
      $\check B_a\phi\to B_a\phi$
      \\
      (D) &
      $B_a\phi\to \check B_a\phi$
      \\
      (SC) &
      $\check B_a\phi \land 
      \check K_a(\lnot\phi\land\psi) \to 
      B_a(\phi\lor\psi)$
      \\
      (L) &
      $\textstyle [(\phi_1,\dots,\phi_m\mathbb{E}_a\psi_1,\dots,\psi_m)
      \land B_a\phi_1 \land \bigwedge_{i=2}^m \check B_a\phi_i] \to
      \bigvee_{i=1}^m B_a\psi_i$
      \\
      (WL) &
      $\textstyle [(\phi_1,\dots,\phi_m\mathbb{E}_a\psi_1,\dots,\psi_m)
      \land \bigwedge_{i=1}^m B_a\phi_i] \to
      \bigvee_{i=1}^m B_a\psi_i$
      \\
    \end{tabular}
    
    \bigskip
    \begin{minipage}{.9\textwidth}
      The $\Lang_\KB$-formula
      $(\phi_1,\dots,\phi_m\mathbb{E}_a\psi_1,\dots,\psi_m)$ is given
      as in Definition~\ref{definition:segerberg-notation} except that
      all formulas are taken from the language $\Lang_\KB$.
    \end{minipage}
  \end{center}
    \caption{Other axiom schemes}
  \label{table:other-schemes}
\end{table}

\subsection{Results for the Basic Calculus $\KB$}

\begin{proposition}[$\KB$ Derivables]
  \label{prop:KB-derivables}
  We have each of the following.
  \begin{enumerate}
  \item $\vdash_\KB B_a(\phi\land\psi)\to(B_a\phi\land B_a\psi)$.

  \item $\vdash_\KB B_a\phi\to B_aB_a\phi$.

  \item $\vdash_\KB \lnot B_a\phi\to B_a\lnot B_a\phi$.

  \item $\vdash_\KB K_a(\phi\to\psi)\to(\check B_a\phi\to\check B_a\psi)$.

  \item $\vdash_\KB K_a\phi\land B_a\psi\to B_a(\phi\land\psi)$.

  \item $\vdash_\KB B_a\phi\leftrightarrow K_aB_a\phi$.

  \item $\vdash_\KB \lnot B_a\phi\leftrightarrow K_a\lnot B_a\phi$.

  \item $\vdash_\KB\phi$ implies $\vdash_\KB B_a\phi$.

  \item $\vdash_\KB\phi\to\psi$ implies $\vdash_\KB B_a\phi\to B_a\psi$.

  \item $\vdash_\KB\phi\to\psi$ implies $\vdash_\KB\check
    B_a\phi\to\check B_a\psi$.
  \end{enumerate}
\end{proposition}
\begin{proof}
  % XXX; show proofs of a couple?
\end{proof}

\begin{theorem}[$\KB$ Neighborhood Completeness]
  \label{theorem:KB-neighborhood-completeness}
  $\KB$ is sound and complete with respect to the class of epistemic
  neighborhood models:
  \[
  \forall\phi\in\Lang_\KB: \vdash_\KB\phi \text{ iff } {}\modelsn\phi
  \enspace.
  \]
  Further, every $\phi\in\Lang_\KB$ that is satisfiable at a pointed
  epistemic neighborhood model is satisfiable at a finite pointed
  epistemic neighborhood model.
\end{theorem}
\begin{proof}
  Soundness is by induction on the length of derivation.  We first
  verify soundness of the axioms.
  \begin{itemize}
  \item Validity of schemes of classical propositional logic is
    immediate.  Validity of schemes of $\mathsf{S5}$ follows by the
    fact that the $R_a$'s are equivalence relations
    \cite{BlaRijVen:ml}.

  \item Scheme (C) is valid: $\modelsn K_a\phi\to B_a\phi$.

    $\M,w\modelsn K_a\phi$ means $[w]_a\subseteq\semn{\phi}$.  Since
    $[w]_a\in N_a(w)$ by (n), we have $[w]_a\cap\semn{\phi}=[w]_a\in
    N_a(w)$.  That is, $\M,w\modelsn B_a\phi$.

  \item Scheme (F) is valid: $\modelsn\lnot B_a\bot$.

    $\semn{\bot}=\emptyset\notin N_a(w)$ by (f).  Hence
    $\M,w\not\modelsn B_a\bot$.

  \item Scheme (N) is valid: $\modelsn B_a\top$.

    $\semn{\top}\cap[w]_a=[w]_a\in N_a(w)$ by (n).  Hence
    $\M,w\modelsn B_a\top$.

  \item Scheme (Ap) is valid: $\modelsn B_a\phi\to K_a B_a\phi$.

    Suppose $\M,w\modelsn B_a\phi$. Then $[w]_a\cap\semn{\phi}\in
    N_a(w)$.  Take $v\in[w]_a$.  We have $[v]_a=[w]_a$ because $R_a$
    is an equivalence relation, and we have $N_a(v)=N_a(w)$ by (a).
    Hence $[v]_a\cap\semn{\phi}\in N_a(v)$; that is, $\M,v\modelsn
    B_a\phi$.  Since $v\in [w]_a$ was chosen arbitrarily, we have
    shown that $[w]_a\subseteq\semn{B_a\phi}$.  Hence $\M,w\modelsn
    K_a B_a\phi$.

  \item Scheme (An) is valid: $\modelsn \lnot B_a\phi\to K_a\lnot B_a\phi$.

    Replace $B_a\phi$ by $\lnot B_a\phi$ and $\in$ by $\notin$ in the
    argument for the previous item.

  \item Scheme (M) is valid: $\modelsn K_a(\phi\to\psi)\to(B_a\phi\to B_a\psi)$.

    Suppose $\M,w\modelsn K_a(\phi\to\psi)$ and $\M,w\modelsn
    B_a\phi$.  This means $[w]_a\subseteq\semn{\phi\to\psi}$ and
    $[w]_a\cap\semn{\phi}\in N_a(w)$. But then
    \[
    [w]_a\cap\semn{\phi}\subseteq
    [w]_a\cap\semn{\phi}\cap\semn{\phi\to\psi}\subseteq
    [w]_a\cap\semn{\psi}\enspace.
    \]
    Hence $[w]_a\cap\semn{\psi}\in N_a(w)$ by (m).
  \end{itemize}
  That validity is closed under applications of the rules MP and MN
  follows by the standard arguments \cite{BlaRijVen:ml}.

  Completeness is by a canonical model argument. The \emph{canonical
    model} for $\KB$ is the structure $\M=(W,R,N,V)$ obtained as
  follows.
  \begin{itemize}
  \item $W$ is the set of maximal $\KB$-consistent sets.\footnote{A
      \emph{$\KB$-consistent set} is a set of $\Lang_\KB$-formulas
      satisfying the property that $\bot$ is not $\KB$-derivable from
      any finite subset.  A \emph{maximal $\KB$-consistent set} is a
      $\KB$-consistent set satisfying the property that adding any
      $\Lang_\KB$-formula not already present will yield a set that is
      not $\KB$-consistent.}

  \item $\Gamma R_a\Delta$ means $\Gamma^{K_a}:=\{\phi\mid
    K_a\phi\in\Gamma\}\subseteq\Delta$.

    Since $K_a$ is an $\mathsf{S5}$ modal operator, it follows by
    standard arguments that $R_a$ is an equivalence relation
    \cite{BlaRijVen:ml}.  As before, we let
    \[
    [\Gamma]_a:=\{\Delta\in W\mid \Gamma R_a\Delta\}
    \]
    denote the equivalence class of $\Gamma$ under $R_a$.

  \item $V(\Gamma):=\Prop\cap\Gamma$.

  \item Define for each $\Gamma\in W$, $a\in A$, and
    $\phi\in\Lang_\KB$ the set
    \begin{eqnarray*}
      \Gamma_a(\phi) &:=&
      \{\Delta\in[\Gamma]_a\mid\phi\in\Delta\}
    \end{eqnarray*}
    and let
    \[
    N_a(\Gamma):= \{X\subseteq[\Gamma]_a\mid \exists B_a\phi\in\Gamma:
    X\supseteq\Gamma_a(\phi)\} \enspace.
    \]
  \end{itemize}
  We verify that $\M$ is an epistemic neighborhood model. $W$ is
  nonempty because $\KB$ is consistent, something it is easy to show
  by constructing a simple one-world epistemic neighborhood model and
  applying soundness.  $R_a$ is an equivalence relation by the
  standard modal arguments \cite{BlaRijVen:ml}.  What remains is to
  check that $N_a$ satisfies (c), (f), (n), (a), and (m).
  \begin{itemize}
  \item For (c), we must show $X\in N_a(\Gamma)$ implies
    $X\subseteq[\Gamma]_a$.  But this follows by the definition of
    $N_a(\Gamma)$.

  \item For (f), we must show that $\emptyset\notin N_a(\Gamma)$. So
    suppose toward a contradiction we had $\emptyset\in N_a(\Gamma)$.
    By the definition of $N_a$, there exists $B_a\phi\in\Gamma$ such
    that for every $\Delta\in[\Gamma]_a$, we have $\phi\notin\Delta$.
    It follows by the maximal consistency of $\Gamma$ that
    $K_a\lnot\phi\in\Gamma$ and hence $K_a(\phi\to\bot)\in\Gamma$ by
    modal reasoning and maximal consistency.  But then it follows by
    (M) and maximal consistency that $B_a\bot\in\Gamma$, contradicting
    the fact that we have $\lnot B_a\bot\in\Gamma$ by (F) and maximal
    consistency.  Conclusion: $\emptyset\notin N_a(\Gamma)$.

  \item For (n), we must show $[\Gamma]_a\in N_a(w)$.  But it follows
    by (N) and maximal consistency that $B_a\top\in\Gamma$ and
    therefore $\Gamma_a(\top)\in N_a(w)$.  Since we have
    $\Gamma_a(\top)=[\Gamma]_a$ by maximal consistency, the result
    follows.

  \item For (a), we must show that $\Delta\in[\Gamma]_a$ implies
    $N_a(\Delta)=N_a(\Gamma)$.  So take $\Delta\in[\Gamma]_a$.  By
    maximal consistency, (Ap), and the definition of $R_a$, we have
    \[
    B_a\phi\in\Gamma \Rightarrow
    K_aB_a\phi\in\Gamma \Rightarrow
    B_a\phi\in\Delta \enspace.
    \]
    By maximal consistency, (An), and the definition of $R_a$, we also
    have
    \[
    B_a\phi\notin\Gamma \Rightarrow
    \lnot B_a\phi\in\Gamma \Rightarrow
    K_a\lnot B_a\phi\in\Gamma \Rightarrow
    \lnot B_a\phi\in\Delta \Rightarrow
    B_a\phi\notin\Delta \enspace.
    \]
    Hence $B_a\phi\in\Gamma$ iff $B_a\phi\in\Delta$.  Further, since
    $R_a$ is an equivalence relation and $\Delta\in[\Gamma]_a$, we
    have $[\Delta]_a=[\Gamma]_a$ and
    $\Delta_a(\phi)=\Gamma_a(\phi)$. Therefore, we have
    \[
    \renewcommand{\arraystretch}{1.3}
    \begin{array}{lcl}
      X\in N_a(\Delta) 
      & \Leftrightarrow &
      \exists B_a\phi\in\Delta:
      \Delta_a(\phi)\subseteq X\subseteq[\Delta]_a)
      \\
      & \Leftrightarrow &
      \exists B_a\phi\in\Gamma:
      \Gamma_a(\phi)\subseteq X\subseteq[\Gamma]_a)
      \\
      & \Leftrightarrow &
      X\in N_a(\Gamma) \enspace.
    \end{array}
    \]

  \item For (m), we must show $X\in N_a(\Gamma)$ and $X\subseteq
    Y\subseteq[\Gamma]_a$ together imply that $Y\in N_a(\Gamma)$.  So
    assume $X\in N_a(\Gamma)$ and $X\subseteq Y\subseteq[\Gamma]_a$.
    We then have the following:
    \[
    \renewcommand{\arraystretch}{1.3}
    \begin{array}{lcl}
      X\in N_a(\Gamma)
      & \Leftrightarrow &
      \exists B_a\phi\in\Gamma:\Gamma_a(\phi)\subseteq X\subseteq[\Gamma]_a
      \\
      & \Rightarrow &
      \exists B_a\phi\in\Gamma:\Gamma_a(\phi)\subseteq
      X\subseteq Y\subseteq[\Gamma]_a
      \\
      & \Rightarrow &
      Y\in N_a(\Gamma)
      \enspace.
    \end{array}
    \]
  \end{itemize}
  So $\M$ is indeed an epistemic neighborhood model.  We prove the
  \emph{Truth Lemma\/}: for each $\phi\in\Lang_\KB$ and $\Gamma\in W$,
  we have $\phi\in\Gamma$ iff $\M,\Gamma\modelsn\phi$.  The proof is
  by induction on the construction of $\phi$.  Most cases are standard
  \cite{BlaRijVen:ml}; we only address the case $\phi=B_a\psi$.
  \begin{itemize}
  \item $B_a\psi\in\Gamma$ implies $\M,\Gamma\modelsn\phi$.

    Suppose $B_a\psi\in\Gamma$.  It follows by the definition of $N_a$
    that $\Gamma_a(\psi)\in N_a(\Gamma)$.  By the induction
    hypothesis, we have $\Gamma_a(\psi)=[\Gamma]_a\cap \semn{\psi}$
    and hence $[\Gamma]_a\cap\semn{\psi}\in N_a(\Gamma)$.  But then
    $\M,\Gamma\modelsn B_a\psi$.

  \item $\M,\Gamma\modelsn\phi$ implies $B_a\psi\in\Gamma$.

    Suppose $\M,\Gamma\modelsn B_a\psi$.  It follows that
    $[\Gamma]_a\cap\semn{\psi}\in N_a(\Gamma)$.  By the induction
    hypothesis, $[\Gamma]_a\cap\semn{\psi}=\Gamma_a(\psi)$ and hence
    $\Gamma_a(\psi)\in N_a(\Gamma)$.  It follows that there exists
    $B_a\chi\in\Gamma$ such that
    $\Gamma_a(\chi)\subseteq\Gamma_a(\psi)\subseteq[\Gamma]_a$.  Hence
    $K_a(\chi\to\psi)\in\Gamma$ (because the negation of this would
    yield a $\Delta\in[\Gamma]_a$ containing $\chi$ and $\lnot\psi$).
    Since $B_a\chi\in\Gamma$ and $K_a(\chi\to\psi)\in\Gamma$, it
    follows by (M) and maximal consistency that $B_a\psi\in\Gamma$.
  \end{itemize}
  This completes the proof of the Truth Lemma.  Completeness follows
  almost immediately: if $\KB\nvdash\varphi$, then $\{\lnot\phi\}$ is
  consistent and may be extended to a maximal $\KB$-consistent
  $\Gamma\in W$; applying the Truth Lemma, it follows that
  $\M,\Gamma\not\models\phi$, from which we conclude that
  $\not\models_n\phi$.

  We now wish to prove that every $\KB$-formula satisfiable at a
  pointed epistemic neighborhood model is satisfiable in a finite
  pointed epistemic neighborhood model.  For this, it suffices (by
  soundness) for us to fix $\theta\in\Lang_\KB$ satisfying
  $\KB\nvdash\lnot\theta$ and prove that $\theta$ is satisfiable at a
  finite pointed epistemic neighborhood model.  Proceeding, for each
  $\star\in\{K_a,\lnot\}$ and $X\subseteq\Lang_\KB$, we define $\star
  X:=X\cup\{\star\phi\mid\phi\in X\}$.  Also, for each
  $\phi\in\Lang_\KB$, we define ${\sim}\phi\in\Lang_\KB$ by
  \[
  {\sim}\phi :=
  \begin{cases}
    \lnot\phi & \text{if } \phi\neq\lnot\psi, \\
    \phantom{\lnot}\psi & \text{if } \phi=\lnot\psi;
  \end{cases}
  \]
  for each $S\subseteq\Lang_\KB$, we set ${\sim}S:=\{{\sim}\psi\mid
  \psi\in S\}$, and and we let $\mathsf{sub}(\chi)$ be the set of
  subformulas of $\chi$ (including $\chi$ itself).  Define
  \begin{eqnarray*}
    L' &:=&
    \textstyle\bigcup_{a\in A}\lnot
    K_a\lnot(\mathsf{sub}(\theta)\cup
    \mathsf{sub}(B_a\top)\cup
    \mathsf{sub}(\bot))
    \\
    \text{for }S\subseteq L'\text{,}\quad
    \overline S &:=&
    \textstyle
    (\bigwedge_{\phi\in S}\phi)\land
    (\bigwedge_{\psi\in L'-S}{\sim}\psi)
    \\
    \text{for }X\subseteq\pow(L')\text{,}\quad
    \overline X &:=&
    \textstyle
    \bigvee_{S\in X}\overline S
  \end{eqnarray*}
  and
  \begin{eqnarray*}
    L &:=&
    \textstyle
    L' \cup \bigcup_{a\in A}\bigl(\bigcup_{X\subseteq\pow(L')}
    \mathsf{sub}(K_a\lnot B_a\lnot\overline X)\cup
    \mathsf{sub}(\lnot K_a\lnot\overline X)\;\cup
    \\
    &&
    \textstyle
    \bigcup_{m=1}^{|\pow(L')|}
    \bigcup_{X_1,\dots,X_m,Y_1,\dots,Y_m\subseteq\pow(L')}
    \mathsf{sub}(\overline X_1,\dots,\overline X_m\mathbb{E}_a
    \overline Y_1,\dots,\overline Y_m)\bigr)
    \enspace.
  \end{eqnarray*}
  To be clear: $\bigwedge_{\phi\in\emptyset}F_\phi=\top\in L'$ and
  $\bigvee_{\phi\in\emptyset}F_\phi=\bot\in L'$.  Notice that $L'$ and
  $L$ are finite, and $\chi\in L$ implies $\mathsf{sub}(\chi)\subseteq
  L$. A simpler definition of $L$ (i.e., taking $L$ to be $L'$) will
  work for the purposes of the present proof.  Later proofs will need
  the more elaborate version of $L$ given above, and we can save some
  space by dispatching with parts of those proofs in the corresponding
  parts of the present proof.  Nevertheless, the reader may find it
  convenient to pretend on the first read of the present proof that we
  have defined $L$ to be $L'$.  A simple check later will show that
  the argument still works for $L$ as defined.

  Beginning with the canonical model $\M=(W,R,N,V)$, we construct the
  \emph{filtration}
  \[
  \M^*=(W^*,R^*,N^*,V^*)
  \]
  as follows.
  \begin{itemize}
  \item For $\Gamma,\Delta\in W$, write $\Gamma\equiv\Delta$ to mean
    that $\Gamma\cap L=\Delta\cap L$.  This defines an equivalence
    relation on $W$.  We let $\Gamma^*:=\{\Delta\in W\mid
    \Delta\equiv\Gamma\}$ to denote the equivalence class of $\Gamma$
    under $\equiv$.

  \item $W^*:=\{\Gamma^*\mid \Gamma\in W\}$.

  \item $\Gamma^*R^*_a\Delta^*$ means
    \begin{equation}
      \forall K_a\phi\in L:(K_a\phi\in\Gamma \Leftrightarrow
      K_a\phi\in\Delta)\enspace.
      \label{eq:KB-filtration}
    \end{equation}
    Note that $R^*_a$ is well-defined: we have
    \eqref{eq:KB-filtration} if and only if we have the statement
    obtained from \eqref{eq:KB-filtration} by replacing $\Gamma$ with
    an arbitrary $\Gamma'\in\Gamma^*$ and $\Delta$ with an arbitrary
    $\Delta'\in\Delta^*$.  It is easy to see that $R^*_a$ is an
    equivalence relation on $W^*$.  We let
    \[
    [\Gamma^*]_a:=\{\Delta^*\in W^*\mid \Gamma^*R^*_a\Delta^*\}
    \]
    denote the equivalence class of $\Gamma^*$ under $R^*_a$.  We
    observe that for each $\Gamma,\Delta\in W$, we have
    \begin{equation}
      \Delta\in[\Gamma]_a \Rightarrow \Delta^*\in[\Gamma^*]_a\enspace.
      \label{eq:RimpR*}
    \end{equation}
    Indeed, $\Delta\in[\Gamma]_a$ means $\Gamma^{K_a}\subseteq\Delta$.
    Since $K_a$ is an $\mathsf{S5}$ modal operator, it follows by
    maximal consistency that $\Gamma^{K_a}\subseteq\Delta$ is
    equivalent to
    \[
    \forall K_a\chi\in\Lang_\KB: (K_a\chi\in\Gamma\Leftrightarrow
    K_a\chi\in\Gamma)\enspace.
    \]
    But this implies $\Delta^*\in[\Gamma^*]_a$ by the definition of
    $R^*_a$.  So \eqref{eq:RimpR*} indeed holds.
    
  \item $V^*(\Gamma^*):=V(\Gamma)\cap L$.

    Note that $V^*(\Gamma^*)$ is well-defined: $V(\Gamma)\cap
    L=V(\Gamma')\cap L$ for each $\Gamma'\in\Gamma^*$.

  \item Define for each $\Gamma^*\in W^*$, $a\in A$, and
    $\phi\in\Lang_\KB$ the set
    \[
    \textstyle \Gamma^*_a(\phi) :=
    \{\Delta^*\in[\Gamma^*]_a\mid\phi\in(L\cap\Delta)\}\enspace,
    \]
    and let
    \[
    \textstyle N^*_a(\Gamma^*):=\{X\subseteq[\Gamma^*]_a\mid \exists
    B_a\phi\in(L\cap\Gamma):X\supseteq\Gamma^*_a(\phi)\}\enspace.
    \]
    Note that $\Gamma^*_a(\phi)$ and $N^*_a(\Gamma)$ are well-defined:
    a set membership assertion occurring in the above definition of
    either of these sets and having the form $\chi\in(L\cap\Omega)$
    holds if and only if $\chi\in(L\cap\Omega')$ holds for an
    arbitrary $\Omega'\in\Omega^*$.
  \end{itemize}
  The structure $\M^*$ is finite: $W^*$ is finite because members of
  $W^*$ are in one-to-one correspondence with the finitely many
  subsets of the finite set $L$, the set $R^*_a$ is a subset of the
  finite set $W^*\times W^*$, the set $V^*(\Gamma^*)$ is the subset of
  the finite set $L$, and $N^*_a(\Gamma^*)$ is a collection of some of
  the finitely many subsets of the finite set $[\Gamma^*]_a$.

  We verify that $\M^*$ is an epistemic neighborhood model.  $W^*$ is
  nonempty because $\KB\nvdash\lnot\theta$ and therefore
  $\{\lnot\theta\}$ may be extended to a maximal $\KB$-consistent set
  $\Gamma_\theta\in W$, which ensures that $\Gamma_\theta^*\in W^*$.
  It is easy to see that $R^*_a$ is an equivalence relation.  What
  remains is to check that $N^*_a$ satisfies (c), (f), (n), (a), and
  (m).
  \begin{itemize}
  \item For (c), we must show $X\in N^*_a(\Gamma^*)$ implies
    $X\subseteq[\Gamma^*]_a$.  But this follows by the definition of
    $N^*_a(\Gamma^*)$.

  \item For (f), we must show that $\emptyset\notin N^*_a(\Gamma^*)$.
    Assume toward a contradiction that $\emptyset\in N^*_a(\Gamma^*)$.
    It follows that $\exists B_a\phi\in(L\cap\Gamma)$ such that
    $\emptyset\supseteq\Gamma^*_a(\phi)$.  Thus
    \[
    \Gamma^*_a(\phi):=\{\Delta^*\in[\Gamma^*]_a\mid \phi\in
    L\cap\Delta\}=\emptyset \enspace.
    \]
    Since $B_a\phi\in L$, we have $\phi\in L$; it therefore follows by
    \eqref{eq:RimpR*} that
    \[
    \Gamma_a(\phi):=\{\Delta\in[\Gamma]_a\mid \phi\in
    \Delta\}=\emptyset \enspace.
    \]
    It follows from this by maximal consistency that
    $K_a\lnot\phi\in\Gamma$ and hence $K_a(\phi\to\bot)\in\Gamma$. By
    (M) and maximal consistency, we have $B_a\bot\in\Gamma$,
    contradicting the fact that we have $\lnot B_a\bot\in\Gamma$ by
    (F) and maximal consistency.  Conclusion: $\emptyset\notin
    N^*_a(\Gamma^*)$.

  \item For (n), we must show that $[\Gamma^*]_a\in N^*_a(\Gamma^*)$.
    Proceeding, for each $\Delta\in W$, it follows by classical
    reasoning and maximal consistency that $\top\in\Delta$.  Further,
    $\top\in L$.  Therefore,
    \[
    \Gamma^*_a(\top):= \{\Delta^*\in[\Gamma^*]_a\mid \top\in
    L\cap\Delta\}=[\Gamma^*]_a\enspace.
    \]
    Moreover, $B_a\top\in\Gamma$ by (N) and maximal consistency, and
    $B_a\top\in L$.  So since $B_a\top\in L\cap\Gamma$, it follows by
    the definition of $N^*_a$ that $\Gamma^*_a(\top)\in
    N^*_a(\Gamma^*)$.  But then $[\Gamma^*]_a\in N^*_a(\Gamma^*)$, as
    desired.

  \item For (a), we must show that $\Delta^*\in[\Gamma^*]_a$ implies
    $N^*_a(\Delta^*)=N^*_a(\Gamma^*)$.  So take
    $\Delta^*\in[\Gamma^*]_a$. By the definition of $L$, maximal
    consistency, (Ap), the definition of $R^*_a$, and the
    $\mathsf{S5}$ axiom $K_a\phi\to\phi$, we have
    \[
    \renewcommand{\arraystretch}{1.3}
    \begin{array}{lcl}
      B_a\phi\in L\cap\Gamma 
      & \Rightarrow &
      K_aB_a\phi\in L\cap \Gamma
      \\
      & \Rightarrow &
      K_aB_a\phi\in L\cap\Delta
      \\
      & \Rightarrow &
      B_a\phi\in L\cap\Delta\enspace.
    \end{array}
    \]
    By the definition of $L$, maximal consistency, (An), the
    definition of $R_a$, and the same $\mathsf{S5}$ axiom, we also
    have
    \[
    \renewcommand{\arraystretch}{1.3}
    \begin{array}{lcl}
      B_a\phi\in L-\Gamma 
      & \Rightarrow &
      \lnot B_a\phi\in L\cap\Gamma
      \\
      & \Rightarrow &
      K_a\lnot B_a\phi\in L\cap \Gamma
      \\
      & \Rightarrow &
      K_a\lnot B_a\phi\in L\cap\Delta
      \\
      & \Rightarrow &
      \lnot B_a\phi\in L\cap\Delta
      \\
      & \Rightarrow &
      B_a\phi\in L-\Delta
      \enspace.
    \end{array}
    \]
    It follows that for each $B_a\phi\in L$, we have
    $B_a\phi\in\Gamma$ iff $B_a\phi\in\Delta$.  Further, since $R^*_a$
    is an equivalence relation and $\Delta^*\in[\Gamma^*]_a$, we have
    $[\Delta^*]_a=[\Gamma^*]_a$ and
    $\Delta^*_a(\phi)=\Gamma^*_a(\phi)$.  Therefore, we have
    \[
    \renewcommand{\arraystretch}{1.3}
    \begin{array}{lcl}
      X\in N^*_a(\Delta^*)
      & \Leftrightarrow &
      \exists B_a\phi\in L\cap\Delta:
      \Delta^*_a(\phi)\subseteq X\subseteq[\Delta^*]_a
      \\
      & \Leftrightarrow &
      \exists B_a\phi\in L\cap\Gamma:
      \Gamma^*_a(\phi)\subseteq X\subseteq[\Gamma^*]_a
      \\
      & \Leftrightarrow &
      X\in N^*_a(\Gamma^*)
      \enspace.
    \end{array}          
    \]

  \item For (m), we must show $X\in N_a^*(\Gamma^*)$ and $X\subseteq
    Y\subseteq[\Gamma^*]_a$ together imply that $Y\in
    N^*_a(\Gamma^*)$. But for $Y$ satisfying $X\subseteq
    Y\subseteq[\Gamma^*]_a$, we have
    \[
    \renewcommand{\arraystretch}{1.3}
    \begin{array}{lcl}
      X\in N^*_a(\Gamma^*) 
      & \Leftrightarrow &
      \exists B_a\phi\in L\cap\Gamma:
      \Gamma^*_a(\phi)\subseteq X\subseteq[\Gamma^*]_a
      \\
      & \Rightarrow &
      \exists B_a\phi\in L\cap\Gamma:
      \Gamma^*_a(\phi)\subseteq X\subseteq Y\subseteq[\Gamma^*]_a
      \\
      & \Rightarrow &
      Y\in N^*_a(\Gamma^*)
      \enspace.
    \end{array}
    \]
    % XXX;
  \end{itemize}
  So $\M^*$ is indeed an epistemic neighborhood model.  We prove the
  following \emph{Filtration Truth Lemma\/}: for each $\phi\in L$ and
  $\Gamma^*\in W^*$, we have $\phi\in\Gamma$ iff
  $\M^*,\Gamma^*\modelsn\phi$. The proof is by induction on the
  construction of $\phi$.  Most cases are straightforward; we only
  consider the modal cases $\phi=K_a\psi$ and $\phi=B_a\psi$.
  \begin{itemize}
  \item For $K_a\psi\in L$: $K_a\psi\in\Gamma$ iff
    $\M^*,\Gamma^*\modelsn K_a\psi$.

    Suppose $K_a\psi\in L\cap\Gamma$. Take $\Delta^*\in[\Gamma^*]_a$.
    It follows that $K_a\psi\in L\cap\Delta$ by the definition of
    $R^*_a$.  Hence $\psi\in L\cap\Delta$ by the definition of $L$,
    the $\mathsf{S5}$ axiom $K_a\psi\to\psi$, and maximal consistency.
    Applying the induction hypothesis, we have
    $\M,\Delta^*\modelsn\psi$.  Since $\Delta^*\in[\Gamma^*]_a$ was
    chosen arbitrarily, we have shown that $\M,\Gamma^*\modelsn
    K_a\psi$.

    Conversely, suppose $K_a\psi\in L-\Gamma$.  It follows that $\lnot
    K_a\psi\in L\cap\Gamma$ by the definition of $L$ and maximal
    consistency.  We claim that the set
    \[
    S:=\{\lnot\psi\}\cup\{K_a\chi\in L\mid K_a\chi\in\Gamma\}
    \]
    is consistent.  Toward a contradiction, suppose $S$ is not
    consistent.  It follows that there are
    $K_a\chi_1,\ldots,K_a\chi_n\in L\cap\Gamma$ such that
    \[
    \vdash_\KB K_a\chi_1\land\cdots\land K_a\chi_n\to \psi\enspace.
    \]
    It follows by modal reasoning using the $\mathsf{S5}$ operator
    $K_a$ that
    \[
    \vdash_\KB K_a\chi_1\land\cdots\land K_a\chi_n\to K_a\psi\enspace.
    \]
    Hence $K_a\psi\in\Gamma$ by maximal consistency, which contradicts
    the consistency of $\Gamma$ because $\lnot K_a\psi\in
    L\cap\Gamma$.  So $S$ is indeed consistent and can therefore be
    extended to a maximal consistent $\Delta\in W$.  By construction,
    $\Delta^*\in[\Gamma^*]_a$ and $\lnot\psi\in\Delta$. But then
    $\psi\notin\Delta$ by the maximal consistency. By the definition
    of $L$, it follows from $K_a\psi\in L$ that $\psi\in L$, so we may
    apply the induction hypothesis: from $\psi\notin\Delta$, we
    conclude that $\M,\Delta^*\not\modelsn\psi$ and therefore
    $\M,\Delta^*\modelsn\lnot\psi$.  But then $\M,\Gamma^*\not\modelsn
    K_a\psi$.
    
  \item For $B_a\psi\in L$: $B_a\psi\in\Gamma$ iff
    $\M^*,\Gamma^*\modelsn B_a\psi$.

    Suppose $B_a\psi\in L\cap\Gamma$. Then $\Gamma^*_a(\psi)\in
    N^*_a(\Gamma^*)$ by the definition of $N^*_a$.  Since $B_a\psi\in
    L$, it follows by the definition of $L$ that $\psi\in L$, and so
    we have by induction hypothesis that
    $\Gamma^*_a(\psi)=[\Gamma^*]_a\cap\semn{\psi}$.  But then
    $[\Gamma^*]_a\cap\semn{\psi}=\Gamma^*_a(\psi)\in N^*_a(\Gamma^*)$,
    and hence $\M^*,\Gamma^*\modelsn B_a\psi$.

    Conversely, suppose $B_a\psi\in L$ and $\M^*,\Gamma^*\modelsn
    B_a\psi$.  Then $[\Gamma^*]_a\cap\semn{\psi}\in N^*_a(\Gamma^*)$.
    Since $B_a\psi\in L$, it follows by the definition of $L$ that
    $\psi\in L$, and so we have by the induction hypothesis:
    $\Gamma_a^*(\psi)=[\Gamma^*]_a\cap\semn{\psi}\in N^*_a(\Gamma^*)$.
    Applying the definition of $N^*_a$, there exists $B_a\chi\in
    L\cap\Gamma$ such that
    $\Gamma^*_a(\chi)\subseteq\Gamma^*_a(\psi)$.  That is,
    \begin{equation}
      \forall\Delta^*\in[\Gamma^*]_a: \chi\in L\cap\Delta\Rightarrow
      \psi\in L\cap\Delta\enspace.
      \label{eq:chiimppsi}
    \end{equation}
    It follows from \eqref{eq:chiimppsi} and \eqref{eq:RimpR*} that
    \[
    \forall\Delta\in[\Gamma]_a:\chi\in\Delta\Rightarrow\psi\in\Delta\enspace.
    \]
    It therefore follows by maximal consistency that
    $K_a(\chi\to\psi)\in\Gamma$.  Since $B_a\chi\in\Gamma$, we have by
    (M) and maximal consistency that $B_a\psi\in\Gamma$.
  \end{itemize}
  This completes the proof of the Filtration Truth Lemma.  Since
  $\theta\in\Gamma_\theta\cap L$, it follows by the Filtration Truth
  Lemma that $\M^*,\Gamma_\theta^*\modelsn\theta$.  Therefore, the
  satisfiable formula $\theta$ is satisfiable in the finite pointed
  epistemic neighborhood model $(\M^*,\Gamma^*_\theta)$.
\end{proof}

\begin{theorem}[$\KB$ Probability Soundness]
  $\KB$ is sound with respect to the class of epistemic probability
  models:
  \[
  \forall c\in(0,1)\cap\Rat,\forall\phi\in\Lang_\KB: \vdash_\KB\phi
  \text{ implies } {}\modelsp\phi^c \enspace.
  \]
\end{theorem}
\begin{proof}
  By induction on the length of
  derivation. Propositions~\ref{prop:knowledge} and \ref{prop:belief}
  together cover all base cases and induction steps.
\end{proof}

\begin{proposition}[$\KB$ Probability Incompleteness]
  $\KB$ is not complete with respect to the class of epistemic
  probability models:
  \[
  \forall c\in(0,1)\cap\Rat,\exists\phi\in\Lang_\KB: {}\modelsp\phi^c
  \text{ and } \KB\nvdash\phi \enspace.
  \]
\end{proposition}
\begin{proof}
  % XXX
\end{proof}

\subsection{Results for the Low-Threshold Calculus $\KBlt$}

\begin{theorem}[$\KBlt$ Neighborhood Completeness]
  \label{theorem:KBlt-neighborhood-completeness}
  $\KBlt$ is sound and complete with respect to the class $\Clt$ of
  epistemic neighborhood models satisfying (lt) and (l):
  \[
  \forall\phi\in\Lang_\KB: \vdash_\KBlt\phi \text{ iff }
  \Clt\modelsn\phi \enspace.
  \]
  Further, every $\phi\in\Lang_\KB$ that is satisfiable at a pointed
  epistemic neighborhood model in $\Clt$ is satisfiable at a finite
  pointed epistemic neighborhood model in $\Clt$.
\end{theorem}
\begin{proof}
  Soundness is by induction on the length of derivation.  Most cases
  are as in the proof of
  Theorem~\ref{theorem:KB-neighborhood-completeness}.  We only need
  consider the remaining axiom schemes.
  \begin{itemize}
  \item Scheme (LT) is valid: $\modelsn \check B_a\phi\to B_a\phi$.

    Suppose $\M,w\modelsn\check B_a\phi$.  This means
    $[w]_a\cap\semn{\lnot\phi}\notin N_a(w)$, which is equivalent to
    $[w]_a-\semn{\phi}\notin N_a(w)$.  Applying (lt), it follows that
    $\semn{\phi}\in N_a(w)$.  That is, $\M,w\modelsn B_a\phi$.

  \item Scheme (SC) is valid: $\modelsn \check B_a\phi \land \check
    K_a(\lnot\phi\land\psi) \to B_a(\phi\lor\psi)$.

    Suppose $\M,w\modelsn\check B_a\phi$ and $\M,w\modelsn \check
    K_a(\lnot\phi\land\psi)$.  It follows that
    \[
    [w]_a-([w]_a\cap\semn{\phi})=[w]_a\cap\semn{\lnot\phi}\notin
    N_a(w)
    \]
    and that there exists $v\in[w]_a$ satisfying
    $\M,v\models\lnot\phi\land\psi$.  But then
    $[w]_a\cap\semn{\phi\lor\psi}\supsetneq[w]_a\cap\semn{\phi}$ and
    therefore $[w]_a\cap\semn{\phi\lor\psi}\in N_a(w)$ by (sc). Hence
    $\M,w\models B_a(\phi\lor\psi)$.

  \item Scheme (L) is valid:
    \[
    \modelsn \textstyle [(\overline\phi\mathbb{E}_a\overline\psi)
    \land B_a\phi_1 \land \bigwedge_{i=2}^m \check B_a\phi_i] \to
    \bigvee_{i=1}^m B_a\psi_i\enspace.
    \]

    Suppose $(\M,w)$ satisfies the antecedent of scheme (L).  It
    follows that each $v\in[w]_a$ satisfies just as many $\phi_i$'s as
    $\psi_i$'s, that $[w]_a\cap\semn{\psi_1}\in N_a(w)$, and that
    $[w]_a-\semn{\phi_k}\notin N_a(w)$ for each $k\in\{2,\dots,m\}$.
    Hence
    \[
    [w]_a\cap\semn{\phi_1},\dots,[w]_a\cap\semn{\phi_m}\mathbb{E}_a
    [w]_a\cap\semn{\psi_1},\dots, [w]_a\cap\semn{\psi_m}\enspace,
    \]
    from which it follows by (l) that $[w]_a\cap\semn{\psi_j}\in
    N_a(w)$ for some $j\in\{1,\dots,m\}$.  Hence $\M,w\modelsn B_a\psi_j$,
    and thus $\M,w\modelsn\bigvee_{i=1}^m B_a\psi_i$.
  \end{itemize}
  Soundness has been proved.  Completeness goes by way of a filtration
  of the canonical model.  The construction of the canonical model
  $\M=(W,R,N,V)$ is the same as in the proof of
  Theorem~\ref{theorem:KB-neighborhood-completeness}, except that
  maximal consistency is taken with respect to $\KBlt$.  We fix
  $\theta\in\Lang_\KB$ such that $\nvdash_\KBlt\lnot\theta$ and define
  the finite set $L'$, the conjunction $\bar S$ for each $S\subseteq
  L'$, the disjunction $\overline X$ for each $X\subseteq\pow(L')$,
  the finite set $L$, and the filtration $\M^*=(W^*,R^*,N^*,V^*)$ as
  in the proof of Theorem~\ref{theorem:KB-neighborhood-completeness}.
  As was argued there, the filtration $\M^*$ is finite.

  We wish to prove that every $X\subseteq[\Gamma^*]_a$ is
  $L$-definable: we show that the expression
  \[
  \hat X := \bigvee_{\Delta^*\in X}\overline{L'\cap\Delta}
  \]
  is a $\Lang_\KB$-formula in $L$ satisfying $\Gamma^*_a(\hat X)=X$.
  We prove this in a few steps.
  \begin{itemize}
  \item $\hat X$ is well-defined.

    Replacing $\Delta$ in the definition of $\hat X$ by an arbitrary
    $\Delta'\in\Delta^*$ results in the same disjunction.

  \item $\hat X\in L$.

    The expression $\hat X$ is a $\Lang_\KB$-formula because
    $L'\subseteq\Lang_\KB$ is finite. Further, if we define
    $Y:=\{L'\cap\Delta\mid \Delta^*\in X\}$ (and note that this is
    well-defined), then $Y\subseteq\pow(L')$ and $\hat X=\overline
    Y\in L$ by definition of $L$.

  \item $\forall\Delta^*,\Omega^*\in[\Gamma^*]_a:
    (L'\cap\Delta=L'\cap\Omega) \Leftrightarrow
    (L\cap\Delta=L\cap\Omega)$.

    The right-to-left direction is obvious (since $L'\subsetneq L$),
    so let us focus on the left-to-right direction.  Proceeding,
    assume $L'\cap\Delta=L'\cap\Omega$. Since ${\sim}L'=L'$ by the
    definition of $L'$, we have for each $S\subseteq L'$ that
    $S\cap\Delta=S\cap\Omega$ and
    ${\sim}(L'-S)\cap\Delta={\sim}(L'-S)\cap\Omega$. By maximal
    consistency and the closure of $L'$ under subformulas, taking $Q$
    to be the conjunction $\overline S$, we have
    \begin{equation}
      \forall\chi\in\mathsf{sub}(Q):\chi\in\Delta \Leftrightarrow
      \chi\in\Omega\enspace.
      \label{eq:L'impL}
    \end{equation}
    It follows by maximal consistency that for each
    $Y\subseteq\pow(L')$, taking $Q$ to be the disjunction $\overline
    Y$, we have \eqref{eq:L'impL}.  Hence taking $Y\subseteq\pow(L')$
    and $Q=\lnot\overline Y$, we have \eqref{eq:L'impL}.

    We note that $\Delta^*,\Omega^*\in[\Gamma^*]_a$ means that
    $\Delta$ and $\Omega$ ``agree on'' (i.e., contain exactly the
    same) knowledge formulas $K_a\chi\in L$.  We will refer to this
    knowledge-formula agreement by writing simply that
    $\Delta^*,\Omega^*\in[\Gamma^*]_a$.

    Now for a $\Lang_\KB$-formula $\chi$, we let ${\pm}\chi$ indicate
    that we mean what is said both for $\chi$ and for $\lnot\chi$.  So
    taking $Y\subseteq\pow(L')$ and $Q={\pm}K_a{\pm}\overline Y$, we
    have \eqref{eq:L'impL} by maximal consistency and the fact that
    $\Delta^*,\Omega^*\in[\Gamma^*]_a$.
    
    Further, again taking $Y\subseteq\pow(L')$, we have by maximal
    consistency, the definition of $L$, the $\mathsf{S5}$ axioms for
    $K_a$, and the fact that $\Delta^*,\Omega^*\in[\Gamma^*]_a$ that
    \begin{eqnarray*}
      B_a{\pm}\overline Y\in L\cap\Delta
      & \Rightarrow &
      K_aB_a{\pm}\overline Y\in L\cap\Delta
      \\
      & \Rightarrow &
      K_aB_a{\pm}\overline Y\in L\cap\Omega
      \\
      & \Rightarrow &
      B_a{\pm}\overline Y\in L\cap\Omega
    \end{eqnarray*}
    and that
    \begin{eqnarray*}
      B_a{\pm}\overline Y\in L-\Delta
      & \Rightarrow &
      \lnot B_a{\pm}\overline Y\in L\cap\Delta
      \\
      & \Rightarrow &
      K_a\lnot B_a{\pm}\overline Y\in L\cap\Delta
      \\
      & \Rightarrow &
      K_a\lnot B_a{\pm}\overline Y\in L\cap\Omega
      \\
      & \Rightarrow &
      \lnot B_a{\pm}\overline Y\in L\cap\Omega
      \\
      & \Rightarrow &
      B_a{\pm}\overline Y\in L-\Omega
    \end{eqnarray*}
    Hence $B_a({\pm}\overline Y)\in L\cap\Delta$ iff
    $B_a({\pm}\overline Y)\in L\cap\Omega$.  By maximal consistency
    and the definition of $L$, we have $\lnot B_a({\pm}\overline Y)\in
    L\cap\Delta$ iff $\lnot B_a({\pm}\overline Y)\in L\cap\Omega$.
    Finally, we have $K_a{\pm} B_a({\pm}\overline Y)\in L\cap\Delta$
    iff $K_a{\pm} B_a({\pm}\overline Y)\in L\cap\Omega$ by the fact
    that $\Delta^*,\Omega^*\in[\Gamma^*]_a$.  It follows that, taking
    $Y\subseteq\pow(L')$ and $Q=K_a{\pm} B_a({\pm}\overline Y)$, we
    have \eqref{eq:L'impL}.  

    Recalling the definition of the Segerberg formula
    $\overline\phi\mathbb{E}_a\overline\psi$
    (Definition~\ref{definition:segerberg-notation}), we fix a
    positive integer $m\leq|\pow(L')|$ and subsets $X_i$ and $Y_i$ of
    $[\Gamma^*]_a$ for each $i\in\{1,\dots,m\}$.  For each
    $k\in\{0,\dots,m\}$, let $C_k$ denote the conjunction
    \[
    d_1\overline X_1\land\cdots\land d_m\overline X_m
    \land
    e_1\overline Y_1\land\cdots\land e_m\overline Y_m
    \]
    having exactly $k$ of the $d_i$'s the empty string, exactly $k$ of
    the $e_i$'s the empty string, and the rest of the $d_i$'s and
    $e_i$'s the negation sign $\lnot$.  We observe that if we take $Q$
    to be one such $C_k$, we have by what we have shown so far and
    maximal consistency that \eqref{eq:L'impL}.  Hence taking $Q$ to
    be the disjunction $\bigvee_{k=0}^m C_k$, we also have
    \eqref{eq:L'impL} by maximal consistency.  Since
    $\Delta^*,\Omega^*\in[\Gamma^*]_a$, it follows that if we take $Q$
    to be the Segerberg formula
    \begin{equation}
      \overline X_1,\dots,\overline X_m\mathbb{E}_a\overline
      Y_1,\dots,\overline Y_m\enspace,
      \label{eq:L'impL-seger}
    \end{equation}
    then we have \eqref{eq:L'impL}.  Since we assumed
    $L'\cap\Delta=L'\cap\Omega$ and $L$ is defined as the extension of
    $L'$ obtained by adding the subformulas of $K_a{\pm}
    B_a({\pm}\overline Y)$, ${\pm}K_a{\pm}\overline Y$, and
    \eqref{eq:L'impL-seger} for each $a\in A$ and for subsets $Y$,
    $X_i$, and $Y_i$ of $[\Gamma^*]_a$, it follows that
    $L\cap\Delta=L\cap\Omega$.

  \item $\forall X\subseteq[\Gamma^*]_a:\Gamma^*_a(\hat X)=X$.
    \begin{eqnarray}
      && \Delta^*\in\Gamma^*_a(\hat X)
      \label{eq:hatX1}
      \\
      &\Leftrightarrow &
      \Delta^*\in[\Gamma^*]_a \land
      \hat X\in L\cap\Delta
      \label{eq:hatX2}
      \\
      &\Leftrightarrow &
      \Delta^*\in[\Gamma^*]_a \land
      \exists\Omega^*\in X:
      \overline{L'\cap\Omega}\in L\cap\Delta
      \label{eq:hatX3}
      \\
      &\Leftrightarrow &
      \Delta^*\in[\Gamma^*]_a \land
      \exists\Omega^*\in X:
      L'\cap\Omega=L'\cap\Delta
      \label{eq:hatX4}
      \\
      &\Leftrightarrow &
      \Delta^*\in[\Gamma^*]_a \land
      \exists\Omega^*\in X:
      L\cap\Omega=L\cap\Delta
      \label{eq:hatX5}      
      \\
      &\Leftrightarrow &
      \Delta^*\in[\Gamma^*]_a \land
      \exists\Omega^*\in X:
      \Omega^*=\Delta^*
      \label{eq:hatX6}
      \\
      &\Leftrightarrow &
      \Delta^*\in X
      \label{eq:hatX7}
    \end{eqnarray}
    We have $\eqref{eq:hatX1}\Leftrightarrow\eqref{eq:hatX2}$ by the
    definition of $\Gamma^*_a$.  The equivalence
    $\eqref{eq:hatX2}\Leftrightarrow\eqref{eq:hatX3}$ follows by
    maximal consistency, the closure of $L$ under subformulas, and the
    definition of the disjunction $\hat X$.  We have
    $\eqref{eq:hatX3}\Leftrightarrow\eqref{eq:hatX4}$ by maximal
    consistency, the fact that $L'\subseteq L$ and $L$ is closed under
    subformulas, and the definition of the conjunction
    $\overline{L'\cap\Omega}$.  The equivalence
    $\eqref{eq:hatX4}\Leftrightarrow\eqref{eq:hatX5}$ was proven in
    the previous bullet point (and makes use of the assumption
    $X\subseteq[\Gamma^*]_a$).  We have
    $\eqref{eq:hatX5}\Leftrightarrow\eqref{eq:hatX6}$ by the meaning
    of the equivalence relation $\equiv$ and the equivalence classes
    $\Omega^*$ and $\Delta^*$, and we have
    $\eqref{eq:hatX6}\Leftrightarrow\eqref{eq:hatX7}$ by the same
    equivalence relation and equivalence classes and the fact that
    $X\subseteq[\Gamma^*]_a$.
  \end{itemize}
  So we have proved that each $X\subseteq[\Gamma^*]_a$ is definable by
  $\hat X\in L$ in the sense that $\Gamma^*_a(\hat X)=X$.  We note
  that $\Gamma^*_a(\lnot\hat X)=[\Gamma^*]_a-X$ for each
  $X\subseteq[\Gamma^*]_a$.

  We verify that $\M^*$ is an epistemic neighborhood model satisfying
  (lt), (sc), and (l).  The properties of epistemic neighborhood
  models other than (lt), (sc), and (l) are checked just as in the
  proof of Theorem~\ref{theorem:KB-neighborhood-completeness}.  So all
  that we must check is that $\M^*$ satisfies these three additional
  properties.
  \begin{itemize}
  \item For (lt), we must show $[\Gamma^*]_a-X\notin N^*_a(\Gamma^*)$
    implies $X\in N^*_a(\Gamma^*)$. So suppose
    $[\Gamma^*]_a-X\notin N^*_a(\Gamma^*)$.  Then
    \[
    \Gamma^*_a(\lnot\hat X)=[\Gamma^*]_a-X\notin
    N^*_a(\Gamma^*)\enspace.
    \]
    Hence $B_a\lnot\hat X\in L-\Gamma$, from which we have $\check
    B_a\hat X=\lnot B_a\lnot\hat X\in L\cap\Gamma$ by maximal
    consistency and the definition of $L$.  Making use of (LT),
    maximal consistency, and the definition of $L$, it follows that
    $B_a\hat X\in L\cap\Gamma$.  Hence $X=\Gamma^*_a(\hat X)\in
    N^*_a(\Gamma^*)$.

  \item For (sc), we must show $[\Gamma^*]_a-X\notin N^*_a(\Gamma^*)$ and
    $X\subsetneq Y\subseteq[\Gamma^*]_a$ together imply $Y\in
    N^*_a(\Gamma^*)$.  Suppose $[\Gamma^*]_a-X\notin N^*_a(\Gamma^*)$ and
    $X\subsetneq Y\subseteq[\Gamma^*]_a$. Then
    \[
    \Gamma^*_a(\lnot\hat X)=[\Gamma^*]_a-X\notin
    N^*_a(\Gamma^*)\enspace.
    \]
    Hence $B_a\lnot\hat X\in L-\Gamma$, from which we have $\check
    B_a\hat X=\lnot B_a\lnot\hat X\in L\cap\Gamma$ by maximal
    consistency and the definition of $L$. Now $X\subsetneq
    Y\subseteq[\Gamma^*]_a$ implies $Y=X\cup Z$ for some nonempty
    $Z\subseteq[\Gamma^*]_a$.  It follows by maximal consistency and
    the definition of $L$ that $\check K_a\hat Z\in L\cap\Gamma$.
    Making use of (SC), maximal consistency, and the definition of
    $L$, it follows that $B_a(\hat X\lor\hat Z)=B_a(\hat Y)\in
    L\cap\Gamma$.  Hence $Y=\Gamma^*_a(\hat Y)\in N^*_a(\Gamma^*)$.

    \XXXcomment{The proof of property (l) needs to be checked
      carefully.\hfill--br}

  \item For (l), we must show $\forall
    X_1,\dots,X_m,Y_1,\dots,Y_m\subseteq[\Gamma^*]_a$, if we have
    \begin{eqnarray}
      &&
      X_1,\dots,X_m\mathbb{E}_aY_1,\dots,Y_m 
      \label{eq:KBlt:lE}
      \\
      &&
      X_1\in N^*_a(\Gamma^*) \enspace
      \label{eq:KBlt:lX1}            
      \\
      &&
      \forall i\in\{2,\dots,m\}:[\Gamma^*]_a-X_i\notin
      N^*_a(\Gamma^*)
      \label{eq:KBlt:lXis}
    \end{eqnarray}
    then $Y_j\in N^*_a(\Gamma^*)$ for some $j\in\{1,\dots,m\}$.  So,
    choosing subsets $X_i$ and $Y_i$ of $[\Gamma^*]_a$ for
    $i\in\{1,\dots,m\}$, suppose we have \eqref{eq:KBlt:lE},
    \eqref{eq:KBlt:lX1}, and \eqref{eq:KBlt:lXis}. We note that there
    can be at most $|\pow(L')|$ pairwise distinct $X_i$'s, and a
    similar comment holds for the $Y_i$'s.  (This is so because we
    have showed that $\Delta\cap L'=\Omega\cap L'$ implies $\Delta\cap
    L=\Omega\cap L$ for each $\Delta,\Omega\in W$.)  Define
    $n:=\min\{|\pow(L')|,m\}$.  We renumber the $X_i$'s so as to
    ensure that no $X_j$ is repeated before an $X_k$ that has yet to
    appear.  We renumber the $Y_i$'s similarly. Let $\Delta^*_{X,k}$
    be the set of positive-integer indicies $i\leq\min\{k,m\}$ such
    that $X_i$ contains $\Delta^*$.  Define $\Delta^*_{Y,k}$ similarly
    with respect to the $Y_i$'s.  If
    $\Delta^*_{X,n}\neq\Delta^*_{X,m}$, then the $X_i$'s with indicies
    $i\in \Delta^*_{X,m}-\Delta^*_{X,n}$ are all repetitions. A
    similar comment goes for indicies $i\in
    \Delta^*_{Y,m}-\Delta^*_{Y,n}$.  Our assumption \eqref{eq:KBlt:lE}
    is equivalent to $|\Delta^*_{X,m}|=|\Delta^*_{Y,m}|$.  But then it
    follows that $|\Delta^*_{X,m-k}|=|\Delta^*_{Y,m-k}|$ for
    $k=0,\dots,m-n$ since increasing $k$ from $0$ to $m-n$ removes
    from consideration the index of a repeated $X_i$ and the index of
    a repeated $Y_i$, and so the size of the resulting sets remains
    the same.  It follows that we have \eqref{eq:KBlt:lE} with $m$
    replaced by $n$.  Applying the definition of the Segerberg
    notation (Definition~\ref{definition:segerberg-notation}) to the
    formula
    \begin{equation}
      \hat X_1,\dots,\hat X_n \mathbb{E}_a \hat Y_1,\dots,\hat Y_n \enspace,
      \label{eq:KBlt:Seger}
    \end{equation}
    it follows by maximal consistency and the definition of $L$ that
    each $\Delta^*\in W^*$ satisfies $C_0\lor\cdots\lor C_n\in
    L\cap\Delta$.  Hence \eqref{eq:KBlt:Seger} is contained in
    $L\cap\Gamma$, for otherwise we would have $\Omega\in[\Gamma]_a$
    containing $\lnot(C_0\lor\cdots\lor C_n)$, which would imply
    $C_0\lor\cdots\lor C_n\notin L\cap\Omega$ by maximal consistency,
    a contradiction.  Further, \eqref{eq:KBlt:lX1} implies $B_a\hat
    X_1\in L\cap\Gamma$.  And by an argument as in the verification of
    property (sc), it follows from \eqref{eq:KBlt:lXis} that $\check
    B_a\hat X_i\in L\cap\Gamma$ for each $i\in\{2,\dots,n\}$.  But
    then it follows by (LT) and maximal consistency that $B_a\hat
    X_j\in\Gamma$ for some $j\in\{1,\dots,n\}$.  Applying the
    definition of $L$, we have $B_a\hat X_j\in L\cap\Gamma$ and
    therefore $X_j=\Gamma^*_a(\hat X_j)\in N^*_a(\Gamma^*)$.
  \end{itemize}
  So $\M$ is indeed an epistemic neighborhood model satisfying (lt),
  (sc), and (l).  The Filtration Truth Lemma is proved as in the proof
  of Theorem~\ref{theorem:KB-neighborhood-completeness}, so it follows
  by similar reasoning as in that proof that $\theta$ is satisfiable
  in a finite pointed epistemic neighborhood model satisfying (lt),
  (sc), and (l).
\end{proof}

\XXXcomment{Not sure about $\KBlt$ probability completeness. --br}


\begin{theorem}[$\KBlt$ Probability Completeness]
  $\KBlt$ is sound and complete with respect to the class of epistemic
  probability models with threshold $c\in(0,\frac 12)$:
  \[
  \textstyle \forall c\in(0,\frac 12)
  \cap\Rat,\forall\phi\in\Lang_\KB: \vdash_\KBlt\phi \text{ iff }
  {}\modelsp\phi^c \enspace.
  \]
\end{theorem}
\begin{proof}
  % XXX
\end{proof}

\subsection{Results for the $0.5$-Threshold Calculus $\KBeq$}

\begin{theorem}[$\KBeq$ Neighborhood Completeness]
  $\KBeq$ is sound and complete with respect to the class $\Ceq$ of
  epistemic neighborhood models satisfying (sc) and (l):
  \[
  \forall\phi\in\Lang_\KB: \vdash_\KBeq\phi \text{ iff }
  \Ceq\modelsn\phi \enspace.
  \]
  Further, every $\phi\in\Lang_\KB$ that is satisfiable at a pointed
  epistemic neighborhood model in $\Ceq$ is satisfiable at a finite
  pointed epistemic neighborhood model in $\Ceq$.
\end{theorem}
\begin{proof}
  % XXX
\end{proof}

\XXXcomment{Not sure about $\KBeq$ probability completeness. --br}

\begin{theorem}[$\KBeq$ Probability Completeness]
  $\KBeq$ is sound and complete with respect to the class of epistemic
  probability models with threshold $c=\frac 12$:
  \[
  \forall\phi\in\Lang_\KBlt: \vdash_\KBeq\phi \text{ iff }
  {}\modelsp\phi^{1/2} \enspace.
  \]
\end{theorem}
\begin{proof}
  % XXX
\end{proof}

\subsection{Results for the High-Threshold Calculus $\KBgt$}

\begin{theorem}[$\KBgt$ Neighborhood Completeness]
  $\KBgt$ is sound and complete with respect to the class $\Cgt$ of
  epistemic neighborhood models satisfying (sc) and (wl):
  \[
  \forall\phi\in\Lang_\KB: \vdash_\KBgt\phi \text{ iff }
  \Cgt\modelsn\phi \enspace.
  \]
  Further, every $\phi\in\Lang_\KB$ that is satisfiable at a pointed
  epistemic neighborhood model in $\Clt$ is satisfiable at a finite
  pointed epistemic neighborhood model in $\Clt$.
\end{theorem}
\begin{proof}
  % XXX
\end{proof}

\XXXcomment{Not sure about $\KBgt$ probability completeness.\hfill--br}

\begin{theorem}[$\KBgt$ Probability Completeness]
  $\KBgt$ is sound and complete with respect to the class of epistemic
  probability models with threshold $c\in(\frac 12,1)$:
  \[
  \textstyle \forall c\in(\frac 12,1)
  \cap\Rat,\forall\phi\in\Lang_\KB: \vdash_\KBgt\phi \text{ iff }
  {}\modelsp\phi^c \enspace.
  \]
\end{theorem}
\begin{proof}
  % XXX
\end{proof}

\section{Three Calculi for Belief as Willingness to Bet} 
\label{Section:TreeCalculi}

\XXXcomment{Old version of this section (written by Jan). Note that
  this section still tacitly assumes that $c=\frac 12$. \hfill{}--br}

The first calculus we will study is given by the following definition. 

\begin{definition} Assume $\phi \in \Lang_\KB$.  Then $\vdash_{\text{KB}} \phi$
  expresses that $\phi$ is derivable with the following axioms and
  rules:
\begin{eqnarray*}
\text{(Prop)} & & \text{All propositional tautologies} \\
\text{(K)} & & \vdash_{\text{KB}} K_a(\phi \rightarrow \psi) \rightarrow K_a \phi \rightarrow K_a\psi \\
\text{(KT)} & & \vdash_{\text{KB}} K_a \phi \rightarrow \phi \\
\text{(K4)} & & \vdash_{\text{KB}} K_a \phi \rightarrow  K_a  K_a \phi \\
\text{(K5)} & & \vdash_{\text{KB}} \check{K}_a \phi \rightarrow  K_a  \check{K}_a \phi \\
\text{(C)} & & \vdash_{\text{KB}} K_a \phi \rightarrow B_a \phi \\
\text{(D)} & &  \vdash_{\text{KB}} B_a \phi \rightarrow \check{B}_a \phi \\
\text{(Mix-4)} & & \vdash_{\text{KB}} B_a \phi \rightarrow K_a B_a \phi \\
\text{(Mix-5)} & & \vdash_{\text{KB}} \check{B}_a \phi \rightarrow K_a \check{B}_a \phi \\
\text{(N)} & & \vdash_{\text{KB}} B_a \top \\
\text{(M)} & &  \vdash_{\text{KB}}  K_a (\phi \rightarrow \psi) \rightarrow B_a \phi 
   \rightarrow B_a \psi  \\
\text{(MP)}  & &  \text{From } \vdash_{\text{KB}} \phi \text{ and } \vdash_{\text{KB}} \phi \rightarrow \psi 
   \text{ derive } \vdash_{\text{KB}} \psi \\ 
\text{(KNec)}  & & \text{From } \vdash_{\text{KB}} \phi 
        \text{ derive } \vdash_{\text{KB}} K_a \phi 
\end{eqnarray*} 
\end{definition}

Here are some derivable principles and rules: 

\begin{proposition} The following are derivable: 
\begin{eqnarray*}
 & & \vdash_{\text{KB}} B_a \phi \rightarrow B_a B_a \phi \\
 & & \vdash_{\text{KB}} \check{B}_a \phi \rightarrow B_a \check{B}_a \phi \\
            & & \vdash_{\text{KB}} \neg B_a \bot \\
            & & \vdash_{\text{KB}} K_a (\phi \rightarrow \psi) \rightarrow \check{B}_a \phi
                 \rightarrow \check{B}_a \psi. \\
            & & \vdash_{\text{KB}} K_a \phi \land B_a \psi \rightarrow B_a (\phi \land \psi). \\
            & & \vdash_{\text{KB}} B_a \phi \leftrightarrow K_a B_a \phi. \\
            & & \vdash_{\text{KB}} \check{B}_a \phi \leftrightarrow K_a \check{B}_a \phi. \\
  & & \text{From } \vdash_{\text{KB}} \phi 
        \text{ derive } \vdash_{\text{KB}} B_a \phi. \\
  & & \text{From } \vdash_{\text{KB}} \phi \rightarrow \psi 
        \text{ derive } \vdash_{\text{KB}} B_a \phi \rightarrow B_a \psi. \\
  & & \text{From } \vdash_{\text{KB}} \phi \leftrightarrow \psi 
        \text{ derive } \vdash_{\text{KB}} B_a \phi \leftrightarrow B_a \psi. \\
  & & \text{From } \vdash_{\text{KB}} \phi \rightarrow \psi 
        \text{ derive } \vdash_{\text{KB}} \check{B}_a \phi \rightarrow \check{B}_a \psi. \\
  & & \text{From } \vdash_{\text{KB}} \phi \leftrightarrow \psi 
        \text{ derive } \vdash_{\text{KB}} \check{B}_a \phi \leftrightarrow \check{B}_a \psi. 
\end{eqnarray*} 
\end{proposition} 
\begin{proof}
Left to the reader. 
\end{proof}

\begin{theorem} \label{SoundEMM}
The KB calculus is sound for the class of epistemic neighborhood models: 
if $\vdash_{\text{KB}} \phi$ then $\phi$ is true in every world $w$ in every epistemic neighborhood model $\M$. 
\end{theorem}
\begin{proof}
  Use induction on the structure of a derivation in the KB calculus. In
  the base case, check the axioms for soundness. In the induction
  step, check that the rules preserve soundness.
\end{proof}

\begin{theorem}
The KB calculus is sound for the class of epistemic probability models, using 
the definitions of $K_a$ as certainty and $B_a$ as betting belief. 
\end{theorem} 
\begin{proof}
Immediate from Theorem \ref{SoundEMM} plus the Betting and Certainty Theorem. 
\end{proof}

\begin{theorem} \label{completeEMM}
The KB calculus is complete for the class of epistemic neighborhood models.
\end{theorem} 
\begin{proof}
Assume $\not\vdash_{\text{KB}} \phi$. Use a canonical model
construction to construct an epistemic neighborhood model $\M$ and a world
$w$ in $\M$ with $\M, w \not\models \phi$.
\XXXcomment{We should spell out the details, for these cannot be gleaned from 
the Chellas book.}
\end{proof}

Still, this logic is very weak, as the following example shows. 

\begin{example} \label{EMMweakExample}
Let $\M$ be the (single agent) epistemic neighborhood model given by $\M =
(W,R,N,V)$ where $W = \{ x, y, z \}$, $R = W \times W$, $\Prop = \{ x,
y , z \}$, $N = \lambda w \mapsto \{ \{ x, y, z \} \}$, and $V$ is
given by $V(w) = \{ w \}$.  This satisfies all epistemic neighborhood model
requirements.
\end{example}

\begin{theorem} \label{incompleteEPM}
The KB calculus is incomplete for the class of epistemic probability models.
\end{theorem}
\begin{proof}
Consider the model $\M$ from Example \ref{EMMweakExample}, with 
world $x$. 
Let $\sigma$ be a description of $\M,x$ in the language $\Lang_{KL}$
over $\Prop$. From $\M \not\models \neg \sigma$ and the soundness of
the KB calculus we get $\not\vdash_{\text{KB}} \neg \sigma$.  Now
suppose $\N$ is a (single agent) epistemic probability model $\N$ for
$\sigma$.

Then in $\N$, $Px + Py + Pz = 1$ must hold, together with 
$Px + Py \leq \frac 1 2$ (for $\{ x,y\}$ is not a neighborhood in 
$\M$) and $Py + Pz \leq \frac 1 2$ (for $\{ y, z\}$ is not a 
neighborhood in $\M$). Therefore $Px + 2Py + Pz \leq 1$. 
Since $Py > 0$ it follows that $Px + 2Py + Pz < 1$, and contradiction. 
This shows that in the class of 
epistemic probability models $\models \neg \sigma$. 

But as we have seen $\not\vdash_{\text{KB}} \neg \sigma$. 
So $\neg \sigma$ is a formula that is valid on the class
of epistemic probability models but does not have a proof
in the KB calculus. 
\end{proof}

The following theorem (adapted from \cite{WalleyFine1979:vomacp})
shows that the requirements for being an epistemic neighborhood model 
are very weak: 

\begin{theorem} \label{pneiModel}
  Let $W$ be a non-empty set of worlds, and let ${\cal X}$ be a family
  of subsets of $W$ with the pairwise non-empty intersection property,
  i.e., the property that if $X,X' \in {\cal X}$ then $X \cap X' \neq
  \emptyset$. Then $\M$ given by $(W,R,N)$ where $R$ is $W\times W$,
  and $N$ is the function that assigns to any $w \in W$ the set
\[
 \uparrow {\cal X} = 
 \{ Y \mid \exists X \in {\cal X} \text{ with } X \subseteq Y \}
\]
is an epistemic neighborhood frame (for a single agent). 
\end{theorem}
\begin{proof}
The only condition we have to check is (d). Let $Y \in \ \uparrow {\cal X}$. 
Then there is an $X \in {\cal X}$ with $X \subseteq Y$. 
Suppose $W-Y \in \ \uparrow {\cal X}$. Then there is an $X' \in {\cal X}$ 
with $X' \subseteq W - Y$. It follows that $X \cap X' = \emptyset$, and 
contradiction with the pairwise non-empty intersection property of ${\cal X}$. 
\end{proof}

The following principle fails in the model of Example \ref{EMMweakExample}. 
\begin{eqnarray*}
\text{(Commit)} & & \vdash_{\text{KB}} \check{B}_a \phi \rightarrow
  \check{K}_a(\neg \phi \land \psi) \rightarrow B_a (\phi \lor \psi) 
\end{eqnarray*}
What this principle expresses is that if an agent is not willing to bet 
against $\phi$ and the agent considers $\neg \phi \land \psi$ possible, 
then the agent should be willing to bet on $\phi \lor \psi$. 

Here is the corresponding requirement on models: 

\begin{definition} \label{commitment}
An epistemic neighborhood model $\M = (W,R,N,V)$ is {\em committed} if 
it holds for all agents $a$, all $w \in W$ that $W - X \notin N_a(w)$ 
and $X \subsetneq Y$ together imply $Y \in N_a(W)$. 
\end{definition}

\begin{theorem} \label{completeCommit}
The calculus KB + Commit is sound and complete for the class of 
committed epistemic neighborhood models. 
\end{theorem} 
\begin{proof}
\ldots
\end{proof}

The following theorem gives us a recipe for constructing committed
epistemic neighborhood models. 

\begin{theorem}\label{pneiCommitModel}
  Let $W$ be a non-empty set of worlds, and let ${\cal X}$ be a family
  of subsets of $W$ with the pairwise non-empty intersection property,
  i.e., the property that if $X,X' \in {\cal X}$ then $X \cap
  X' \neq \emptyset$, and the following maximality property: If
  $Y \notin {\cal X}$ then $\exists X \in {\cal X}$ with $Y \cap X
  = \emptyset$.  Then $\M$ given by $(W,R,N)$ where $R$ is $W\times
  W$, and $N$ is the function that assigns to any $w \in W$ the set
  ${\cal X}$ is a committed epistemic neighborhood frame (for a
  single agent).
\end{theorem}
\begin{proof}
The three conditions we have to check are (m), (d) and the commitment property. 

We first show that ${\cal X}$ is upward closed. Suppose $X \in {\cal X}$, 
Then $X$ has a non-empty intersection with all members of ${\cal X}$. 
Let $X \subseteq Y$. Then $Y$ has a non-empty intersection with all members of ${\cal X}$,
so by the maximality property of ${\cal X}$, $Y \in {\cal X}$. This takes care of (m). 

Let $U \in {\cal X}$. Then $(W - U) \cap U = \emptyset$, so $W-U \notin
{\cal X}$, by the fact that ${\cal X}$ has the pairwise non-empty
intersection property. This takes care of (d).

Finally, let $W - U \notin  {\cal X}$. Then, by the maximality property of 
${\cal X}$, there is some $X \in {\cal X}$ with $X \cap (W-U) = \emptyset$. 
It follows that $X \subseteq U$. But since $X$ has a non-empty intersection with 
each member of ${\cal X}$, the same holds for $U$. By the maximality property 
of ${\cal X}$, $U \in {\cal X}$. So the frame is committed. 
\end{proof}

We will now use Theorem \ref{pneiCommitModel} to show incompleteness
of the KB + Commit calculus for the class of epistemic probability
models, by exhibiting a committed epistemic neighborhood model $\M$
that cannot be extended to an epistemic probability model.

The following example model is inspired by
\cite{WalleyFine1979:vomacp} (see also \cite{Herzig2003:mpbaa}).

\begin{example} \label{WFmodel}
We define an epistemic neighborhood model $\M = (W,R,N,V)$, 
as follows. 
Let $W = \Prop = \{ a,b,c,d,e,f,g \}$. Let $R = W \times W$. 
Let $N$ be the function that assigns to every $w \in W$ the
set ${\cal X}^\bullet$, where 
\[
 {\cal X} = 
 \{ 
\{ e,f,g \}, 
\{ a,b,g \},
\{ a,d,f \}, 
\{ b,d,e \}, 
\{ a,c,e \},
\{ c,d,g \}, 
\{ b,c,f \} \}
\]
and 
\[
 {\cal X}^\bullet = \{ Y \subseteq W \mid \forall Z \in {\cal X}: 
 Y \cap Z \neq \emptyset \}.
\]
Let $V$ be given by $V(w) = \{ w \}$.  

Notice that since ${\cal X}$ has the pairwise nonempty intersection
property, ${\cal X}^\bullet$ has this property as well. ${\cal
X}^\bullet$ is also a maximal set with this property, for let
$U \notin {\cal X}^\bullet$. Then by the definition of ${\cal
X}^\bullet$, there is some $Z \in {\cal X}$ with $U \cap Z
= \emptyset$. Since ${\cal X}^\bullet$ has the pairwise non-empty
intersection property, plus the maximality property, by
Theorem \ref{pneiCommitModel}, $\M$ is indeed a committed epistemic
neighborhood model.
\end{example}

\begin{theorem} \label{incompleteCEPM}
The KB calculus with Commit is incomplete for the class of epistemic probability models.
\end{theorem}
\begin{proof}
Consider the model $\M$ from Example \ref{WFmodel}, with world $a$. 
Let $\sigma$ be a description of $\M,a$ in the language $\Lang_{KL}$
over $\Prop$. From $\M \not\models \neg \sigma$ and the soundness of
the KB + Commit calculus we get $\not\vdash_{\text{KB+Commit}} \neg \sigma$.  Now
suppose $\N$ is a (single agent) epistemic probability model $\N$ for
$\sigma$.

Now let ${\cal Y}$ be the result of applying to all members of ${\cal
  X}$ the permutation of $W$ that shifts each letter one position to
the right in the alphabet, and shifts $g$ back to $a$. Then 
\[
 {\cal Y} = 
 \{ \{f,g,a \}, 
    \{b,c,a \}, 
    \{ b,e,g \}, 
    \{ c,e,f \}, 
    \{ b,d,f \}, 
    \{ d,e,a \}, 
    \{ c,d,g \} \}. 
\]
This set has again the pairwise non-empty intersection property.
But notice that ${\cal X} \cap {\cal Y} = \emptyset$. 
In other words, no member of ${\cal Y}$ occurs in a neighborhood
in $\M$. 

Each world in $W$ occurs exactly three times in ${\cal X}$, and exactly 
three times in ${\cal Y}$. 

In $\N$ each member of ${\cal X}$ must get assigned probability
$> \frac 12$, and each member of ${\cal Y}$ must get assigned
probability $\leq \frac 12$.  Adding these values for the members of
${\cal X}$ gives
\[
  3(Pa + \cdots + Pg) > 4,
\]
and adding these values for the members of  ${\cal Y}$ gives
\[
 3(Pa + \cdots + Pg) \leq 4, 
\]
and contradiction. This shows that in the class of 
epistemic probability models $\models \neg \sigma$. 
But as we have seen $\not\vdash_{\text{KB+Commit}} \neg \sigma$. 
So $\neg \sigma$ is a formula that is valid on the class
of epistemic probability models but does not have a proof
in the KB calculus + Commit. 
\end{proof} 

Can we find an axiom that rules out the model from 
Example \ref{WFmodel}? For that, we borrow
from \cite{Segerberg1971:qpiams} and \cite{Lenzen2003:kbasp}.
Segerberg proposes a formula
\[
  (\phi_1, \ldots, \phi_n {\cal E}\psi_1, \ldots, \psi_n)
\]
to express that every accessible world belongs to exactly as many of
the $\phi$s as $\psi$s. We have a multi-agent
language, so we use $(\phi_1, \ldots, \phi_n {\cal
E}_a \psi_1, \ldots, \psi_n)$ to express that every $a$-accessible
world belongs to exactly as many of the $\phi$s as $\psi$s.

To see how this can be expressed as a $\Lang_\KB$ formula, 
let $\chi_i$ be the disjunction of all conjunctions of the form
\[
  d_1 \phi_1 \land  \cdots \land d_n \phi_n 
  \land e_n \psi_1 \land \cdots \land e_n \psi_n
\]
such that exactly $i$ of the $d$s and exactly $i$ of the $e$s are 
the empty symbol string and the rest of them are the negation sign
(with $1 \leq i \leq n$). Then $\chi_i$ expresses that exactly $i$ 
of the $\phi$ and exactly $i$ of the $\psi$ are true, and 
\[
  K_a (\chi_1 \lor \cdots \lor \chi_n)
\]
expresses 
\[
 (\phi_1, \ldots, \phi_n \ {\cal E}_a \ \psi_1, \ldots, \psi_n). 
\]
An axiom that must hold in epistemic probability models is
(see \cite{Lenzen2003:kbasp}): 
\begin{eqnarray*}
\text{(Len)} & & 
(\phi_1, \ldots, \phi_n {\cal E}_a \psi_1, \ldots, \psi_n)
\rightarrow \\ 
  & & (B_a \phi_1 \land \check{B}_a \phi_2
\land  \cdots \land\check{B}_a \phi_n) 
\rightarrow 
(B_a\psi_1 \lor \cdots \lor B_a \psi_n). \text{ for all } n \geq 1.
\end{eqnarray*}

\begin{theorem}
(Len) is not derivable in the KB + Commit calculus. 
\end{theorem}
\begin{proof}
Notice that (Len) fails in the committed epistemic neighborhood
model $\M$ from Example \ref{WFmodel}. For let $abc$ be shorthand
for $a \lor b \lor c$, and so on. Let $w$ be any world in $\M$. Then,
since every accessible world of $\M$ is true in exactly three members
of ${\cal X}$ and in exactly three members of ${\cal Y}$, we have:
\[
 \M, w \models (efg, abg, adf, bde, ace, cdg, bcf {\cal E} 
fga, bca, beg, cef, bdf, dea, cdg). 
\]
We also have: 
\[
   \M, w \models Befg \land Babg \land Badf \land Bbde \land Bace \land Bcdg 
   \land Bbcf, 
\]
since this expresses that all of ${\cal X}$ are neighborhoods. 
Therefore also: 
\[
   \M, w \models Befg \land \check{B}abg \land \check{B}adf \land \check{B}bde 
        \land \check{B}ace \land \check{B}cdg \land \check{B}bcf.
\]
But: 
\[
   \M, w  \not\models Bfga \lor Bbca \lor Bbeg \lor Bcef \lor Bbdf \lor Bdea \lor Bcdg, 
\]
since none of the ${\cal Y}$ are neighborhoods. 
Since KB + Commit is sound for committed epistemic neighborhood models, 
(Len) is not derivable in KB + Commit. 
\end{proof}

\begin{question} 
Does (Commit) follow from (Len), or can we find a epistemic neighborhood model
where (Len) holds but (Commit) fails? 
\end{question}
\XXXcomment{We should try to answer this!}

Let KB$^{+}$ be the result of adding axioms (Commit) and  (Len) to the KB calculus. 

\begin{theorem} \label{sound+EPM}
The calculus KB$^{+}$ is sound for the class of epistemic probability models.
\end{theorem}
\begin{proof}
The only thing we have to show is that (Commit) and (Len) hold in all epistemic probability 
models. The first of these follows from the fact that in all epistemic probability 
models $\M$, in all worlds $w$, for all lotteries $l$:
\[
  \text{ if } \M,w,l \models P_a \phi \geq \frac 1 2 \land 
      P_a (\neg \phi \land \psi) > 0
  \text{ then } \M, w, l \models  (\phi \lor \psi) > \frac 1 2. 
\]
Soundness of (Len) is easy to see, as follows. First notice that in any epistemic 
probability model $\M$ in any world $w$ it follows from 
\[
  \M, w \models (\phi_1, \ldots, \phi_n {\cal E}_a \psi_1, \ldots, \psi_n)
\]
that
\[ \tag{*}
   \M, w \models P_a \phi_1 + \cdots + P_a \phi_n = P_a \psi_1 + \cdots + P_a \psi_n. 
\]
Now assume also
\[
  \M, w \models B_a \phi_1 \land \check{B}_a \phi_2
      \land  \cdots \land\check{B}_a \phi_n. 
\]
It follows that
\[
  \M, w \models  P_a \phi_1 + \cdots + P_a \phi_n > \frac n 2. 
\]
From this and (*): 
\[
   \M, w \models  P_a \psi_1 + \cdots + P_a \psi_n > \frac n 2, 
\]
and therefore there has to be at least one $\psi_i$ with 
\[
  \M, w \models P_a \psi_i > \frac 1 2. 
\]
\end{proof}

\begin{theorem} \label{completeEPM}
The calculus KB$^{+}$ is complete for the class of epistemic probability models.
\end{theorem}

\begin{proof}
Assume (for simplicity) that there is just a single agent.

Assume $\not\vdash_{\text{KB}^{+}} \phi$. Use a canonical model 
construction to create a canonical committed
  epistemic neighborhood model $\M = (W,R,N,V)$ and a world $w$ in $\M$ for
  which $\M, w \not\models \phi$, and moreover, (Len) holds in $\M$. 

Suppose we want to construct an epistemic probability model from this. 
The neighborhood function for $\M$ gives us a list of linear equalities 
and inequalities for the lottery function $l$ in the probability model, as follows. 
For $X \subseteq W$ let $lX = \sum \{ l(w) \mid w \in X \}$.
If $X \subseteq W$ is such that $X \notin N(w), W-X \notin N(w)$, then 
\[
  lX = l(W-X),
\]
i.e., $\frac{lX}{lW} = \frac 1 2$. 
If $X \subseteq W$ is such that $X \in N(w)$
then 
\[
  lX > l(W-X),
\]
i.e., $\frac{lX}{lW} > \frac 1 2$. 
Assume that this set of linear (in)equalities does not have a solution. 
Then there are sets $X_1, \ldots, X_n, Y_1, \ldots, Y_n$ with 
\[
 X_1 \in N(w), \overline{X}_2, \ldots,  \overline{X}_1, \overline{Y}_1, \ldots, 
\overline{Y}_n \notin N(w), 
\]
\[
X_1 \ldots, X_n {\cal E} Y_1, \ldots, Y_n
\]
and 
\[
lX_1 + \cdots lX_n > lY_1 + \cdots + lY_n. 
\]
But this means that (Len) does not hold in $\M$, and contradiction. 

\XXXcomment{Proof to be completed. First spell out the single agent case in more
detail, next tackle the multi-agent case \ldots} 
\end{proof}

\section{Future and Related Work} \label{SectionFRW} 

\paragraph{Future Work} 

An obvious next step in the analysis would be the study of the relation between 
abstract and probabilistic models for conditional belief. 
For this, introduce an operator $\phi \Rel{a} \psi$ for $P_a \phi \leq P_a\psi$. 
Since $B_a \phi$ expresses $P_a\phi > P_a \neg \phi$, we can 
define  $B_a$ in terms of $\Rel{a}$ by means of $\neg (\neg \phi \Rel{a} \phi)$. 

A slight difficulty with the analysis of conditional belief is that 
$P_a \phi \leq P_a\psi$ cannot be expressed in $\Lang$, where all equality and inequality 
statements are linear. Terms on the lefthand side are of the form 
$q \cdot P_a \phi$, terms on the righthand side are of the form $q$. 
To express  $P_a \phi \leq P_a\psi$, one needs to extend the language $\Lang$ to 
polynomial (in)equalities. This is left for future work. 

There is also an obvious connection with Bayesian updating to be explored. 
In a model for epistemic probability logic, model restriction
corresponds to Bayesian updating, in the following precise sense.

If $\M = (W,V,R,L,n,E)$ is an epistemic probability model and 
$\phi$ a purely propositional formula, then let $U = \{ w \in W \mid 
  w \models \phi \}$. Let $\M^U$ be the restriction of
  the model to $U$. If $w \in U$ then: 
\[
  \M^U, w , i \models P \psi = a/b 
 \text{ iff } 
  \M,w,i\models P(\phi \land \psi) = a 
                   \land 
                  P(\psi) = b 
\]
Writing $P(\psi | \phi)$ for $\frac{P(\phi \land \psi)}{P\psi}$.  we
see that this is Bayesian updating.  In future work we hope to
investigate the corresponding operation in the associated epistemic
neighborhood models.

\paragraph{Related Work} 

\cite{KyburgTeng2012:tlorkr} investigates the logic of almost certain
belief, where the $\epsilon$ accepted statements are the statements 
$\phi$ for which the probability that $\neg \phi$ holds is at most 
$\epsilon$. It turns out that this set can be characterized as the 
neighborhood modal logic EMN. 

\cite{Herzig2003:mpbaa} 

\paragraph{Acknowledgements} 

Thanks to Alexandru Baltag, Jim Delgrande, Andreas Herzig and Sonja Smets for 
helpful comments and pointers to the literature. 

\bibliographystyle{alpha}
\bibliography{mybibAG,mybibHZ}  

\end{document} 
